\documentclass{amsart}
\usepackage{master}
\begin{document}
\title{Lecture 12\\Math 229}
\author{Lecture: Professor Pierre Simon\\Notes: Jackson Van Dyke}
\date{February 28, 2019}
\maketitle

The combinatorial portion of the course is now more-or-less over. Now we will start
discussing the automorphism group. This will deal with topological
dynamics and descriptive set theory.

\section{Kechris-Pestov-Todorcevic correspondence}
% {section 10}

Let $M$ be a homogeneous structure (locally finite, or just relational).
Recall that $M$ is Ramsey iff for every universal theory $T'$
in a language $L'\supeq L$ consistent with $\theory\left( M \right)$, $M$ has a definable
expansion to a model of $T'$.

Take such a $T'$, and let $M'$ be some expansion of $M$ to a model of $T'$ in $L'\supeq
L$. Now we have an action $\Aut\left( M \right)\acts M'$.
Another way of thinking of this is the following. For simplicity say $L' =
L\un\left\{ R \right\}$ for $R$ some $k$-ary relation. Then an expansion of $M$ to $L'$
can be thought of as an element of $2^{\left( M \right)^k}$.
Note that the set $S$ of elements of $2^{\left( M \right)^k}$ corresponding to
models of $T'$ is closed in the product topology. In particular, it is compact.

Now we have an action $\Aut\left( M \right)\acts 2^{\left( M \right)^k}$ and then the
action we really care about is the induced action $\Aut\left( M \right) \acts S$. 
This is well defined because the forbidden conditions defining $S$ are invariant under
automorphisms. Furthermore this action is continuous, i.e. the map
\begin{equation}
\Aut\left( M \right) \times 2^{\left( M \right)^k} \fromto  2^{\left( M \right)^k}
\end{equation}
is continuous. 
To see this let $V\subeq 2^{\left( M \right)^k}$ be a basic open, i.e. it depends only on a finite set
$\oline{a}\in M$. If $g\cdot \oline{n}\in V$,
then for any $\oline{n}'$ which coincides with $\oline{n}$ on $g^{-1}\left( \oline{a}
\right)$ and
for any $h$ such that $\restr{h}{g^{-1}\left( \oline{a} \right)} = \restr{g}{g^{-1}\left(
\oline{a} \right)}$ we have $h\cdot \oline{n}'\in V$.

So we have a continuous action on a compact space $S$. Now we can restate the criterion
from above as follows:
\begin{center}
$M$ is Ramsey iff for all $T'$, the action $\Aut\left( M \right)\acts S$ has a fixed
point.
\end{center}
Now we might wonder what is so special about $S$, and as it turns out the only special
thing is that it's compact. 
To see this we need some definitions. 

\begin{defn}
Let $G$ be a topological group. Then a $G$-flow is a compact Hausdorff space $X$ along with a
continuous action $G\acts X$.
\end{defn}

\begin{thm}[KPT]
If $M$ is as above, then $M$ is Ramsey iff all $\Aut\left( M \right)$ flows have a fixed
point.
\end{thm}

\begin{proof}
$\left( \converse \right)$: See above.

$\left( \implies \right)$: Assume $M$ is Ramsey and all substructures are
rigid.\footnote{This is certainly true if we have a linear order.} Let $G = \Aut\left(
M \right)$, and let $X$ be a $G$-flow. Assume $X$ does not have a fixed point.
Now we want to use compactness to find finitely many elements of the group which move all
of the points. First we recall the following lemma:
\begin{lem}[Urysohn]
If $X$ is compact Hausdorff, and $A,B\subeq X$ are two disjoint, closed subspaces, then
there is a continuous function $f: X\fromto \left[ 0,1 \right]$ such that $\restr{f}{A} =
0$ and $\restr{f}{B} = 1$. 
\end{lem}

By assumption, for all $x\in X$, there is $f: X\fromto \left[ 0,2 \right]$ and $g\in G$
such that $f\left( x \right) = 0$ and $f\left( gx \right) = 2$. 
Given such $f$ and $g$, the set:
\begin{equation}
U_{f,g} = \left\{ x\in X \st \abs{f\left( x \right)  -f\left( gx \right)} > 1 \right\}
\end{equation}
is open. By compactness, there are $f_1 , \cdots , f_n$ and $g_1 , \cdots , g_n\in G$
such that for all $x\in X$ there is $i\leq n$ such that
\begin{equation}
\abs{f_i\left( x \right) - f_i\left( g_i x \right)}  > 1 \ .
\end{equation}
Define
\begin{equation}
\begin{tikzcd}
X\arrow{r}{F}& \RR^2\\
x\arrow[mapsto]{r}&\left( f_1\left( x \right) , \cdots , f_n\left( x \right) \right)
\end{tikzcd}
\end{equation}
so for all $x\in X$ there exists $i\leq n$ such that
\begin{equation}
\norm{F\left( x \right) - F\left( g_i x \right)}_\infty > 1 \ .
\end{equation}
Now notice that $F$ is continuous and $X$ is compact, so the image of $F$ in $\RR^n$ is
compact. So now we can discretize the image into finitely many blocks each with diameter
less than $1/3$. I.e. we can write
\begin{equation}
F\left( X \right) = Y_1 \un \cdots \un Y_n
\end{equation}
where the $Y_i$ have diameter $< 1/3$. 

Now we claim there is some neighborhood $V\subeq G$ of the identity 
such that for all $g\in V$ and $x\in X$ 
\begin{equation}
\norm{F\left( x \right) - F\left( gx \right)} < 1/3 \ .
\end{equation}
This is because $F$ is continuous and $X$ is compact.
There is a finite\footnote{We are using the locally finite condition to get this to be
finite.} substructure $A\subeq M$ such that
\begin{equation}
\left\{ g\in G \st \restr{g}{A} = \restr{\id}{A} \right\}\subeq V \ .
\label{eqn:this}
\end{equation}
In particular, if $\restr{g}{A} = \restr{h}{A}$, then $\norm{F\left( x \right) - F\left(
h^{-1} g x \right)} < 1/3$, so for all $x\in X$, 
\begin{equation}
\norm{F\left( g^{-1} x \right) - F\left( h^{-1} x \right)} < \frac{1}{3} \ .
\end{equation}
The idea is that if two things do the same thing on $A$, then their image is very close. 
So they almost do the same thing even after applying $F$.

Fix $x_0\in X$, and consider a copy $A'$ of $A$ in $M$. 
Now we want to color these copies of $A$. Let $c\left( A' \right)$ be the minimal $j\leq
m$ for which there is $g\in G$ such that $g\left( A \right) = A'$ with $F\left( g^{-1} x_0
\right)\in Y_j$. 
What we're really coloring is cosets of
\eqref{eqn:this} (which is equivalent to
a coloring of copies of $A$ in $M$).

Now we want to construct some substructure $B$ which contradicts Ramsey.
So let $B$ be the structure generated by
\begin{equation}
A\un \bun_{i\leq n} g_i^{-1}\left( A \right) \ .
\end{equation}
Since $M$ is Ramsey, there is $B'\subeq M$ a copy of $B$ such that all copies of $A$ in
$B'$ have the same color $j$. 

Let $g_* \in G$ map $B$ to $B'$, and let $x_* = g^{-1}_*\left( x_0 \right)$. 
Now we show that 
\begin{equation}
\norm{F\left( g_i x_* \right) - F\left( x_* \right)} < 1
\end{equation}
for all $i\leq n$.
So somehow this $x_*$ is not moved much. This will be the contradiction.
Set $g_0 = \id$. For $i\leq n$ we have
that $g_* g_i^{-1}\left( A \right)$ is a copy of $A$ in $B'$, so it has color $j$. 
By the definition of the color, there is some $h_i\in G$ such that 
\begin{align}
\restr{h_i}{A} = \restr{g_* g_i^{-1}}{A}
&&
F\left( h_i^{-1} x_0 \right)\in Y_j \ .
\end{align}
So now we just have to compute. We know that
\begin{equation}
1/3 > \norm{F\left( h_i^{-1} x_0 \right) - F\left( g_i g_*^{-1} x_0 \right)} 
= \norm{F\left( h_i^{-1} x_0 - F\left( g_i x_* \right) \right)} \ .
\label{eqn:fe}
\end{equation}
Now we have that:
\begin{align}
\norm{F\left( g_i x_* \right) - F\left( x_* \right)}&\leq 
\norm{F\left( g_i x_* \right) - F\left( h_i^{-1} x_i \right)}
\\ & \qquad
+ \norm{F\left( h_i^{-1} x_0 \right) - F\left( h_0^{-1} x_0 \right)}
+ \norm{F\left( h_0^{-1} x_0 \right) - F\left( x_* \right)} \ .
\label{eqn:be}
\end{align}
On the RHS of \eqref{eqn:be}, the first and third terms are less than $1/3$ because of
\eqref{eqn:fe}, and the middle term is less than $1/3$ since they are both in $Y_j$. So the
entire RHS is less than $1$.
\end{proof}

\subsection{Some definitions}

Let $G$ be a topological group.

\begin{defn}
A topological group $G$ is \emph{extremely amenable} if all $G$-flows have a fixed point.
\end{defn}

\begin{defn}
A $G$-flow $X$ is minimal if every orbit is dense.
Equivalently there is no proper non-empty $G$-invariant closed $Y\subeq X$. 
Such a $Y$ is called a subflow.
\end{defn}

\begin{rmk}
Every $G$-flow has a minimal subflow (using Zorn's lemma) so
$G$ is extremely amenable iff its only minimal flow is $\left\{ \pt \right\}$.
\end{rmk}

Next time we will define a universal flow
which measures how far you are from being extremely amenable, and somehow tells us
something about a minimal Ramsey expansion.

\end{document}

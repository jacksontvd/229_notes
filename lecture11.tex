\documentclass{amsart}
\usepackage{master}
\begin{document}
\title{Lecture 11\\Math 229}
\author{Lecture: Professor Pierre Simon\\Notes: Jackson Van Dyke}
\date{February 26, 2019}
\maketitle

\section{Ramsey expansions}
% section 9

The general problem that people in this area are interested in is the following:

\begin{qn}
Given a \Fraisse class, or equivalently a homogeneous structure, find a minimal Ramsey
expansion.
\end{qn}

\begin{exm}
For the class of graphs, the minimal Ramsey expansion is the class of ordered graphs.
We will soon see a more general way to see that this is in fact minimal.
\end{exm}

\subsection{Criterion for being Ramsey}

\begin{prop}
An $\om$-categorical structure $M$ (homogeneous in a relational language $L$)
is Ramsey iff for every universal theory $T'$
in a language $L'\supeq L$ consistent with $\theory\left( M \right)$, $M$ has a definable
expansion to a model of $T'$.
\end{prop}

\begin{proof}
$\left( \implies \right)$: Assume $M$ is Ramsey and $T'$ as above. Let $M'$ be an
expansion of $M$ to a model of $T'$. 
By compactness we can assume $L'\minus L$ is finite. 
Now we want to color the substructures (subsets since relational) of $M$ according to how
they expand to $L'$. Since $M$ is Ramsey, for every finite substructure $A\subeq M$, and every 
$B\subeq M$, there is some $C$ such that there is a copy $B'$ of $B$ in
$C$ such that all copies of $A$ in $B'$ have the same expansion to $L'$.
Since this is how we colored copies of $A$.

In particular, we can find a homogeneous copy of $B$ in $M$. And then by compactness, in
an elementary extension $M^*$ of $M$
we can find a copy $M_0$ of $M$ where all of the copies of $A$ in $M_0$ have the same
expansion to $L'$. This implies that $L'$ is invariant under $L$-automorphisms. 
So by $\om$-categoricity, the expansion to $L'$ is $0$-definable. 
Therefore $M_0\models T'$ as $T'$ is universal.

$\left( \converse \right)$: Assume the conclusion. 
Let $A$ and $B$ be given, and write $r$ for the number of colors. Then consider
\begin{equation}
L' = L\un \left\{ R_1\left( \oline{x} \right) \cdots R_r\left( \oline{x} \right) \right\}
\end{equation}
where $\abs{\oline{x}} = \abs{A}$. Then we want to show $M\fromto \left( B \right)_r^A$.
A coloring of copies of $A$ in $M$ gives an expansion of $M$ to $L'$.
Let $T'$ be its universal theory. By assumption, there is a definable expansion $M^*$ of
$M$ to a model of $T'$. Since $M^*$ models the universal theory of $M$,
$M^*\models T_{\forall}\left( M \right)$, we know
there is a copy of $B$ in $M$, which is colored as some/any copy
of $B$ in $M^*$. On that copy of $B$, all copies of $A$ have the same color.
\end{proof}

This proposition means that in order to prove something isn't Ramsey, all we have to do is
show it doesn't have such a definable expansion.

\begin{cor}
A Ramsey structure has a definable linear order.
\end{cor}

\begin{proof}
Take $L' = L\un \left\{ \leq \right\}$, and then we can take $T'$ to be the theory of linear orders,
which is universal.
\end{proof}

\subsection{Examples}

\begin{exm}
Take $L = \left\{ E \right\}$ to be equivalence relation.
If we consider $T'$ in $L' = L\un \left\{ \leq \right\}$ where $\leq$
is a linear order with convex classes, we see that ordered equivalence relations are not
Ramsey since the \Fraisse limit of that class would have all classes dense and does not
induce an order on the quotient. And without this then we definitely don't have a
definable order with convex classes. So this is not Ramsey. 

It is true however that the class of convexly ordered equivalence relations is Ramsey.
\begin{exr}
Show that this is Ramsey.
[This does not follow from the theorems we have seen. This should be done by hand.]
\end{exr}
\end{exm}
\begin{exm}
Consider circular orders. Recall $\left( \QQ , C\left( x,y,z \right) \right)$ is a reduct of DLO. 
A Ramsey expansion must define an order of which $C$ is the corresponding
circular order (this is a universal condition). So the minimal Ramsey expansion 
is just DLO.
\end{exm}
\begin{exm}
We can also look at partial orders. 
Any Ramsey expansion of partial orders will have to define an order which extends the
partial orders. We claim this class is Ramsey. Take $L' = \left\{ \leq , \order \right\}$
where $\order$ is a linear extension of $\leq$.
The forbidden substructures are as follows. First we cannot have $a\geq b$, $a\order b$
for $a\neq b$, and the second is that $a_1 \leq a_2 \leq \cdots\leq a_n$ and $\neg \left(
a_1\leq a_n \right)$ for all $n$.

Now we apply the theorem. So we need a Ramsey class. 
We claim that if only the first is satisfied, then this is the class of ordered graphs. 
So for the Ramsey class we take $\cR$ to be the class of structures in $\left\{ \leq ,
\order\right\}$ with $a\leq b\implies a\order b$ and $\order$ is a linear order. This is
the same thing as ordered graphs, and hence is Ramsey.
Let $\cK\subeq R$ be the subclass where $\leq$ is transitive. 
It is clear that it is hereditary. Then we need to check that it has
strong (disjoint) amalgamation. This is clear because, e.g. if $a\leq b$ and $b\leq c$, we
just insist that $a\leq c$, and then amalgamate this arbitrarily to a linear order. 
Now we need to check that it is locally finite. Let $C_0\in \cR$. Then we can take $n =
\abs{C_0}$. Now let $C$ be an $L'$-structure such that:
\begin{itemize}
\item $C_0$ is a completion of $C$, i.e. there is
a homomorphism embedding $\pi:C\fromto C_0$, and
\item every substructure of size $\leq n$ of $C$ has a strong $\cK$-completion.
\end{itemize}
Concretely this means the following. The only situation where we don't have a strong
$\cK$-completion is if we have something like
\begin{align}
a_1 \leq a_2 \leq \cdots \leq a_k &&
a_1 \order a_k &&
\neg\left( a_1 \leq a_k \right)
\label{eqn:forbidden}
\end{align}
since we can't add the `edge' between $a_1$ and $a_k$.
Now we have to check there is a bound on $k$.
So consider some situation as in \eqref{eqn:forbidden}. Then $\pi$ has to be injective on
$\left\{ a_1 , \cdots , a_k \right\}$ because we must have
\begin{equation}
\pi\left( a_1 \right) \sorderl \pi\left( a_2 \right) \sorderl \cdots \sorderl \pi\left( a_k \right)
\end{equation}
so $k\leq n$ and the theorem tells us this is a Ramsey class.
\end{exm}
\begin{exm}
A Ramsey expansion of a (nontrivial) reduct of the random graph is just given by the
ordered random graph, and this is minimal in the sense that any Ramsey expansion must have
a $0$-definable graph whose reduct is the given one.
\end{exm}
\begin{exm}
If we take $L = \left\{ \leq_1 , \leq_2 \right\}$ and the class $\cC$ of finite
structures where $\leq_1$ and $\leq_2$ are linear orders, then $\cC$ is Ramsey. 
The \Fraisse limit is two linear orders with no link between them.
The fact that this is Ramsey is something which we can't see from the theorem since it is
not locally finite.
\begin{exr}[*]
Prove this. 
[Hint: Think of such structures as two linear orders with a bijection between them.]
\end{exr}
\end{exm}

\end{document}

\documentclass{amsart}
\usepackage{master}
\begin{document}
\title{Lecture 5\\Math 229}
\date{February 5, 2019}
\author{Lecture: Professor Pierre Simon\\Notes: Jackson Van Dyke}
\maketitle

Today we will do something fun and somewhat unrelated to the previous topics.

\section{$\om$-categorical groups}

Let $G$ be an infinite $\om$-categorical group.
Recall this means there are only finitely many $n$-types for any $n$.
Note the following immediate observations:

\begin{obs}
$G$ is locally finite.\footnote{
That is, every finitely generated subgroup is finite.}
\end{obs}

\begin{proof}
This is true because if $\oline{a}\in G$ is a finite set, 
then the elements of $\lr{\oline{a}}$ have distinct types.
\end{proof}

In fact, $G$ is uniformly locally finite. In particular there is some $n$ such that
for all $g\in G$, $g^n = e$. In this case we say that $G$ is of \emph{bounded exponent}.

\begin{obs}
If $G$ is abelian and of bounded exponent, then the pure group $\left( G , + \right)$ is
$\om$-categorical.
\end{obs}

\begin{proof}
First we have the following:
\begin{fact}
An abelian group of bounded exponent is a direct sum of cyclic groups of prime power
orders.
\end{fact}
\begin{exr}
Prove this. This is effectively a group theory question.
\end{exr}

Now we need to prove that such groups are $\om$-categorical.
To prove this it suffices to find sentences that completely characterize one given group
in this class:
\begin{equation}
G = \bdsum_{p,n} \left( \ZZ / p^n \ZZ \right)^{s_{p^n}} \ .
\end{equation}
In other words, we need to find formulas that give the $s_{p^n}$s.

For this, we use Ulm invariants. Let
\begin{equation}
P_{p,q} = \left\{ x\in G \st px = 0 \right\}\cap p^k G
\end{equation}
and
\begin{equation}
f_p\left( k-1 \right) = \dim\left( P_{p,k-1} / P_{p,k} \right)
\end{equation}
where we view these quotients as $\FF_p$ vector spaces.
Note that
\begin{equation}
s_{p^k} = f_p\left( k-1 \right) \in \NN\un\left\{ \infty \right\} \ .
\end{equation}
This is first-order definable, so we are done.
\end{proof}

So the abelian case is settled. The nonabelian case is much more interesting, and in some
ways open.

\begin{fact}
An infinite locally finite group has an infinite abelian subgroup.
\end{fact}

\begin{wrn}
The proof of this is not an exercise.
It is a few pages long.
\end{wrn}

\begin{cor}
An infinite $\om$-categorical group has a subgroup isomorphic to $\FF_p^\om$
for some $p$.
\end{cor}

\begin{thm}[MacPherson]
A finitely homogeneous structure does not interpret an infinite group.
\end{thm}

The idea of the proof is the following.
Why is $\FF_p^\om$ not finitely homogeneous?
We can take a basis $a_1 , \cdots , a_n , \cdots$ and consider the tuples
$\left( a_1 , \cdots , a_n , \sum_{i = 1}^n a_i \right)$
and $\left( a_1 , \cdots , a_{n+1} \right)$. The types of these are not the same, 
but the type of any two tuples of length $n-1$ inside these
is the same, which is a contradiction.

\begin{proof}
Let $G$ be definable in a finitely homogeneous structure $\cM$,
and let $V\subeq G$ be an infinite $\FF_p$ vector space in $G$.
Now we want to apply the Ramsey argument.
Fix a linear order $<$ on $\FF_p$, and fix a basis
$\left( v_i \st i < \om \right)$ of $V$.
Now we want to find a sub-vector space which is somehow homogeneous, so all bases have 
the same type.

If $W\subeq V$ is a finite dimensional subspace, 
define the canonical basis of $W$ as the basis whose matrix in the $v_i$s is in
row-reduced form. This identifies $W$ uniquely.
Let $t$ be the maximal arity of the (finite relational) language.
We color subspaces of $V$ of dimension $t$ according to the type of the canonical basis. 

\begin{fact}
For all $r,t,k < \om$, and finite field $\FF$, there is some $N$ such that any coloring of
$t$-dimensional subspaces of $\FF^N$ in $r$ colors has a $k$ dimensional
homogeneous\footnote{This means all $t$ dimensional subspaces have the same color.}
subspace.
\end{fact}

Using the fact, let $W\subeq V$ be a $t+1$-dimensional homogeneous subspace. Let $\left(
u_0 , \cdots , u_t \right)$ be its canonical basis. Consider the tuples:
\begin{align}
w_1 = \left( u_0 , \cdots , u_{t-1} , u_0 + \cdots + u_{t-1} \right)
&&
w_2 = \left( u_0 , \cdots , u_{t-1} , u_0 + \cdots + u_t \right) \ .
\end{align}
Note that clearly $\type w_1 \neq \type w_2$. 
However, if we restrict to a subtuple of size $t$, say
\begin{align}
s_1 &= \left( u_0 , \cdots , \hat u_j , \cdots , u_{t-1} , u_0 + \cdots + u_{t-1}
\right)\\
s_2 &= \left( u_0 , \cdots , \hat u_j , \cdots , u_{t-1} , u_0 + \cdots + u_t \right)
\end{align}
where we have omitted $u_j$, then we claim $\type s_1 = \type s_2$.
The canonical basis of $\Vect \left( s_1 \right)$ is $u_0 , \cdots , u_{t-1}$
and the canonical basis of $\Vect\left( s_2 \right)$ is
$u_0 , \cdots , u_j + u_t , u_{j+1} , \cdots , u_{t-1}$.
By homogeneity of $W$, they have the same type.
Now $s_1$ and $s_2$ are expressed by the same term in their respective bases, so they have
the same type. This contradicts the fact that the language has arity $t$.
\end{proof}

\section{Structure of $\om$-categorical groups}

\begin{thm}
Let $G$ be an $\om$-categorical group. Then $G$ has a finite series 
\begin{equation}
1 < G_0 < G_1 < \cdots < G_n = G
\end{equation}
where each $G_i$ is characteristic\footnote{This is a subgroup invariant under all
automorphisms of the group. This certainly means the subgroup is normal, but it also means
it is definable. 
In particular, if the group has no other structure, this is the same as a definable subgroup.}
and each $G_{i+1} / G_i$ is either:
\begin{enumerate}[label = (\iii)]
\item abelian,
\item isomorphic to $P^R$ (where $P$ is a finite simple group, and $R$ is either a finite
number, or it is the boolean ring $V_0$ or $V_1$), or
\item an infinite $\om$-categorical, non-abelian, characteristically simple $p$-group.
\end{enumerate}
\end{thm}

Now we explain some aspects of the theorem.
First note that a finite non-abelian characteristically simple group is of the form $P^n$,
for $P$ a simple group and $n < \om$.
Now we explain these groups defined using boolean rings.
For $G$ any finite group, the structure of $G^{V_0}$, an infinite $\om$-categorical group, is the following:
\begin{itemize}
\item Its universe consists of continuous maps from $K$ (Cantor space) to $G$.
Or equivalently, denoting by $\cB$ the unique countable atomless boolean algebra, such a
map is given by a finite sequence of the form $\left( B_1 , g_1 \right) \cdots \left( B_n,
g_n\right)$ for $g_i\in G$ where $\left( B_1 , \cdots , B_n \right)$ is a partition of $\cB$.
\item Multiplication is defined coordinate-wise on $K$.
\end{itemize}

For $G^{V_1}$, instead of taking Cantor space, we remove a point. 
So we have a locally compact space. 
Then we look at maps $K\minus \left\{ * \right\}\fromto G$
which are compactly supported, i.e. near $*$ it is the identity.

\begin{exr}
Convince yourself that these groups are not elementarily equivalent.
\end{exr}

\begin{comment}
\begin{sol}
Commute elements.
{1:03}
One element with abelian commutator group.
\end{sol}
\end{comment}

\begin{exr}[*]
If $G$ is not abelian, then the infinite product $G^\om$ 
is never $\om$-categorical. [Hint: Show there are infinitely many $1$-types
by counting the number of conjugates.]
\end{exr}

Now we discuss the third case in the theorem.
It is not known if such a thing exists.
In fact, it is conjectured that there is no such object.
It is worth noting that infinite $p$-groups can be quite complicated.
For example, there are these Tarski monsters, which are infinite groups all of whose
proper subgroups have order $p$.

\begin{thm}[Olshanskii]
Tarski monsters exist and are simple.
\end{thm}

\end{document}

\documentclass{amsart}
\usepackage{master}
\usepackage{overlay}
\begin{document}
\title{Lecture 8\\Math 229}
\author{Lecture: Professor Pierre Simon\\Notes: Jackson Van Dyke}
\date{February 14, 2019}
\maketitle

\section{EPPA}

The methods we saw last time for proving EPPA work for a large number of classes. 
However, there are some
classes where these methods don't work. For example they do not work for
groups.\footnote{This is expected, since we shouldn't be able to prove EPPA for groups
using only combinatorics.}
Note that EPPA was proven for groups via other means.
The methods also fail for equivalence relations, but this can be solved by adding unary
functions. They also fail for binary functions, but we do not know how to fix
this. For example, consider the reduct of the random graph with an extra automorphism
switching edges and non-edges. 

They also fail if there is no canonical way to complete a partial structure like we did
with metric spaces. A notorious example of this is tournaments. This is perhaps the
biggest open problem in this area.

\section{There's order everywhere: Ramsey theory}

Recall Ramsey's theorem:

\begin{thm}[Ramsey]
For all $k,c,m < \om$, there is some $N < \om$ such that
\begin{equation}
N\fromto \left( m \right)_c^k \ .
\end{equation}
\end{thm}

We now explain this notation. Let $\left[ m \right]^k$ denote the set of $k$-subsets of $m$.
Then the notation $\left( m \right)_c^k$ means that for every coloring
$f: \left[ N\right]^k \fromto c$ of $N$ there is a map $g:m\inj N$ which is homogeneous, i.e. $f$
is constant on $\left[ \left\{ g\left( 1 \right) , \cdots , g\left( m \right) \right\}
\right]^k = \left[ g\left[ m \right] \right]^k$.

Note there is an infinite version which says
\begin{equation}
\om \fromto \left( \om \right)^k_c \ .
\end{equation}
We can recover the finite one from the infinite one by some sort of compactness. In
particular, it certainly tells us that $\om \fromto \left( m \right)^k_c$ but
propositional compactness gives us that there is an $N$ such that $N\fromto\left( \om
\right)_c^k$. The opposite however doesn't work.

\subsection{Notation}

For $A$ and $B$ two $L$-structure, let $\binom{B}{A}$ consist of the copies of $A$ in $B$.
For now this could mean two things, either subsets isomorphic to $A$, or embeddings of $A$
in $B$. There might be more embeddings, but we will resolve this ambiguity by restricting
to certain structures shortly.
Now define $C\fromto\left( B \right)^A_r$ where $r$ is the number of colors. 
This means the following. For every coloring $f:\binom{C}{A}\fromto r$, there is a copy of
$B$ in $C$, $g\in \binom{C}{B}$, which is homogeneous.

\subsection{Resolving the ambiguity}

\begin{exm}
Let $A$ be the graph $
\begin{cd}
\point \arrow[dash]{r}\arrow[dash]{d}&\point \\ \point & 
\end{cd}$
and $b$ be $
\begin{cd}
\point \arrow[dash]{r}\arrow[dash]{d}&\point \arrow[dash]{d} \\ \point \arrow[dash]{r} & \point
\end{cd}$.
Then we claim there is no $C$ such that $C\fromto \left( B \right)_2^A$ where we interpret
$\binom{B}{A}$ as denoting subsets of $B$ isomorphic to $A$. First notice there are four
copies of $A$ inside $B$.
\begin{proof}
Take any $C$, fix a linear order $\leq$ on it, and color a copy of $A$ black if the middle
point is maximal, and white otherwise.
\end{proof}
If we try embeddings here, then it isn't going to work. Let $A$ be the same as before, and
now choose $B$ to be the same as $A$. Then this has two embeddings of $A$ into it.
Number the points of $A$.
Let $C$ be any graph, and fix a linear order $\leq$ on $C$. Color an embedding of $A$
black if $\pt \ 1 < \pt \ 3$ and white otherwise.
\end{exm}

We can conclude the following from this:
\begin{enumerate}
\item embeddings don't work as soon as $A$ has a nontrivial automorphism,
\item subsets don't seem to work so well either, because they already don't work for the
above sort of example.
\end{enumerate}

\begin{sol}
Only consider rigid structures. I.e. they have no nontrivial automorphisms.
In fact from now on (in this section) we will only consider structures with a linear
order.
\end{sol}

Now we have that embeddings and subsets are the same, so the ambiguity is resolved.

\section{Ne\v{s}et\v{r}il-R\"odl theorem}

\begin{thm}[Ne\v{s}et\v{r}il-R\"odl,Abramson-Harrington]
Let $A$ and $B$ be finite ordered graphs and $r<\om$.
Then there is $C$ a finite ordered graph
such that $C\fromto \left( B \right)_r^A$.
\end{thm}

First we will prove a baby case.\footnote{This won't be that instructive since it won't
generalize to the full theorem, but it's still a good example.}
Let $A = \left\{ \point \right\}$ and $B$ be anything.

\begin{proof}[Proof of baby case]
Take $V\left( C \right) = V\left( B \right)^r$. Then we define the edges as follows.
Let $n = \abs{V\left( B \right)}$ and $r = 2$. So we have $n$ copies of $B$.
Now we can think of these copies as points, and connect them as they are connected in $B$.
So if we find one white point in each copy, we have a monochromatic copy, and if we can't
do this then one must be completely black so we also have a monochromatic copy of $B$.

More specifically, put an edge 
\begin{equation}
\left( \left( x_0 , \cdots , x_{r-1} \right) , \left( y_0 , \cdots , y_{r-1}
\right)\right)\in E\left( C \right)
\end{equation}
iff there exists $j<r$ such that
$x_i = y_i$ for $i < j$ and
$\left( x_j , y_j \right)\in E\left( B \right)$.
Now proceed by induction.

If for each $v\in V\left( B \right)$ there is a red $\left( v , x_1 , \cdots , x_{r-1}
\right)$ this gives a red copy of $B$. 
Otherwise for some $v$ 
\begin{equation}
\left\{ \left( v , x_1 , \ldots , x_{r-1} \right) \st x_i\in V\left( B \right) \right\}
\end{equation}
has $r-1$ colors.
\end{proof}

\subsection{Consequences of the theorem}

\begin{prop}[Ordering property for graphs]
Let $H$ be a finite graph with a linear order $\leq$.
Then there is a finite graph $G$ such that for any ordering $\order$ there is a copy of
$\left( H , \leq \right)$ in $\left(G , \order\right)$.
\end{prop}

\begin{proof}
Let $\left( H , \leq \right)$ be given. Enumerate $H = \left\{ h_1 , \cdots , h_n
\right\}$ and assume $h_i < h_{i+1}$ where $\left( h_i , h_{i+1} \right)\in E\left( H
\right)$. If this is not satisfied, then add a new vertex between every two consecutive
vertices of $H$. We can do this because if it's true for a larger graph, then this is
sufficient.
Define $\left( H^* , \order \right) = \left( H , \leq \right)\dsum\left( H , \geq \right)$
and let $\left( G , \orderr \right)\fromto \left( H^* \right)_2^e$ for
$e=\begin{cd}\point\arrow[dash]{r}&\point\end{cd}$ an edge.

Let $\orderrr$ be any order on $G$. Color edges of $G$ according to whether the two orders
coincide or not on the edge. Then we get a homogeneous copy of $\left( H^* , \order \right)$
inside $\left( G , \orderr \right)$.

If colored edges of $H^*$ ``coincide'' then the left copy of $H$ in $H^*$ has the
correct order in $\left( G , \orderrr \right)$. Otherwise, the right copy does.
\end{proof}

\begin{prop}
If $A$ is a finite graph which is neither a complete or empty graph, then there is a $B$
such that for no graph $C$ do we have $C\fromto \left( B \right)_2^A$ in the sense of
subsets.
\end{prop}

\begin{proof}
$A$ admits two non-isomorphic order $\leq_1$ and $\leq_2$.
Take $B$ to be a finite graph such that for any order on $B$ this contains an isomorphic
copy of $\left( A , \leq_1 \right)$ and $\left( A , \leq_2 \right)$.
Let $C$ be any graph with fixed order $R$. Color a copy of $A$ black if the induced
order is isomorphic to $\leq_1$, and white otherwise. But now there cannot be a homogeneous
copy of $B$ inside this.
\end{proof}

Note that we cannot hope to prove a Ramsey statement without a linear order or something
from which a linear order follows.
We will see other arguments which show this later.

\section{Ramsey structures}

\begin{defn}
Let $\cC$ be a (Fra\"{i}ss\'{e}) class. Then $\cC$ has the \emph{Ramsey property} if for all
$A,B\in \cC$, and for all $r < \om$, there is $C\in \cC$ such that $C\fromto \left( A
\right)_r^B$.
\end{defn}

\begin{defn}
A homogeneous structure $M$ is Ramsey if its age has the Ramsey property.
\end{defn}

\begin{exm}
The random graph is not Ramsey, however the random ordered graph is a Ramsey expansion of
it.
\end{exm}

This is an example of a minimal Ramsey expansion.
For a given structure, the game for Ramsey theorists is to either prove it has the Ramsey
property, or find its minimal Ramsey expansion.

\section{Comparing EPPA and Ramsey}

There are somehow three things people look for here. EPPA, Ramsey, and big Ramsey.
\begin{center}
\begin{tabular}{|c|c|c|}
\hline
EPPA & Ramsey & big Ramsey\\
\hline\hline
forbids a linear order 
&needs a linear order&-
\\
\hline
less known
&well known\footnotemark&
not well known\\
\hline
\end{tabular}
\footnotetext{For almost all classical structures.}
\end{center}
there is some idea that there should be a unifying method of proof for all of these
things.

Recall that EPPA was a generalized version of amalgamation in the sense that together with
JEP it implies AP.
As it turns out, we also
have that $\cC$ being Ramsey (and having JEP) implies that $\cC$ has AP.

The idea is that we somehow start with the left figure in \cref{fig:ap}.
Then by JEP we can find some $C$ as in the right half of \cref{fig:ap}.
\begin{figure}
\begin{overlay}
\pict{AP1.pdf}{0.2}
\toptext{$B_1$}{1.7,0.7}
\toptext{$B_2$}{1.7,-0.7}
\toptext{$A$}{-0.6,0}
\end{overlay}
\qquad $\leadsto$ \qquad
\begin{overlay}
\pict{AP2.pdf}{0.2}
\toptext{$B_1$}{1.7,0.7}
\toptext{$B_2$}{1.7,-0.7}
\toptext{$C$}{-1.7,1}
\toptext{$A$}{-0.6,0.5}
\toptext{$A$}{-0.6,-0.5}
\end{overlay}
\caption{Starting with the situation on the left, JEP can give us the situation on the
write for some $C$ in the \Fraisse class.}
\label{fig:ap}
\end{figure}
Now we can find some $D$ such that $D\fromto \left( C \right)_2^A$.
Color copies of $A$ in $D$ according to whether they extend to a copy of $B_1$
inside of $D$. 

So the Ramsey property is somehow also a generalized version of amalgamation.

\subsection{Big Ramsey}

The idea of satisfying big Ramsey is that we can find 
a homogeneous copy of the \Fraisse limit, i.e. the infinite structure, in a
coloring.

\begin{fact}
If $R$ is the random (ordered) graph, then $R\not\fromto \left( R \right)^e_2$ for $e$ an edge.
\end{fact}

In other words big Ramsey is false for the random ordered graph.
What is true however is that one can always find a copy that has at most, say, $d$ colors.
In this case we would say this has finite big Ramsey degree $d$.
Note that big-Ramsey implies Ramsey, but finite-big Ramsey does not on the face of
it imply Ramsey. It does however turn out that the methods used to find a
finite-big Ramsey degree typically also yield the Ramsey property.

\end{document}

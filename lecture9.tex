\documentclass{amsart}
\usepackage{master}
\begin{document}
\title{Lecture 9\\Math 229}
\author{Lecture: Professor Pierre-Simon\\Notes: Jackson Van Dyke}
\date{February 29, 2019}
\maketitle

\section{Proof of Ne\v{s}et\v{r}il-R\"odl theorem}

Recall the statement is:

\begin{thm}[Ne\v{s}et\v{r}il-R\"odl,Abramson-Harrington]
Let $A$ and $B$ be finite ordered graphs and $r<\om$.
Then there is $C$ a finite ordered graph
such that $C\fromto \left( B \right)_r^A$.
\label{thm:nr}
\end{thm}

\subsection{Hales-Jewett theorem}

First we need to know the Hales-Jewett theorem.\footnote{
A mnemonic for remembering this is that it somehow says that for a fixed board size, 
as long as you are playing in a sufficiently large dimension, a tic-tac-toe
game will never be a draw.}

Consider a finite alphabet $A$ with $n$ letters. A \emph{combinatorial line} is a set
generated by a word in $A\un\left\{ x \right\}$ with at least one $x$.
Generate here means that the extra letter is replaced by each of the
possible actual letters. For example if $A = \left\{ a,b,c \right\}$, 
$xabacxx$ generates $aabacaa$, $babacbb$, and $cabaccc$.
It is always a set of size $n$ so a line has size $n$.
Note that the combinatorial lines in the usual tic-tac-toe board only comprise some of the
winning lines, it misses one of the diagonals.

\begin{exm}
For $A = \left\{ a,b,c \right\}$ we have:
\mathtabular{
\begin{tabular}{|C|C|C|}
\hline ca & cb& cc\\ \hline ba&bb&bc\\ \hline aa&ab&ac \\\hline
\end{tabular}
}
and then the combinatorial lines are all of the horizontal and vertical lines, and the
diagonal from the top right to bottom left, but not from the top left to the bottom right.
\end{exm}

\begin{thm}[Hales-Jewett]
For all $n,c < \om$ there is $H$ such that if words of length $H$ in $\left\{ 1 , \cdots ,
n \right\}$ are $c$-colored, there is a monochromatic combinatorial line.
\end{thm}

\subsection{Partite lemma}

We will not work with ordered graphs, we will work with the similar notion of a partite graph. 

\begin{defn}
A $k$-partite graph, is a graph $G$ along with a map $\pi:G\fromto k = \left\{ 0 , \cdots
, k-1\right\}$ such that edges are transversal. That is, there is no edge inside
a part.\footnote{A part is $\pi^{-1}\left( \left\{ \rho \right\} \right)$ for some number
$\rho$.}
\end{defn}

We say a $k$-partite graph $A$ is transversal if it has at most one vertex per part.
This case is exactly an ordered graph.
The main difference between these and ordered graphs is that 
an embedding of a partite graph is an embedding of graphs which has to respect the
parts in the strict sense of sending the $i$th part to the $i$th part.

\begin{lem}
Fix $k$, $c$ and let $A$ be a $k$-partite transversal graph, $\abs{A} = k$.
Let $B$ be any $k$-partite graph, then there is some $k$-partite graph $C$ such that
\begin{equation}
C\fromto \left( B \right)^A_c
\end{equation}
as $k$-partite graphs.
\end{lem}

\begin{Proof}
Assume $B$ contains copies of $A$, and that every vertex of $B$ is in a copy of $A$.
For an integer $d$, define the $k$-partite graph $C_d$ as follows. The vertices are
\begin{equation}
V\left( C_d \right) = V\left( B \right)^{\lr{d}}
\end{equation}
where this means we take $d$-tuples of elements of $V\left( B \right)$ all from the same
part. This is naturally a partite set.
Let $\overl{a},\overl{b}\in V\left( C_d \right)$. If for all $i\leq d$ we have
$\left( a_i , b_i \right)\in E\left( B \right)$, then $\left( \overl{a} , \overl{b}
\right)\in E\left( C_d \right)$. Similarly, if for all $i\leq d$ we have $\left( a_i ,
b_i\right)\not\in E\left( B \right)$ then $\left( \overl{a} , \overl{b} \right)\not\in
E\left( C_d \right)$. Otherwise, do the opposite of what $A$ does between the parts of
$\overl{a}$ and $\overl{b}$.
\begin{clm}
If $d$ is large enough, then $C_d\fromto \left( B \right)_r^A$.
\end{clm}
The idea is that we want to apply Hales-Jewett, and then a combinatorial line will be a
homogeneous copy of $B$. 
\begin{proof}
Enumerate the embeddings of $A$ in $B$ as $h_1 , \cdots , h_m$.
Our alphabet is $\left\{ h_i \right\}$, and the size of the board is $d$.
Define a map $g: \left[ m \right]^d \times A \fromto C_d$ by sending
\begin{equation}
\left\{ \left\{ \al_1 , \cdots , \al_d\right\},a\right\}
\mapsto \left( h_{\al_1}\left( a \right) , \cdots , h_{\al_d}\left( a \right) \right) \ .
\end{equation}
Then color $\left[ m \right]^d$ according to the color of the image.
Note that for a fixed $\oline{\al}\in \left[ m \right]^d$, the image $g\left( \oline{\al}
, A\right)$ is a copy of $A$ in $C_d$.

Now by Hales-Jewett, if $d$ is large enough, there is a monochromatic combinatorial line
in $\left[ m \right]^d$, say $\al_1 , \cdots , \al_d$ such that $\al_i\in \left[ m \right]\un
\left\{ x \right\}$. Now we claim this is a copy of $B$.

Say the line is $\al_1 xx \al_4$. Now we look at two different values $t$ and $s$ of $x$.
Then we have
\begin{align}
\barr{\al_0}\left[ t \right] = \bar{\al} = \al_1 tt \al_4: a&\mapsto
\left( h_{\al_1}\left( a \right) , h_t\left( a \right) , h_t\left( a \right) ,
h_{\al_4}\left( a \right) \right)
\\
\barr{\b_0}\left[ s \right] = \bar{\b} = \al_1 ss \al_4: a&\mapsto
\left( h_{\al_1}\left( a' \right) , h_s\left( a' \right) , h_s\left( a' \right) ,
h_{\al_4}\left( a' \right) \right) 
\end{align}
where $\al_0$ is a word in $\left[ m \right]\un \left\{ x \right\}$,
and $\barr{\al_0}\left[ t \right]$ replaces the instances of $x$ by $t$.
Now there is an edge between $g\left( \bar{\al} , a \right)$ and $g\left( \bar{\b} ,
a'\right)$ iff there is an edge between $h_t\left( a \right)$ and $h_s\left( a'
\right)$. 
Therefore, the map which sends $h\left( \bar{\al}\left[ t \right] , a \right)\mapsto h_t\left(
a \right)$ is an isomorphism between the image
\begin{equation}
\left\{ g\left( \barr{\al}\left[ t \right] , a \right)\st t\in \left[ m \right] , a\in A
\right\}
\end{equation}
and $B$. This copy of $B$ in $C_d$ is homogeneous, because each copy of $A$ in $B$
is the image of some $\barr{\al_0}\left[ t \right]$.
\end{proof}
By the claim we are finished.
\end{Proof}

\subsection{Partite construction}

\begin{proof}[Proof of \cref{thm:nr}]
Suppose we are given $A$ and $B$ two ordered graphs, and a number of colors $r$. We view
these as two transversal partite graphs of different sizes.
First we apply the usual Ramsey theorem to get $N$ such that
\begin{equation}
N\fromto \left( \abs{B} \right)^{\abs{A}}_r \ .
\end{equation}
Now we want to built an $N$-partite graph in a series of steps, the so-called pictures.

Picture $0$, $P_0$, is the disjoint union (free amalgamation/joint embedding) of
$\binom{N}{\abs{B}}$ copies of $B$ which respect the ordering of $B$.

Let $d = \binom{N}{\abs{A}}$, and enumerate the $\abs{A}$-subsets of $N$ as $a_1 , \cdots
, a_d$. These are all the possible places where we could find a transversal copy of $A$.

Inductively we build the pictures $P_1 , \cdots , P_d$.
Assuming we have built $P_i$, we construct $P_{i+1}$. Let $A_{i+1}$ be the
$\abs{A}$-partite sub-graph of $P_i$, composed of the parts of $P_i$ corresponding to $a_{i+1}$.
Now we apply the partite lemma to this to get some $C_{i+1}$ such that
\begin{equation}
C_{i+1} \fromto \left( A_{i+1} \right)_r^A\ . 
\end{equation}
Now we want to extend each copy of $A_{i+1}$ in $C_{i+1}$ to a copy of $P_i$ and
amalgamate those freely. This is $P_{i+1}$.

The conclusion will be that $P_d$ has the required Ramsey property.
This has a natural partial order, but we might worry this doesn't have a linear order. 
As it turns out we can just extend this arbitrarily in the last step. 

To see that $P_d$ is indeed what we want, we have the following claim:

\begin{clm}
If the copies of $A$ in $P_i$ (with set of parts $a_i$) are $r$ colored, then there is a copy
of $P_{i-1}$ where all the copies of $A$ have the same color.
\end{clm}

The idea is as follows. First consider $P_d$, and color the transversal copies of $A$
inside it.
We only really care about the transversal copies since the $B$ in the end will be
transversal. 
So $a_d$ tells us to look at certain parts of $P_d$. By the Partite lemma, if
we look at only these parts we get a copy of $A_d$ in which all of the transversal
copies of $A$ have the same color. But $A_d$ inside $P_d$ corresponds to a copy of $P_{d-1}$,
so really we have a copy of $P_{d-1}$ where all of the transversal copies of $A$ have the
same color. And now we make our way down to $P_0$.
In particular we get a copy of $P_0$ which has the following property. Any two transversal
copies of $A$ on the same parts must have the same color.
I.e. we have a coloring $\binom{N}{\abs{A}}\fromto r$. By the choice of $N$, we get
$b\subeq N$ a homogeneous subset of size $\abs{B}$. 
Therefore we have found a homogeneous copy of $B$ in $P_0$.
\end{proof}

\end{document}

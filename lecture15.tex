\documentclass{amsart}
\usepackage{master}
\begin{document}
\title{Lecture 15\\Math 229}
\author{Lecture: Professor Pierre Simon\\Notes: Jackson Van Dyke}
\maketitle

\section{Subgroups of $\Aut\left( M \right)$}

We are moving towards viewing automorphism groups as Polish groups as in descriptive
set theory. But for now we will be consider subgroups of $\Aut\left( M \right)$.

Let $M$ be an $\om$-categorical structure.\footnote{Not necessarily homogeneous.}
Write $G = \Aut\left( M \right)$. 
Note that if $\oline{a}\in M$, then $G_{\oline{a}}$, the pointwise stabilizer of
$\oline{a}$, is a clopen subgroup of $G$ since it is defined by a finite condition. The
index of this subgroup is $\leq \aleph_0$ since cosets are in bijection with orbits over
$\oline{a}$. In particular it is finite iff $\oline{a}\in \algcl\left( \emp \right)$.

If $X\subeq M$, $G_{ \left\{ X \right\}}$ is a closed subgroup of $X$.
If $X$ is definable, then $G_{ \left\{ X \right\}}$ is clopen. To see this we can look at
the parameters defining $X = \phi\left( M , \oline{a} \right)$.
In particular we note that $\sigma\in
G_{ \left\{ X \right\}}$ iff $\phi\left( x , \oline{a} \right)\iff \phi\left( x ,
\sigma\left( \oline{a} \right)\right)$.

The way we should think about this is by looking at imaginaries.
Consider the equivalence relation 
\begin{equation}
E_\phi\left( \oline{x} ,\oline{y} \right) = \left( \forall x \right)
\phi\left( \oline{x} , \oline{y} \right) \iff \phi\left(\oline{y},\oline{x} \right)
\end{equation}
and let $e = \oline{a} / E_\phi \in M^{eq}$.
Then $G_{\left\{ X \right\}} = G_e$, where we identify $\Aut\left( M \right)\simeqq
\Aut\left( M^{eq} \right)$. 
$M^{eq}$ is the structure one obtains after adding all imaginary sorts,
i.e. it is the structure obtained by adding a new sort for every
$\emp$-definable quotient of some $\emp$-definable set.
Therefore an automorphism of $M$ induces a unique automorphism of $M^{eq}$. 
The topology is also somehow the same.

We also have a converse to this:

\begin{prop}
If $H\leq G$ is an open subgroup, then there is some $e\in M^{eq}$ such that
$H = G_e$.
\end{prop}

\begin{proof}
Since $H$ is open we can take a neighborhood of the identity in it:
$1\in U\subeq H$ where $U$ is a basic neighborhood of the form:
\begin{equation}
\left\{ \sigma \in G \st \sigma\left( \oline{a} \right) = \oline{a} \right\} \ .
\end{equation}
Now consider $D = H\oline{a}\subeq M^k$. $D$ is fixed setwise by $G_{\oline{a}}\subeq H$
so it is $\oline{a}$-definable. 
Let $e$ be the canonical parameter of $D$. Then we claim that $H$ is the setwise
stabilizer of $D$, i.e. $H = G_e = G_{ \left\{ D \right\}}$. 
For $\sigma \in H$ we have 
\begin{equation}
\sigma\left( D \right) = gH\oline{a} = H\oline{a} = D \ .
\end{equation}
If $\sigma\in G_e$ we have
$\sigma\left( \oline{a} \right)\in D$
so there is $\tau\in H$ such that 
$\sigma\left( \oline{a} \right) = \tau \left( \oline{a} \right)$.
Then $\tau^{-1} \sigma \in G_{\oline{a}} \subeq H$.
\end{proof}

\begin{rmk}
There are at most $\aleph_0$ open subgroups of $\Aut\left( M \right)$ since they are each
stabilizers of some set.
\end{rmk}

\begin{defn}
$\Aut\left( M \right)$ has the \emph{small index property} (sip) if any small
subgroup\footnote{I.e.  a subgroup of index less than $2^{\aleph_0}$.} is open.
\end{defn}

\begin{rmk}
The point of the sip is that the topology can be recovered from the group structure.
In particular, if $\Aut\left( M \right)$ and $\Aut\left( N \right)$ both have sip, and 
$\Aut\left( M \right)\simeq \Aut\left( N \right)$ in $\Grp$\footnote{Note this is not
$\TopGrp$.} then $M$ and $N$ are bi-interpretable. This is shown by looking at the basis
of neighborhoods of the identity.
\end{rmk}

This property is common for finite homogeneous structures and some $\om$-category
structures. 
In fact not many counterexample have been found at all. The following is an example of
one.

\begin{exm}[Counter-example]
Let 
\begin{equation}
L= \left\{ R_n\left( \oline{x} , \oline{x}'\right) \st 0 < n < \om \right\}
\end{equation}
with $2$ classes and $R_n$ some $2n$-ary equivalence relation on
\begin{equation}
\left\{ \oline{x}\in M^n \st\forall i\neq j, x_i\neq x_j \right\} \ .
\end{equation}
Let $M$ be the \Fraisse limit of this class. Note this is $\om$-categorical since there
are only finitely many quantifier free types, and $M$ is a \Fraisse limit. 

Now we have a map $\Aut\left( M \right)\surj \left( \ZZ / 2 \ZZ \right)^\om$ such that for
every $M$, the two classes are either preserved or switched. 
We might give a more general argument that this is surjective. 
This group already has too many subgroups of index $2$. So if $\cU$ is an ultrafilter on
$\om$, we have a subgroup $H_{\cU}\leq \left( \ZZ / 2\ZZ \right)^\om$ defined by
\begin{equation}
\left\{ f\in \left( \ZZ / 2\ZZ \right)^\om \st \left\{ t\in \om , f\left( t \right) = 0
\right\}\in \cU \right\} \ .
\end{equation}
Then $H_\cU$ has index $2$.
By taking the preimage along the surjective map, we get $2^{2^{\aleph_0}}$ many subgroups
of $\Aut\left( M\right)$ having index $2$, so they cannot all be open.
\label{exm:isp_counter}
\end{exm}

\section{Galois group of $M$}

The idea of the above example is that $\left( \ZZ / 2\ZZ \right)^\om$ is compact, 
and the obstruction to having the sip is having a large compact quotient.

\begin{defn}
A \emph{strong automorphism} of $M$ is an element of $\Aut\left( M \right)$ which fixes
every $\algcl^{eq}\left( \emp \right)$.
I.e. it fixes setwise each class
of each $\emp$-definable finite equivalence relation.
\end{defn}

Define $G^0\leq G$ to consist of the strong automorphisms. 
Then $G^0\nsubeq G$, and the quotient is:
\begin{equation}
G / G^0 \simeq \Aut\left( \algcl^{eq}\left( \emp \right) \right)
\end{equation}
which is a compact pro-finite group. 
$\algcl^{eq}\left( \emp \right)$ is the union of finitely many zero definable sets, so the
inverse image consists of actions with finitely many orbits.

\begin{lem}
The map $G\fromto \Aut\left( \algcl^{eq}\left( \emp \right) \right)$ is surjective.
\end{lem}

This map is just restriction to the empty set. 
There is indeed something to prove here because
$\algcl^{eq}\left( \emp \right)$ is countable, and if you take a
countable subset of $M$, restriction onto it is not surjective in general.

\begin{proof}
Fix $\sigma \in \Aut\left( \algcl^{eq}\left( \emp \right) \right)$. We build a preimage
$g$ by the back and forth argument since we only care about a finite part of $\sigma$ each
time. Inductively we ensure that if $g:\oline{a}\fromto \oline{b}$ then for
any image of $\oline{a}$ in $\algcl^{eq}$ we have
\begin{equation}
\restr{g}{\dcl\left( \oline{a} \right) \cap \algcl^{eq}\left( \emp \right)} = 
\restr{\sigma}{\dcl\left( \oline{a} \right) \cap \algcl^{eq}\left( \emp \right)}
\ .
\end{equation}
So if we add $\oline{a}\comp a_0$, we want $b_0$ such that
\begin{equation}
\type\left( \oline{a} , \algcl^{eq}\left( \emp \right) \right) = 
\type\left( \oline{b} , b_0 \sigma\left( \algcl^{eq}\left( \emp \right) \right) \right) 
\ .
\end{equation}

We need to preserve a type over an infinite set. But for a fixed $n$ there are only
finitely many $n$-types over $\algcl^{eq}\left( \emp \right)$ since the relation
$\oline{x} \equiv_{\algcl^{eq}} \oline{x}'$ (where $n = \abs{\oline{x}}$)
has cardinality at most the continuum and furthermore is zero definable in an
$\om$-categorical structure. Therefore a definable equivalence relation can have only
finitely many classes. So every class appears in $M$ and we can find such a $b_0$. 

Now since the map is a homeomorphism we are done.
\end{proof}

\begin{prop}
$\Aut\left( \algcl^{eq}\left( \emp \right) \right)$ is the largest compact quotient of
$G$. 
\end{prop}

\begin{exr}
Show this.
\end{exr}

\begin{prop}
Given any metrizable profinite group $\Gamma$, there is an $\om$-categorical $M$ with
$\Aut\left( \algcl^{eq}\left( \emp \right) \right)\simeq\Gamma$.
\end{prop}

\begin{prop}
If $\Gamma$ is profinite and has infinitely many open subgroups of index $n$, then it has
$2^{2^{\aleph_0}}$ many subgroups of finite index.
\end{prop}

The point is that there is nothing we can say about this compact quotient, so for the sip
we have to ignore this.
Note that one can ask what conditions we need to add for the converse to hold. There is a
theorem which says:

\begin{thm}[Nikolov-Segal]
If $\Gamma$ is profinite and topologically finitely generated, then finite index subgroup
are open.
\end{thm}

This is surprising because it uses the classification theorem for finite simple groups.
The other known example hides the problem with the quotient inside the classes.

\begin{exm}
Consider the counterexample in \cref{exm:isp_counter}.
Now add an equivalence relation with infinitely many classes, and put a copy of the
previous thing in each class. 
This has no $\algcl^{eq}\left( \emp \right)$ and therefore no compact quotient.
\end{exm}

\section{Polish spaces}

\begin{defn}
A \emph{Polish space} is a topological space which is separable and completely metrizable. 
\end{defn}

\begin{exm}
The following are all example:
$\RR$, $\CC$, $\left( 0,1 \right)$, $\II=\left[ 0,1 \right]$, countable discrete sets, $C =
2^\NN$ (Cantor space), $\HH = \II^\NN$ (Hilbert cube), and $W = \NN^\NN$ (Baire space).
\end{exm}

\begin{exr}
Show that Baire space is homeomorphic to $\RR \minus \QQ$.
\end{exr}

In general the countable product of Polish spaces is Polish because of the
following. 
Given $X$ Polish, there is a compatible complete metric bounded by $1$ by truncation. To
put a metric $d$ on the product
\begin{equation}
\prod_{i<\om} X_i
\end{equation}
we take the metric $d_i$ on each $X_i$, and then just define
\begin{equation}
d\left( \oline{x} , \oline{y} \right) = \sum_i \frac{d_i\left( x_i , y_i \right)}{2^n} \ .
\end{equation}

\begin{lem}
A compact Hausdorff space $X$ is Polish iff it is metrizable iff it is
second-countable.
\end{lem}

Note there are some nice universality properties that won't be useful for us, but are
worth mentioning. 

\begin{prop}
Every Polish space is homeomorphic to a subset of the Hilbert cube.
\end{prop}

\begin{proof}
Let $X$ be Polish, and let $\left( x_n \right)$ be a countable dense subset. 
Let $d$ be a compatible metric bounded by $1$. Define a map $f:X\fromto \HH$ where
$x\mapsto \left( d\left( x , x_n \right) \st n<\om \right)$. 
$f$ is injective because $x$ is a limit of a sequence of $x_n$s, so if $x$ and $y$ have
the same distance then they are the limits of the same sequence. 
$f$ is also continuous. This comes down to an argument on one coordinate. 
To check that $f^{-1}:f\left( X \right) \fromto X$ is continuous, assume that $f\left(
x^m \right)$ converges to $f\left( x \right)$.
This means $d\left( x^m , x_n \right)$ converges to $f\left( x , x_n \right)$ for each
$n$.
Fix $\e > 0$ and let $n$ be such that $d\left( x , x_n \right) < \e$. 
then $d\left( x^m , x_n \right) < 2\e$ for $n$ large enough such that $d\left( x^m , x
\right) < 3\e$.
\end{proof}

\begin{exr}
Show that you can embed any Polish space $X$ as a closed subspace of $\RR^\NN$. 
\end{exr}

In general, the result might not be closed, but it can at least be $G_\dd$. 

\begin{defn}
A set is $G_\dd$ if it is the countable intersection of open sets, and $F_\sigma$
if it is the countable union of closed sets.
\end{defn}

The Borel sets comprise the smallest $\sigma$ algebra\footnote{Recall a $\sigma$-algebra
is defined to contain the whole space, and be closed under complements and countable
unions. As such it is also closed
under countable intersections.} containing open sets. 

\begin{prop}
Let $X$ be Polish with $Y\subeq X$. 
$Y$ is Polish iff it is $G_\dd$.
\end{prop}

\begin{proof}
$\left( \converse \right)$:
Closed subsets of Polish spaces are Polish. If we take an open subset, then we can change
the metric such that it blows up at the boundary. If $U\subeq X$ is open, define a metric
$d_U$ on $U$ by setting
\begin{equation}
d_U\left( x,y \right) = d\left( x,y \right) + \abs{\frac{1}{d\left( x , X\minus U \right)}
-\frac{1}{d\left( y , X\minus U \right)}} \ .
\end{equation}
This way, if we take a sequence converging to the boundary, it can't be Cauchy and not
convergent. 
This gives us completeness. 
Then $d_U$ is a complete compatible metric on $U$. 

Separability follows from being a subspace. If it is $G_\dd$,
\begin{equation}
Y = \binter_{i < \om} U_i
\end{equation}
for some open $U_i$s. Now we can truncate distances at $1$ and take the sum
\begin{equation}
d_Y = \sum_i \frac{\min\left( 1 , d_{U_i}\left( x,y \right) \right)}{2^i}
\end{equation}
so we are done.

We will not show the other direction.
\end{proof}

\end{document}

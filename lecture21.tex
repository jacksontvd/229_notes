\documentclass{amsart}
\usepackage{master}
\begin{document}
\title{Lecture 21\\ Math 229}
\author{Lecture: Professor Pierre Simon\\ Notes: Jackson Van Dyke}
\maketitle

Recall we had all these lemmas last time. Now we have the following corollary:
\begin{cor}
If $G\acts X$ is topologically transitive, then the set of $G$-big
elements is either empty or is one orbit.
\end{cor}

\begin{proof}
This follows almost immediately. We have seen that as long as the action is topologically
transitive (some dense orbit) then $x$ being $G$-big implies $G\cdot x$ is dense. 
Then this means any $y$ which is $G$-big is in $\interior{\tclosure{G\cdot x}}$ so is in
$G\cdot x$ by the lemma.
\end{proof}

\begin{cor}
If $G\acts X$ is topologically transitive and $x\in X$ then TFAE:
\begin{enumerate}[label = (\iii)]
\item $x$ is $G$-big,
\item $G\cdot x$ is comeager, 
\item $G\cdot x$ is not meager.
\end{enumerate}
\end{cor}

\begin{proof}
$\left(i\right)\implies \left(ii\right)$: This follows directly from previous results.

$\left(ii\right)\implies\left(iii\right)$: This is clear. 

$\left(iii\right)\implies\left(i\right)$: This is just a calculation.
Assume $G\cdot x$ is not meager and let $1\in U\subeq G$. So now we want to show that
$U\cdot x$ is somewhere dense. Write
\begin{equation}
G = \bun_{n < \om} g_n U
\end{equation}
so
\begin{equation}
G\cdot x = \bun_{n < \om} g_n U\cdot x
\end{equation}
but this means for some $n$
$g_n  U\cdot x$ is somewhere dense, and hence $U \cdot x$ is somewhere dense and we are
done.
\end{proof}

\begin{thm}
Let $\cK$ be a \Fraisse class with limit $M$, $G = \Aut\left(M\right)$, $n < \om$.
Then TFAE:
\begin{enumerate}[label = (\iii)]
\item $\cK\left(n\right)$ has EAP and JEP.
\item There is a generic tuple in $G^n$.
\end{enumerate}
\end{thm}
\begin{proof}
$\left(i\right)\implies \left(ii\right)$: Assuming $\left(i\right)$ the JEP implies that
the action $G\acts G^n$ by conjugation is topologically transitive by a previous theorem.
We need to show that for all $\e > 0$ the union of all $\e$-small open sets is dense. 

So let $\e > 0$, $U\subeq G^n$ open. We can assume $U$ is of the form 
\begin{equation}
U = U_{\oline{f}} =
\left\{ \oline{\sigma}\in G^n \st \sigma_i \supeq f_i\right\} \ .
\end{equation}
Let $A_0\in \cK$ contain the domain and range of the $f_i$s so that
$\left(A_0 , \oline{f}\right)\in \cK\left(n\right)$. 
Now increase $A_0$ to $M\supeq A\supeq A_0$ such that for all $\tau\in
\Stab\left(A\right)$ we have
\begin{equation}
\dist\left(\tau , 1\right) < \e \ .
\end{equation}
Then we still have $\left(A , \oline{f}\right) \in \cK\left(n\right)$. 

Now by EAP there is 
\begin{equation}
\left(A , \oline{f}\right) \subeq \left(B , \oline{g}\right)
\end{equation}
definite on $\left(A , \oline{f}\right)$. 
So $U_{\oline{g}} \subeq U_{\oline{f}} = U$. Then the claim is the following:
\begin{clm}
$U_{\oline{g}}$ is
$\e$-small.
\end{clm}
But we have actually already shown this. We gave a characterization of what it means to
have this EAP property. For $V_1, V_2 \subeq U_g$, by the previous lemma we have that 
\begin{equation}
\Stab\left(A\right)\cdot V_1 \cap V_2 \neq \emp \ ,
\end{equation}
and hence $\left(V_1\right)_{< \e}\cap V_2 \neq \emp$ which is exactly what we want.

$\left(ii\right)\implies \left(i\right)$: We have JEP since there is a dense orbit. 
We now show EAP. Consider
$\left(A , \oline{f}\right)$. Then choose $\e > 0$ such that
\begin{equation}
\Stab\left(A\right)\supeq B_\e\left(1\right) \ .
\end{equation}
Let $U_{\oline{g}} \subeq U_{\oline{f}}$ be $\e$-small corresponding to some $\left(B ,
\oline{g}\right)$, $A\subeq B$. This is definite on $A$.
\end{proof}

It remains to prove/give a criterion for EAP. 

\begin{prop}
The class of graphs has EAP.
\end{prop}

\begin{proof}
Let $\left(A , \oline{f}\right) \in \cK\left(n\right)$. Take $B$ an EPPA witness for $A$,
and extend each $f_i$ to $g_i \in \Aut\left(B\right)$. Then $\left(B , \oline{g}\right)$
is definite over $\left(A , \oline{f}\right)$. If
\begin{equation}
\begin{tikzcd}
& \left(C_1 , \oline{h_1}\right)\\
\left(B , \oline{g}\right)\arrow{ur}\arrow{dr} & \\
& \left(C_2 , \oline{h_2}\right)\\
\end{tikzcd}
\end{equation}
let 
\begin{equation}
D = C_1 \dun C_2
\end{equation}
with no edges between $C_1 \minus B$ and $C_2 \minus B$, $k_i = \left(h_1\right)_i \un
\left(h_2\right)_i$. Then
\begin{equation}
\begin{tikzcd}
& \left(C_1 , \oline{h_1}\right)\arrow{dr}&\\
\left(B , \oline{g}\right)\arrow{ur}\arrow{dr} && \left(D , \oline{k}\right)\\
& \left(C_2 , \oline{h_2}\right)\arrow{ur}&\\
\end{tikzcd} \ .
\end{equation}
\end{proof}

This argument shows ample generics, and therefore sip, for (almost) all classes for which
we have proved EPPA. It fails for, say, DLO.
However, it is true that DLO has sip, but by a completely different argument. 

\begin{rmk}
Sip is open for:
\begin{itemize}
\item random ordered graph,
\item generic tournament (EPPA is not known)
\end{itemize}
\end{rmk}

\begin{Proof}[Original proof of sip for $S_\infty$; Dixon, Neumann, Thomas (1985)]
Let $G\leq S_\infty = \Sym \Om$ (where $\Om$ is a countable set) of index $< 2^{\aleph_0}$. 
We want to show there is a finite $A_0 \subeq \Om$ such that
\begin{equation}
G_{\left(A_0\right)} \leq G \leq G_{ \left\{A_0\right\}} \ .
\end{equation}
\begin{lem}
Let $\Gamma_1 , \Gamma_2 \subeq \Om$ be infinite such that $\Gamma_1 \cap \Gamma_2$ is
infinite and write
\begin{equation}
\Sym\left(\Gamma_1\right) = G_{\left(\Om \minus \Gamma_1\right)} \ .
\end{equation}
Then 
\begin{equation}
\lr{\Sym\left(\Gamma_1\right) , \Sym\left(\Gamma_2\right)} = \Sym\left(\Gamma_1 \un
\Gamma_2\right)
\end{equation}
\end{lem}
\begin{proof}
Exercise.
\end{proof}
\begin{clm}
There is a moiety\footnote{This means $\Sigma$ and $\Om\minus \Sigma$ are infinite.}
$\Sigma\subeq \Om$ such that $\Sym\left(\Sigma\right)\leq G$.
\end{clm}
\begin{proof}
Let $\left(\Sigma_i , i < \om\right)$ be disjoint moieties partitioning $\Om$.
Let $S_i = \Sym\left(\Sigma_i\right)\leq S_\infty$. 
Let $H\leq G$ be
\begin{equation}
H = \left\{g\in G \st \forall i, g\Sigma_i = \Sigma_i\right\} \ .
\end{equation}
Then $H\leq \prod S_i$ has index less than $2^{\aleph_0}$ (since $G$ has
index less than this and $H = G\cap \left(\prod S_i\right)$). Then it follows that for
some $i$, the image of the projection $H\fromto S_i$ is equal to $S_i$.

Now consider $K = G\cap \Sym \Sigma_i$. 
\begin{clm}
$K\nsubeq \Sym\left(\Sigma_i\right)$.
\end{clm}
Let $\sigma \in K, \tau\in \Sym\Sigma_i$. Then we want to show $\tau\sigma\tau^{-1}\in K$.
Now let $h\in H$ which projects to $\tau\in S_i$. Then 
\begin{equation}
h\sigma h^{-1} = \tau \sigma \tau^{-1} \in K \ .
\end{equation}

\begin{fact}
The normal subgroups of $S_\infty$ are $0$, $S_\infty$, finitary permutations, and
finitary alternating permutations. 
\end{fact}
But there is only one group here with index less than $2^{\aleph_0}$ is $S_\infty$, 
so $K = S_\infty$.
\end{proof}

Now note that there
is an almost disjoint family of $\left(A_j , j < 2^{\aleph_0}\right)$ of moieties of
$\Sigma_i$.
Then for each $j < 2^{\aleph_0}$, let $g_j\in S_\infty$ be an involution which exchanges
$A_j$ and $\Om\minus \Sigma_i$, and fixes $\Om\minus A_j$. 
Now we can find $j,k < 2^{\aleph_0}$ such that
\begin{equation}
g_j^{-1} g_k \in G \ .
\end{equation}
Now we have $A_j$ and $A_j$ which might have finite intersection. Then this group element
takes an element of $A_k$, throws is out to $\Omega\minus \Sigma_i$, brings it
back into $A_j$, and throws it back out. Specifically, we claim that 
\begin{equation}
g_j^{-1} g_k\left(\Sigma_i\right) \supeq \left(A_j \minus A_k\right) \un g_j 
\left(A_j\minus A_k\right) \ .
\end{equation}
The first set is infinite, and the second is the complement minus finitely many points.
Now let $B_0$ be finite such that
\begin{equation}
\Om\minus \Sigma_i = g_j\left(A_j \minus A_k\right) \un B_0 \ .
\end{equation}

Now apply the lemma to $\Sigma_i$ and $g_i^{-1} g_k\left(\Sigma_i\right)$ to get 
\begin{equation}
\Stab\left(B_0\right) \leq G 
\end{equation}
and we are basically done.
\end{Proof}

This is the proof that would be generalized to DLO, but the first lemma doesn't quite
apply, so we would need to somehow leverage the product argument much more.

\end{document}

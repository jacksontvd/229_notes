\documentclass{amsart}
\usepackage{master}
\begin{document}
\title{Lecture 27\\ Math 229}
\author{Lecture: Professor Pierre Simon\\ Notes: Jackson Van Dyke}
\date{May 2, 2019}
\maketitle

\section{Stable case}

\subsection{Setup and statement of the theorem}

Today we will (hopefully) finish the stable case. 

Let $M$ be an $\om$-stable ($\om$-categorical) structure. 
We didn't define $\om$-stable, so we will restrict to the case that $M$ is finitely
homogeneous and stable, but everything we say will be true in this generality. 
In particular, the features of $\om$-stability that we will use are the following:
\begin{enumerate}[label = (\iii)]
\item $M$ is ranked (in fact we will assume finite rank\footnote{There is an extra step to
get infinite rank which would happen at the end, but we won't have time to treat this.}).
\item If we have a definable subset $X\subeq M^k$ which has rank $n$, there is $X_0\subeq
X$ of rank $n$ and
\emph{indivisible} in the sense that it cannot be split into two disjoint sets of rank
$n$. 
\item If $X$ is defined over $a$ $X_0$ cannot be defined over $\algcl^{eq}\left(a\right)$.
\item If $\phi\left(\oline{x},\oline{y}\right)$ is a formula there is a
\emph{normalization} $\phi^*\left(\oline{x},\oline{y}\right)$ such that
\begin{itemize}
\item $\rank\left(\phi\left(\oline{x},\oline{y}\right)\Del
\phi^*\left(\oline{a},\oline{b}\right)\right) < n$
\item $\forall \oline{b}, \oline{b'}\satisfies q$
\begin{align}
\rank\left(\phi^*\left(\oline{x},\oline{b}\Del\right) \phi^*\left(\oline{x} ,
\oline{b'}\right)\right)<n
&&\implies &&
\phi^*\left(\oline{x},\oline{b}\right) = \phi^*\left(\oline{x} , \oline{b'}\right)
\end{align}
\end{itemize}
\end{enumerate}

Let $P$ be a $\emp$-definable set $M^{eq}$. Say that $P$ coordinatizes $M$ if for all
$a\in M$
\begin{equation}
\acl\left(a\right)\cap P\neq \emp \ .
\end{equation}

\begin{thm}
Let $M$ be $\om$-stable, $\om$-categorical, (of finite rank). There exists is a rank $1$ 
set (finite rank, strongly minimal) which coordinatizes $M$.
\end{thm}

\begin{cor}
If $M$ is primitive then $M$ is a Grassmannian over a (finite union of) strongly minimal
set(s).
\end{cor}

\subsection{Proof of the theorem}

\subsubsection{Aside on strongly minimal}

Let $X$, $Y$ be $\emp$-definable, strictly minimal (i.e. $\acl\left(a\right) =
\left\{a\right\}$)
then either
\begin{itemize}
\item there is a unique $\emp$-definable bijection between $X$ and $Y$, or
\item $X$, $Y$ are orthogonal: $\oline{a} \in X^k$, $\oline{b}\in Y^l$, $\oline{a}\indfrom
\oline{b}$.
\end{itemize}

\begin{proof}
For $X$, $Y$ indiscernible sets take $\oline{a}\in X^k$, $\oline{b}\in Y^l$ such that
$\oline{a}\not\indfrom \oline{b}$ where $k+l$ is minimal.
There is $x_0\in Y\cap \acl\left(\oline{a}\right)$ (look at
$\type\left(\oline{b}/\oline{a}\right)$ this is not $Y^l$ so some element of $Y^l$ is in
$\acl$ of $\oline{a}$). Then $\oline{a}\not\indfrom b_0$ so $l = 1$, and $k = 1$ by
symmetry. 
\end{proof}

Note that if $\oline{b}\in Y^k$ ($x_i\neq b_j$ for $i\neq j$) then
$\rank\left(\oline{b}\right) = k$. 
If $a\in X$, $\rank\left(a\right) = 1$ so 
$\rank\left(\acl\left(a\right)\right) = 1$
so $\abs{\acl\left(a\right)\cap Y}\leq 1$.
This means $\abs{\acl\left(a\right)\cap Y} = 1$ which gives us a definable bijection.

Now the following is a consequence of uniqueness.
Let $X_1 \ldots X_n$ be strictly minimal 
and assume $\left(X_i , X_j\right)$ are orthogonal for $i\neq j$.
Then for $\oline{a_i}\in X_i^{k_i}$
\begin{equation}
\rank\left(\oline{a_1}^n \land \ldots \land\oline{a_n}\right)
= \sum \rank\left(\oline{a_i}\right) \ .
\end{equation}

\subsubsection{Back to the proof}

Write $\rank\left(M\right) = n$ and assume $M$ is indivisible. There is $\phi\left(x ,
\oline{b}\right)$ of rank $n-1$. Write $q  = \type\left(\oline{b}\right)$
We can assume
\begin{itemize}
\item $\phi\left(\oline{x} , \oline{y}\right)$ is normalized
\item $\phi\left(x , \oline{b}\right)$ is indivisible for $\oline{b}\satisfies q$.
\item $\oline{b}\neq \oline{b'}$ implies $\phi\left(X , \oline{b}\right)\neq \phi\left(x ,
\oline{b'}\right)$ (taking $\oline{b}\in M^q$).
\end{itemize}

Then the goal is to show $\rank\left(\oline{b}\right) = 1$. 

Let $F = \type\left(\oline{b}\right)$ definable set. Assume $\rank\left(f\right)\geq 2$. 
Let $I\left(\oline{d}\right)\subeq F$ be a strongly minimal definable subset.
Let $H\left(\oline{d}\right)$ be the associated strictly minimal definable subset.
We can assume $\oline{d}\in \acl\left(\emp\right)$, $q= \type\left(\oline{d}\right)$.

\begin{clm}[Main claim]
If $\oline{d_1}\indfrom \oline{d_2}$, then $I\left(\oline{d_1}\right)\Del
I\left(\oline{d_2}\right)$ is finite. 
\end{clm}

\begin{proof}
Assume not, then $I\left(\oline{d_1}\right)\cap I\left(\oline{d_2}\right)$ is finite. Take
$\oline{d_1}, \ldots , \oline{d_N}$ indiscernible independent $N$ large. 

Let $e = d_1\land \ldots \land d_N$. Let
\begin{equation}
Q = \left\{x\in M \st \rank\left(x/e\right) = n\right\} \ .
\end{equation}
Then $Q$ is transitive over $e$ since $M$ is indivisible. 

For a given $i$, 
\begin{equation}
\bun_{\oline{b}\in I\left(\oline{d_i}\right)} \phi\left(x , \oline{b}\right)
\end{equation}
has rank $n$. so for all $x_0\in Q$, for all $i$ there is $\oline{b}\in I \oline{d_i}$
such that $\phi\left(x_0 , \oline{b}\right)$ holds.
\end{proof}

We work over $e$.

\begin{clm}
If $a\in Q$ and $\phi\left(a, \oline{b}\right)$ holds with $\oline{b}\in
I\left(\oline{d_i}\right)$ for some $i$. Then $\oline{b}\in \algcl\left(a\right)$.
\end{clm}

\begin{proof}
Otherwise $a$ is in almost all $\phi\left(x , \oline{b'}\right)$ for $\oline{b'}\in
I\left(d_i\right)$. But then for any two, we can intersect them and
\begin{equation}
\rank\left(\phi\left(x , \oline{b}\right) \cap \phi\left(x , \oline{b'}\right)\right) = n
- 1
\end{equation}
which means
\begin{equation}
\rank\left(\phi\left(x , \oline{b}\right) \Del \phi\left(x , \oline{b'}\right)\right) <
n-1
\end{equation}
which contradicts normalization.
Now we have two cases:
\begin{enumerate}[label = case \numbers.]
\item $H\left(\oline{d_1}\right) \ldots H\left(\oline{d_N}\right)$ are orthogonal.
\label{case:1}

\item There is a unique $\emp$-definable set between $H\left(\oline{d_i}\right)$ and
$H\left(\oline{d_j}\right)$. 
\label{case:2}
\end{enumerate}
\end{proof}

Assume \ref{case:1}
Then for any $a\in Q$, $\acl\left(a\right)\cap H\left(\oline{d_i}\right)\neq 0$ for all
$i$ so $\rank\left(a\right) \geq N > n$ which is a contradiction.

Assume \ref{case:2}. 
\begin{clm}
$\rank\left(\acl\left(a\right)\cap \left(H\left(\oline{d_1}\right)\un \ldots \un
H\left(\oline{d_N}\right)\right)\right)\geq N$ .
\end{clm}

\begin{proof}
For $\oline{b_1} \in I\left(\oline{d_1}\right)$ and $\oline{b_2}\in I\left(\oline{d_2}\right)$ we have
\begin{equation}
\rank\left(\phi\left(x , \oline{b_1}\right)\cap \phi\left(x , \oline{b_2}\right)\right)<
n-1
\end{equation}
so if $a\in \phi\left(x , \oline{b_1}\right)\cap \phi\left(x , \oline{b_2}\right)$ then 
\begin{align}
\rank\left(a / \oline{b_1}\right) = n-1
&&
\rank\left(a / \oline{b_1}\oline{b_2}\right) \leq n-2
\end{align}
so
\begin{equation}
\rank\left(\oline{b_2} / \oline{b_1}\right)\geq 1
\end{equation}
so
\begin{equation}
\oline{b_2}\not\in \acl\left(\oline{b_1}\right) \ .
\end{equation}
So we got the same contradiction.
\end{proof}

So if we take for each $i$, $\oline{b_i}\in I\left(d_i\right)$ for each $i$ then
$\phi\left(a_, \oline{b_i}\right)$ holds
\begin{equation}
\rank\left(\oline{b_1}^n \land \ldots \land \oline{B_N}
\right) \geq N
\end{equation}
so $\rank\left(a\right)\geq N$ which is a contradiction.

This being done, normalize the family $I\left(\oline{b}\right)$ into
$I^*\left(\oline{b}\right)$. Then 
\begin{equation}
I^*\left(\oline{b_1}\right) = I^*\left(\oline{b_2}\right)
\end{equation}
for $\oline{b_1}\indfrom \oline{b_2}$ so
\begin{equation}
I^*\left(\oline{b_1}\right) = I^*\left(\oline{b_2}\right)
\end{equation}
for all $\oline{b_1},\oline{b_2}\satisfies q$
so $\rank F = 1$.

By the proof, $F$ coordinatizes $M$.

\end{document}

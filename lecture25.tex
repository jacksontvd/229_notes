\documentclass{amsart}
\usepackage{master}
\begin{document}
\title{Lecture 25\\ Math 229}
\author{Lecture: Professor Pierre Simon\\ Notes: Jackson Van Dyke}
\date{April 25, 2019}
\maketitle

\section{More lemmas}

Today we will finish the proof we have been doing. 
The last thing we proved was that if we have an intersection of $D_a$ and $D_{a'}$ then
there is a neighborhood of the point in common such that the orders are either the same or
opposite. 
Now we have the following.

\begin{lem}
Let $D_1$, $D_2$ be almost linear, transitive (and definable over some parameters
$\oline{a}$ and $\oline{b}$ respectively). Assume they have an interval $I$ in common.
Take the maximal such $I$. Then $I$ is cofinal either in $D_1$ or $D_2$.
\end{lem}

\begin{proof}
Assume $I$ is not cofinal in either. Let $c\in D_1$ be the $\sup$ of $I$ in $D_1$, and
$d\in D_2$ be the $\sup$ of $I$ in $D_2$. Then $c\in
\dcl\left(\oline{a}\land\oline{b}\right)$.
Therefore $c$ is interalgebraic with some element of $\oline{b}$.
Let $d = d_0 > d_1 > d_2 > \ldots$ be a sequence of elements of $D_2$. By transitivity,
for each $i$ there is $\sigma_i\in \Aut\left(M / \oline{b}\right)$ such that
$\sigma_i\left(d\right) = d_i$. Set $D_{1,i} = \sigma_i\left(D_1\right)$, and $c_i =
\sigma_i\left(c\right)$.
\begin{clm}
The elements $c_i$ are pairwise distinct.
\end{clm}
This must be the case because otherwise $D_{1,i}$ and $D_{1,j}$ would have a neighborhood
of $c_i = c_j$ in common, but this is impossible by construction.

Since each $c_i$ is n $\acl\left(\oline{b}\right)$, this is a contradiction.
\end{proof}

\begin{cor}
Let $D_{0}$, $D_1$ be as above and $I = D_0\cap D_1$. Then we have one of the following:
\begin{enumerate}[label = (\abc)]
\item $I = \emp$,
\label{item:a}
\item $I$ is an initial segment of $D_i$ and a final segment of $D_j$ and the two orders
disagree,
\label{item:b}
\item $I$ is an initial segment of $D_i$ and a final segment of $D_j$ and the two orders
agree, and
\label{item:c}
\item $I = I_1\un I_2$ where each of $I_1$ and $I_2$ are as in $\ref{item:b}$, or
\label{item:d}
\item $I = I_1\un I_2$ where each of $I_1$ and $I_2$ are as in $\ref{item:c}$.
\label{item:e}
\end{enumerate}
\end{cor}

Let $\cL$ be the set of definable subsets of $M$ which are almost linear and transitive
over some $\oline{a}$. If $D_0,D_1\in \cL$, write $D_0\order D_1$ if 
$D_0\cap D_1$ is as in case \ref{item:b} or \ref{item:d}.

Say that $c\in M$ is of \emph{order type} if there is $D\in \cL$ such that $c\in D$.
Let $\Om$ be the set of $c\in M$ of order type. Define an equivalence relation $\cE$ on
$\Om$ by $c\cE d$ if there are $D_0\order D_1 \order \ldots \order D_k$ with $c\in D_0$
and $d\in D_k$ (or $d\in D_0$ and $c\in D_k$).

On each $\cE$-equivalence class we have a definable separation relation
$S\left(a,b,c,d\right)$ defined as follows. Every path as above from $a$ to $b$ contains
either $c$ or $d$.
Note that $D$ is really on the quotient of the class by a finite equivalence
relation.\footnote{This is just interalgebraicity as we saw.}

\section{The main theorem}

\begin{thm}
Assume that $M$ has few substructures. Then there is a $\emp$-definable equivalence
relation $F$ with finite classes and a $\emp$-definable equivalence relation $E$ on the
quotient $M_0= M / F$ such that the following is true.
Let $N\ceqq M_0 / E$, and $M_*$ denote the reduct of $M_0$ to pullbacks of definable
subsets of $N$.
Then $M_*$ is stable, and
\begin{equation}
f_n\left(M_*\right) = f_n\left(M_0\right)
\end{equation}
for all $n$.
\end{thm}

\begin{proof}
$F$ is interalgebraicity for points of order type and equality elsewhere. 
Then for $M_0 = M / F$, $f_n\left(M_0\right) \leq f_n\left(M\right)$, so $M_0$ also has
few substructures.

Let $E$ be the equivalence relation $\cE$ on $M_0$ defined above (extended by $=$ on
points which are not of order type).
Let $N = M_0 / E$. 
\begin{clm}
$N$ is stable.
\end{clm}
Otherwise, as $N$ is NIP, there would be a definable map $f:N\fromto V$ ($V$ linearly
ordered) which would lift to an almost linear subset of $M_0$ 
contradicting the definition of $E$.

Define $M_*$ as in the statement. Then $M_*$ is stable. It just remains to see that
$f_n\left(M_*\right) = f_n\left(M_0\right)$. I.e. a finite substructure $A\subeq M_*$ has
a unique expansion
(up to isomorphism) to a substructure of $M_0$.

Proceed by induction on $\abs{A}$. 
Write
\begin{equation}
A = A_0 \dun A_1 \dun \cdots \dun A_k
\end{equation}
where $A_0$ consists of the points not of order type, and the remaining $A_i$ consist of
points of order type grouped by $E$-equivalence classes.
We expand $A$ to an $M_0$ substructure one $A_i$ at a time.
At stage $i$ the $E$-class of $A_i$ has a structure over $A_0\ldots A_{i-1}$ which is
isomorphic to one of the four unstable reducts of DLO. Therefore there is a unique
embedding of
$A_i$ into $M_0$ up to isomorphisms over the previous $A_{<i}$.
\end{proof}

\begin{thm}
Assume $M$ has few substructures and is primitive. Then 
$M$ is either stable or $M$ is one of the four unstable reducts of DLO.
\end{thm}

In fact we recover the classification of the unstable reducts of DLO.

\begin{proof}
As $M$ is primitive $F$ is equality, and $E$ is either equality or empty and this
gives us the two cases.
\end{proof}

\begin{cor}
Assume that for no polynomial $p$ we have
\begin{equation}
f_n\left(M\right)\geq \frac{\Phi^n}{p\left(n\right)}
\end{equation}
for $\Phi\approx 1.618$ is the golden ratio. Then $M$ has a stable reduct $M^*$ such that
$f_n\left(M\right) = f_n\left(M^*\right)$.
\end{cor}

\begin{proof}
We only need to show that $F$ is equality.
If it wasn't, then we get at least $F_n$ substructures, where $F_n$ is the $n$th Fibonacci
number, and we know $F_n\sim \Phi^n/\sqrt{5}$. 

Note that this is optimal because if we do take a trivial $2$-cover of DLO, then 
$f_n\left(M\right) = F_n$. For any stable reduct $M_*$, $f_n\left(M_*\right) \leq n/2$.
\end{proof}

\section{The stable case}

We will now deal with the stable case and then consider more general NIP.
What remains:
\begin{itemize}
\item Understand the stable case:
\begin{itemize}
\item $\om$ stable: well understood,
\item strictly stable: not well understood at all.
\end{itemize}
\item generalize to $f_n\left(M\right)\sim c^n$ or even $f_n\left(M\right)\sim
e^{2n\log n}$.
\end{itemize}

\begin{qn}
Are there uncountable many $M$ such that $f_n\left(M\right)\sim e^{cn \log n}$?
\end{qn}

The idea to approach this is the following.
Reduce to finite homogeneous case and induct on the (pseudo-)arity.

\begin{defn}
The pseudo-arity of $M$ is the minimal arity of a finitely homogeneous structure $N$ in
which $M$ can be interpreted. 
\end{defn}

\begin{exm}
The arity of the circular order is $3$, however the pseudo-arity is $2$.
\end{exm}

Note that a structure of arity $2$ cannot interpret a random $3$-hypergraph.
Similarly a structure of arity $2$ cannot interpret a tree $\left(T , \leq\right)$ with
$>1$ branch above every node.

There is another parameter called the ``dimension'' which is the number of
independent linear orders. 
\begin{exm}
The dimension of DLO is $1$, and $\DLO^2$ has dimension $1$.
We also want something with two independent linear orders to have $\dim\left(M , \leq_1,
\leq_1\right) = 2$.
\end{exm}

The point is that somehow arity $2$ is well understood, and as arity increases thins get
more complicated. Trees are dimension $1$ and arity $3$ which is the first case which is
somehow not well understood.

\end{document}

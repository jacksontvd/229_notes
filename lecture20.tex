\documentclass{amsart}
\usepackage{master}
\begin{document}
\title{Lecture 20\\ Math 229}
\author{Lecture: Professor Pierre Simon\\ Notes: Jackson Van Dyke}
\maketitle

We will continue trying to prove the sip via proving ample generics. 
What we're aiming at, which we may or may not get to today is a
characterization of having ample generics.

Recall last week we had a criterion for having a dense conjugacy class. 
Now we want to upgrade this to a criterion for having a comeager class.

\section{Comeager conjugacy class}

\subsection{Topological lemmas}

Fix $G\acts X$ polish.

\begin{defn}
We say $x\in X$ is \emph{$G$-big} if $Ux\subeq X$ is somewhere dense 
for any $1\in U \subeq G$, i.e. $\interior{\clos{U x}}\neq \emp$ for any such $U$.
\end{defn}

This will turn out to be equivalent to having comeager orbit. 

Fix a left-invariant metric on $G$. 
Recall that since $G$ is polish it has a complete metric and a left-invariant metric.
It is somehow rare that these coincide.
This is impossible for $S_\infty$, and in general
if the left-invariant metric is not also right-invariant it cannot be complete. 
In any case, we can take the left-invariant metric $d$ and then
\begin{equation}
d\left(x,y\right) = d\left(x^{-1} , y^{-1}\right)
\end{equation}
is complete. 

For $x\in X$, let
\begin{equation}
\left(x\right)_{<\e} = B_\e\left(1\right)\cdot x
\end{equation}
be the open ball of radius $\e$ (wrt d) around $1$ in $G$. Similarly if $V\subeq X$, 
\begin{equation}
\left(V\right)_{<\e} = B_\e\left(1\right)\cdot V \ .
\end{equation}

\begin{lem}
\begin{enumerate}
\item For all $A\subeq X$, $\e > 0$, 
\begin{equation}
\left(\clos{A}\right)_{<\e} \subeq \clos{\left(A\right)_{<\e}}
\end{equation}
and
\begin{equation}
\left(\interior{\clos{A} }\right)_{<\e} \subeq
\interior{\clos{\left(A\right)_{<\e}}} \ .
\end{equation}

\item If $x\in X$ is $G$-big, then
\begin{equation}
x\in \interior{\clos{\left(x\right)}_{<\e}}
\end{equation}
for all $\e > 0$.
\end{enumerate}
\end{lem}
\begin{proof}
$\left(1\right)$: Let $x\in \left(\clos{A}\right)_{<\e}$ and $U\ni x$. 
Then we need to show $U$ intersects $\left(\clos{A}\right)_{<\e}$. This is equivalent to
showing that $\left(U\right)_{<\e}$ intersects $\clos{A}$. 
In fact, we have a distance function on $X$, which is
\begin{equation}
\oline{d}\left(x,y\right) = \inf\left(d\left(g,1\right) \st gx = y\right)
\end{equation}
and $\left(A\right)_{<\e}$ is an $\e$-neighborhood for this distance.
In particular we have $\oline{d}\left(x,y\right) =
\oline{d}\left(y,x\right)$ since $d\left(g,1\right) = d\left(1,g^{-1}\right)$.
Now we certainly have $\left(U\right)_{<\e} \cap \clos{A}\neq \emp$, and being open this
means,
\begin{equation}
\left(U\right)_{<\e} \cap A \neq \emp
\end{equation}
which is equivalent to
\begin{equation}
U\cap \left(A\right)_{<\e} \neq \emp
\end{equation}
so we are done.

We skip the next part.

$\left(2\right)$:
Let $x\in X$ be $G$-big. 
Then $\interior{\clos{\left(x\right)_{<\e}}}\neq \emp$, so
\begin{equation}
V\cap \left(x\right)_{<\e} \neq \emp
\end{equation}
so $x\in \left(V\right)_{<\e}$ and
\begin{equation}
\left(V\right)_{<\e} \subeq
\left(\interior{\clos{\left(x\right)_{<\e}}}\right)_{<\e}\subeq\footnote{By
$\left(1\right)$.} \interior{\clos{\left(x\right)_{<\e}}}
\end{equation}
\end{proof}

\begin{defn}
A set $W\subeq X$ is $\e$-small if for any two nonempty open subsets $U_1 U_2\subeq W$ we
have
\begin{equation}
\left(U_1\right)_{<\e} \cap U_2 \neq \emp \ .
\end{equation}
\end{defn}

\begin{lem}
Suppose that for every $\e > 0$ the union of all $\e$-small open subsets of $X$ is dense.
Then the set of $G$-big $x\in X$ is comeager.
\end{lem}
\begin{proof}
For $\left\{V_n \st n < \om\right\}$ a countable basis of $X$. For all $n,m<\om$, by
hypothesis, $V_n$ intersects some $1/m$-small open set.
Let $\emp \neq W_{n,m}\subeq V_n$ be $1/m$ small. 
For $m<\om$, let
\begin{equation}
W_m = \bun_{n< \om} W_{n,m} \ .
\end{equation}
Then $W_m$ is open, dense in $X$. 
Let 
\begin{equation}
D_{n,m} = \binter\left\{\left(V_k\right)_{<1/m} \st V_k\subeq W_{n,m}\right\} \ .
\end{equation}
Then $D_{n,m}$ is comeager in $X$ in $W_{n,m}$, as each $\left(V_k\right)_{<1/m}$
is dense open in $W_{n,m}$. 
Let
\begin{equation}
D_m = \bun_{n<\om} D_{n,m}
\end{equation}
so $D_m \subeq W_m$ and $D_m$ is comeager in $W_m$.
Since $W_m$ is dense open we have that $D_m$ is comeager in $X$. 
Finally, let 
\begin{equation}
D = \binter_{m<\om} D_m
\end{equation}
so $D$ is comeager. 

Finally we show 
if $x\in D$, then $x$ is $G$-big. Fix $\e > 0$, then we want to show
$\interior{\clos{\left(x\right)_{<\e}}}\neq \emp$.
Take $\e = 1/m$, $x\in D_{n,m}$ for some $n < \om$. 
So $x\in \left(V_k\right)_{<1/m}$ for all $k$ such that $V_k \subeq W_{n,m}$, so 
in fact
\begin{equation}
V_k \cap \left(x\right)_{<1/m} \neq \emp
\end{equation}
for all $k$ such that $V_k \subeq W_{n,m}$. 
Therefore $\left(x\right)_{<1/m}$ is dense in $W_{n,m}$ so
$W_{n,m}\subeq \interior{\clos{\left(x\right)_{1/m}}}$.
\end{proof}

\begin{lem}
Let $x\in X$ be $G$-big. Then the action is topologically transitive iff $G\cdot x$ is
dense. 
\end{lem}

\begin{proof}
$\left(\converse\right)$: Clear. 

$\left(\implies\right)$: Let $V = \interior{\clos{G\cdot x}}$. $x$ being
$G$-big means $U\cdot x$ is somewhere dense, i.e. $G\cdot x$ is somewhere dense, i.e. $V$
is nonempty. 
Then $V$ is $G$-invariant, but now the action is topologically transitive, so for every
$U$ there is $V$ such that $g\left(V\cap U\right)\neq \emp$, so $V$ is dense, so $G\cdot
x$ is dense.
\end{proof}

The conclusion is the following. 

\begin{lem}
Suppose $G\acts X$ is topologically transitive. Then
TFAE:
\begin{enumerate}[label = (\iii)]
\item For every $\e$ the union of $\e$-small open sets is dense.
\item The set of $G$-big points is comeager.
\item There is a $G$-big point.
\end{enumerate}
\end{lem}

\begin{proof}
$\left(iii\right)\implies \left(i\right)$: Let $x\in X$ be $G$-big.
Note that for any $y\in X$, $\clos{\left(y\right)_{<\e}}$
is $2\e$ small because of the following. 
For $U,V\subeq \clos{\left(y\right)_{<\e}}$ then $U$, $V$ both intersect
$\left(y\right)_{<\e}$ so their distance is at most $2\e$. 

Note also that if $x$ is $G$-big, so is any point in any point in $G\cdot x$. 
By assumption $G\cdot x$ is dense, so 
\begin{equation}
\bun_{y\in G\cdot x} \interior{\clos{\left(y\right)_{\e / 2}}}
\end{equation}
is dense in $X$.
\end{proof}

So now we've found a collection of dense points, and now we see when they are actually an
orbit.

\begin{lem}
If $x,y\in X$ are both $G$-big and $y\in \interior{\clos{G\cdot x}}$, then $Y\in G\cdot
x$.
\end{lem}

\begin{proof}
Let $x,y$ be $G$-big, $y\in \interior{\tclosure{G\cdot x}}$.
We will construct points $g_n \in G$ and open symmetric neighborhoods $1\in U_n\subeq G$ such that
\begin{enumerate}
\item $g_{2n}^{-1} y\in \interior{\tclosure{U_{2n} x}}$
\label{item:1}
\item $g_{2n+1} \in g_{2n} U_{2n}$
\label{item:2}
\item $g_{2n+1} x \in \interior{\tclosure{U_{2n+1} y}}$
\label{item:3}
\item $g_{2n+2} \in  U_{2n+1} g_{2n+1}$
\label{item:4}
\item $g_{2n}^{-1} U_{2n+1} g_{2n} \subeq U_{2n}$
\label{item:5}
\item $g_{2n+1} U_{2n+2} g_{2n+1}^{-1} \subeq U_{2n+1}$
\label{item:6}
\item For $n\geq 1$, if $h\in U_n^3 \supeq U_n$, then $d\left(1,n\right) <
2^{-\left(n+1\right)}$.\label{item:7}
\item The set $\interior{\tclosure{U_{2n+1} y}}$ is included in a closed ball of radius
$<1/n$ around $g$. 
\label{item:8}
\end{enumerate}
The last two are somehow saying $U_n$ is small, and the others are somehow saying that we
are getting closer. If we succeed, then $\left(g_n\right)_{n < \om}$ is a Cauchy sequence
with limit $g_*$ and \cref{item:3,item:8} imply that $g_* x = y$.
To show this is Cauchy it is enough to show 
\begin{align}
d\left(g_n , g_{n+1}\right) < 2^{-n}
&&
d\left(g_n^{-1} , g_{n+1}^{-1}\right) < 2^{-n} \ .
\end{align}

Say $n = 2k+2$. Then $g_n^{-1} g_{n+1}\in U_n$ by \cref{item:2} so
\begin{equation}
d\left(g_n , g_{n+1}\right) = d\left(1 , g_n^{-1} g_{n+1}\right) < 2^{-\left(n+1\right)}
\end{equation}
by \cref{item:7}.
Then
\begin{equation}
g_n g_{n+1}^{-1} = g_n\left(g_{n+1}^{-1} g_n\right) g_n^{-1}
\end{equation}
and
\begin{equation}
g_n = u' g_{n+1}
\end{equation}
for $u'\in U_{n-1}$ by \cref{item:4}.
So
\begin{equation}
g_n\left(g_{n+1}\right)^{-1} = u' g_{n-1}\left(g_{n+1}^{-1} g_n\right)
\left(g_{n-1}\right)^{-1} \left(u'\right)^{-1} \ .
\end{equation}
Then we know $g_{n+1}^{-1} g_n \in U_n$, and by \cref{item:6} we have
$g_{n-1}\left(g_{n+1}^{-1} g_n\right)\left(g_{n-1}\right)^{-1} \in U_{n-1}$
and therefore $g_ng_{n+1}^{-1}\in U_{n-1}^3$.
So finally
\begin{equation}
d\left(g_n^{-1} , g_{n+1}^{-1}\right) = d\left(1 , g_n g_{n+1}^{-1}\right) < 2^{-n} \ .
\end{equation}
The argument is similar for $n$ odd, but we would use \cref{item:2,item:5} instead of
\cref{item:3,item:6}.

It remains to construct the $g_n$s and $U_n$s. Set $g_{0} = 1$ and $U_0 = G$.
Note \cref{item:1} holds by assumption.
Now assume we have $g_{2n}$ and $U_{2n}$. For any open $V$, 
\begin{equation}
y\in g_{2n}\cdot \interior{\tclosure{U_{2n}x}}\cap \interior{\tclosure{Vy}}
\end{equation}
so
\begin{equation}
\interior{\tclosure{Vy}}\cap g_{2n} \tclosure{U_{2n}x} \neq \emp
\end{equation}
so
\begin{equation}
\interior{\tclosure{Vy}}\cap g_{2n} U_{2n} x \neq \emp \ .
\end{equation}
Now take $U_{2n+1}$ so that \cref{item:5,item:7,item:8} hold and 
pick $g_{2n+1}\in g_{2n} U_{2n}$ such that
$g_{2n+1} x\in \interior{\tclosure{U_{2n+1} y}}$, i.e. \cref{item:2,item:3} hold. 
This can be done for any choice of $U_{2n+1}$. 

Now we take $U_{2n+2}$ and $g_{2n+2}$ similarly using the fact that $x$ is $G$-big.
\end{proof}

\end{document}

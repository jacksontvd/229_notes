\documentclass{amsart}
\usepackage{master}
\usepackage{overlay}
\usetikzlibrary{decorations.pathmorphing}
\begin{document}
\title{Lecture 3\\Math 229}
\author{Lectures by: Professor Pierre Simon\\Notes by: Jackson Van Dyke}
\date{January 29, 2019}
\maketitle

Recall we were about to finish proving the following theorem:

\begin{thm}
Let $M$ and $N$ be $\om$-categorical. 
Then $\Aut\left( M \right)$ and $\Aut\left( N \right)$
are homeomorphic iff $M$ and $N$ are bi-interpretable.
\end{thm}

\begin{Proof}[Continued proof]
Recall we proved a claim which told us that for a homeomorphism
$\sigma : \Aut\left( M \right)\lfromto{\sim}\Aut\left( N \right)$,
we get interpretations $f: M\leadsto N$ and $g:N\leadsto M$,
such that $\Aut\left( f \right) = \sigma$, and $\Aut\left( g \right) = \sigma^{-1}$.
In particular, 
\begin{align}
\Aut\left( f \right) \Aut\left( g \right) = \id_{\Aut\left( N \right)} &&
\Aut\left( g \right) \Aut\left(f \right) = \id_{\Aut\left( M \right)} \ .
\end{align}
To see that $f$ and $g$ form a bi-interpretation we just need the following:
\begin{clm}
If $s,t:M\leadsto N$ are interpretations such that
$\Aut\left( s \right) = \Aut\left( t \right)$, then $s$ and $t$ are homotopic.
\end{clm}
\begin{proof}
We need to show that the set $\left( s=t \right)$ is $0$-definable,
but since everything is $\om$-categorical,
this just means we need to show that it is $\Aut\left( M \right)$-invariant.
If $\left( x,y \right) \in \left( s=t \right)$, i.e.
$s\left( x \right) = t\left( y \right)$, 
then for $\sigma \in \Aut\left( M \right)$, we need to show 
$s\left( \sigma\left( x \right) \right) = t\left( \sigma\left( y \right) \right)$. 
We compute
\begin{align}
s\left( \sigma\left( x \right) \right) &= \Aut\left( s \right)\left( \sigma \right)
\left( s\left( x \right) \right) 
= \Aut\left( t \right)\left( \sigma \right)\left( t\left( y \right) \right)
= t\left( \sigma\left( y \right) \right)
\end{align}
and we are done.
\end{proof}
The result follows from this claim.
\end{Proof}

\section{Overview}

Recall we are in the following situation.
For $M$ an $\om$-categorical structure,
we can consider $\Aut\left( M \right)$. 
We have at least three ways we can see this group. 
The first is as a permutation group.
We have seen that this gives us exactly the same information as $M$ up to interdefinability.
Another thing we can do, is just see $\Aut\left( M \right)$ as a topological group,
and then this determines exactly $M$ up to bi-interpretability.
Finally, we could just see it as a pure group.
Here we don't get much, but sometimes, we can actually obtain its structure as a topological group
via something called the small index property.
In addition, we can sometimes go from viewing $\Aut\left( M \right)$ as a topological group
to viewing it as a permutation group, though of course
we need to insist on some property of $M$ because
there are no structures such that all bi-interpretable structures are
interdefinable.
The sort of structure for which this works is called a primitive structure, as defined below.
This is summarized in the following diagram:
\begin{equation}
\begin{tikzcd}
\Aut\left( M \right) 
\text{ as permutation group }
\arrow[leftrightsquigarrow]{r}
&
M 
\text{ up to interdefinability}
\\
\Aut\left( M \right) 
\text{ as topological group }
\arrow[leftrightsquigarrow]{r}
\arrow[bend left,dashed]{u}{\text{primitive}}
&
M 
\text{ up to bi-interpretability}
\\
\Aut\left( M \right) 
\text{ as a group }
\arrow[leftrightsquigarrow]{r}
\arrow[bend left,dashed]{u}{\text{small index property}}
&
\text{ not much\ldots }
\end{tikzcd}
\end{equation}

\begin{defn}
A permutation group $G\acts X$ is \emph{primitive} if there is no non-trivial
invariant equivalence relation on $X$.
\end{defn}

\begin{rmk}
As long as $G$ is nontrivial, primitivity implies transitivity.
\end{rmk}

\begin{defn}
A structure $M$ is primitive if $\Aut\left( M \right)\acts M$ is primitive.
Equivalently, if $M$ is $\om$-categorical, $M$ is primitive if there are no nontrivial 
$0$-definable equivalence relation on $M$.
\end{defn}

\begin{rmk}
Effectively all of the examples we have seen are primitive.
\end{rmk}

These $\om$-categorical structures have three ways of looking at them as well.
First, we can look at the \Fraisse class, $\cC$, the infinite \Fraisse limit, $M$,
and the Polish group $\Aut\left( M \right)$.
The study of the first involves mostly combinatorics, 
the study of the second is mostly model theory, and the third is descriptive set theory.
In this course we will study the first, then the third, 
and if there is time we will study the second object.

\section{Subgroups and supergroups of $\Aut\left( M \right)$}

Recall we have seen that $\Aut\left( M \right) \leq \Sym\left( M \right)$ is closed.
Conversely, any closed subgroup of $\Sym\left( M \right)$ is equal to $\Aut\left( \tilde M \right)$
for some structure $\tilde M$.
Similarly, if $G \leq \Aut\left( M \right)$ is closed, 
then the structure $\tilde M$ such that $G = \Aut\left( \tilde M \right)$
is an expansion of $M$ (probably more generally, but certainly if $M$ is $\om$-categorical).
Conversely, if $\tilde M$ is an expansion of $M$, then 
$\Aut\left( \tilde M \right)\leq \Aut\left( M \right)$ is closed.
So we have a correspondence between closed subgroups and expansions of $M$,
and in particular, for oligomorphic subgroups the expansions are $\om$-categorical.
The issue is that the expansion corresponding to some subgroup is not unique,
so this is not a bijection.

We can instead look at supergroups,
i.e. any group $G$ such that
$\Aut\left( M \right) \leq G\leq \Sym\left( M \right)$
where $G$ is closed inside of $\Sym\left( M \right)$.
If $M$ is $\om$-categorical, we have a bijection between 
reducts of $M$ up to inter-definability
and such supergroups $G$.
This is often used to classify reducts of $\om$-categorical structures.

\subsection{Classifying reducts}

\begin{exm}
Reducts of countable dense linear orders (DLO), i.e.
reducts of $\left( \QQ , \leq \right)$ were classified by Cameron:
\mathtabular{
\begin{tabular}{|C|C|}
\hline
M & \Aut\left( M \right)\\
\hline\hline
\left( \QQ , \leq \right) & \text{order-preserving permutations of } \QQ \\
\hline
\left( \QQ , B\left( x,y,z \right) \right) & \text{order-preserving or reversing permutations of } \QQ \\
\hline
\left( \QQ , C\left( x,y,z \right) \right) & \text{order preserving up to `cutting and pasting' pieces}\\
\hline
\left( \QQ , S\left( x,y,s,t \right) \right)
&\text{(closure of) the group generated by the two previous ones}\\
\hline
\left( \QQ , = \right) & \Sym\left( \QQ \right)\\
\hline
\end{tabular}
}
A first observation about the second row, is that there is no point in even looking
for binary relations.
These permutations are $2$-transitive, meaning
any two distinct points can be sent to any distinct points, i.e.
it is transitive on pairs of distinct points.
Therefore we should be looking for a ternary relation.
As it turns out, the appropriate relation is betweenness $B\left( x,y,z \right)$.

For the third line, we have that
$C\left( x,y,z \right)$ is the circular order.
So we somehow bring together $\pm \infty$ 
and then we insist on three points being in some order on the circle.
These automorphisms are also $2$-transitive but for a different reason.

The final nontrivial one is $3$-transitive, since we are allowing the automorphisms
to sort of flip the circle.
The relation on this structure is the
separation relation.
This holds if to travel from $s$ to $t$ you have to cross 
$x$ or $y$ and vice versa.
\end{exm}

\begin{exr}
Show the bijection in the second row.
Also show that $\left( \QQ , B \right)$ is homogeneous in this language. 
\end{exr}

\begin{exm}
Consider the reducts of the random graphs.
The classifications is due to Thomas:
\mathtabular{
\begin{tabular}{|C|c|}
\hline
\cM & $\Aut\left( \cM \right)$\\
\hline\hline
\left( M , R \right)
& graph automorphisms
\\
\hline
\left( M , R_4\left( w,x,y,z \right) \right)
& automorphisms, 
or they exchange all edges and non-edges
\\
\hline
\left( M , R_3\left( x,y,z \right) \right)
& switching automorphisms of the graph
\\
\hline
\left( M , R_5\left( x_1 , \ldots , x_5 \right) \right)
& composition of the $2$ previous ones
\\
\hline
\left( M , = \right)
& $\Sym\left( M \right)$
\\
\hline
\end{tabular}
}
Given a partition of $M$ in two classes $E_1$ and $E_2$,
the switch between $E_1$ and $E_2$ is the operation which exchanges 
edges and non-edges between $E_1$ and $E_2$ while leaving the $E_i$ unchanged.
This is an automorphism up to a switch.
This is sort of like a circular order.
The following is such an example:
\begin{align}
\begin{overlay}
\pict{switch1.pdf}{0.1}
\end{overlay}
&&\leadsto &&
\begin{overlay}
\pict{switch2.pdf}{0.1}
\end{overlay}
&&\leadsto &&
\begin{overlay}
\pict{switch1.pdf}{0.1}
\end{overlay}
\end{align}
To go from the first to the second, we swap the bottom and right vertices,
and to go from the second to the third, we perform a 
`switch automorphism' where $E_1$ consists of the top vertex, and $E_2$ consists of the other two.
The switching automorphisms are $2$-transitive, so again we should expect at least
a ternary relation here.
We get that the proper relation $R_3$ is parit of edges, as is $R_4$.
Then parity of the number of edges between $5$-points, $R_5$,
gives us the final nontrivial structure.
\begin{wrn}
$\left( M , R_4\left( w,x,y,z \right) \right)$
is not homogeneous in this language.
If we want something which is homogeneous we would just have to put
a ternary relation which tells us if those two edges are the same or different.
$\left( M , R_5\left( x_1 , \cdots , x_5 \right) \right)$ is also not homogeneous since
it is not $4$ or $3$ transitive.
\end{wrn}
\end{exm}

\end{document}

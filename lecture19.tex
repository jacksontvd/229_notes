\documentclass{amsart}
\usepackage{master}
\begin{document}
\title{Lecture 19\\ Math 229}
\author{Lecture: Professor Pierre Simon\\ Notes: Jackson Van Dyke}
\maketitle

Recall we defined ample generics for general actions $G\acts X$. 
This means there is a comeager orbit on $X^n$ for all $n$. 
In particular we are interested in $G = \Aut\left(M\right)$ where $G\acts G$ by
conjugation. 
Then the goal is to give a criterion for ample generics to exist. 
Then the point is that this implies sip which tells us that the topology is given by the
pure group.

Also recall we defined a group action to be topologically transitive if there exists some dense orbit. 
This is weaker in the sense that it doesn't have to be comeager, just dense, and not on
finite tuples but just $G\acts G$.

\section{JEP}

\begin{defn}
A \emph{partial isomorphism} from $A$ to $B$ is an isomorphism from a finitely generated
substructure of $A$ to a finitely generated substructure of $B$. 
\end{defn}

If $\cK$ is an amalgamation class (for us typically a \Fraisse class) of structures we
let $\cK\left(n\right)$ be the class of tuples which are of the form $\left(A , f_1 ,
\ldots , f_n\right)$,
where $A\in \cK$ and the $f_i:A\paut A$ are partial automorphisms.
We say that
\begin{equation}
\left(A , f_1 , \cdots , f_n\right) \leq \left(B , g_1 , \cdots , g_n \right)
\end{equation}
if $A\leq B$ and $f_i \subeq g_i$.
We say that $\cK\left(n\right)$ has JEP if for any 
\begin{equation}
\left(A , \oline{f}\right) , \left(B ,\oline{g}
\right)\in \cK\left(n\right)
\end{equation}
there is $\left(C, \oline{h}\right)$ with 
\begin{equation}
\begin{cd}
\left(A , \oline{f}\right)\arrow{dr}{\leq}& \\
& \left(C , \oline{h}\right)\\
\left(B , \oline{g}\right)\arrow{ur}{\leq}
\end{cd} \ .
\end{equation}

\begin{exm}
Consider the class of graphs. 
Then $\cK\left(n\right)$ has JEP. In particular we can just take $C = A\dun B$ with no
additional edges.
\end{exm}

\begin{cexm}
Let $\cK$ be equivalence relation with two classes. 
$\cK\left(1\right)$ does not have JEP. Take $f$ switching the two classes and $g$ not
switching them.
\end{cexm}

\begin{exr}
Prove that (for $\cK$ the circular order) $\cK\left(1\right)$ does not have JEP.
\end{exr}

As it turns out JEP is equivalent to having a
dense conjugacy class. This isn't really a deep theorem, it just depends on understanding
what this really means. 

\begin{prop}
Let $\cK$ be a \Fraisse class, $M$ its limit, and $G = \Aut\left(M\right)$. Fix $n < \om$. 
TFAE:
\begin{enumerate}[label = (\iii)]
\item The class $\cK\left(n\right)$ has JEP.
\item The action of $G\acts G^n$ by conjugation is topologically transitive.
\item There is a dense conjugacy class of an $n$-tuple of $G$.
\end{enumerate}
\end{prop}

\begin{proof}
$\left(ii\right)\iff \left(iii\right)$: We have already seen this. 

$\left(i\right)\implies\left(ii\right)$: We prove for $n = 1$. There is really no
formal difference, but the notation and intuition is easier. Let $U$, $V$ be nonempty open
subsets of $G$. WLOG 
\begin{align}
U  = U_f = \left\{\sigma \in G \st \sigma \supeq f\right\}
&&
V = V_g
\end{align}
for $f$ and $g$ partial isomorphisms. 
Let $A\subeq M$ be finitely generated containing domain and range of $f$, and $B\subeq M$
be finitely generated containing domain and range of $g$. I.e.
$\left(A , f\right) , \left(B , g\right)\in \cK\left(1\right)$. 
Since we are assuming JEP we can amalgamate them to get some $\left(C , h\right)$ with
embeddings
\begin{align}
\phi_1\left(A , f\right) \fromto \left(C , h\right)&&
\phi_2\left(B , g\right) \fromto \left(C , h\right) \ .
\end{align}
Let $\tau_1$ extend $\phi_1$, $\tau_2$ extend $\phi_2$, and $\e\in G$ extend $h$.
Then 
\begin{align}
\tau_1^{-1}\e \tau_1 \supeq f
&& \tau_2^{-1}\e \tau_2 \supeq g
\end{align}
so
\begin{equation}
\tau_2 \tau_1 \left(U\right)\cap V \neq \emp \ .
\end{equation}

$\left(ii\right)\implies\left(i\right)$: Let $\left(A , f\right), \left(B , g\right) \in
\cK\left(1\right)$. Again, let
\begin{align}
U = U_f \subeq G && V = V_f
\end{align}
By assumption there is $\tau\in G$ such that $\tau\left(U\right)\cap V\neq \emp$.
Let $\e\in \tau\left(U\right)\cap V$.
Let $C$ be the structure generated by $\tau\left(A\right)\un B$, and let $h =
\restr{\e}{C\cap \e^{-1}\left(C\right)}$. 
Then 
\begin{align}
\restr{\tau}{A}\left(A , f\right) \fromto \left(C , h\right) &&
\id: \left(B , g\right) \fromto \left(C , h\right)
\end{align}
are embeddings. 
This proves JEP.
\end{proof}

\section{Stronger than JEP, weaker than Amalgamation}

Amalgamation turns out to be too strong. Consider the case of graphs. If we have $\left(A
, f\right)$, $\left(B , g\right)$
overlapping in $\left(C , h\right)$, these cannot amalgamate in general because simple
because these might disagree. But if, for example, $h$ is surjective, i.e. it is a proper
isomorphism then we can always amalgamate. 
More formally this property says that for every $\left(A , f\right)$ there is some
$\left(B , g\right)$ such that
\begin{equation}
\begin{cd}
& \left(C , h\right)\\
\left(B , g\right)\arrow{ur}\arrow{dr}\\
& \left(D , k\right)
\end{cd}
\end{equation}
can be amalgamated. But this is even too strong. We instead ask for the following
property. For any $\left(A , f\right)$ we have some $\left(B , g\right)$ such that we can
extend it to $\left(C , h\right)$ and $\left(D,k\right)$ such that
\begin{equation}
\begin{cd}
& \left(C , h\right)\\
\left(A , f\right)\arrow{ur}\arrow{dr}\\
& \left(D , k\right)
\end{cd}
\end{equation}
can be amalgamated.
This is called the weak amalgamation property or sometimes the
existential amalgamation property.\footnote{Professor Simon things this should really be
called the weak existential amalgamation property, or WEAP.}

\begin{defn}
We say $\cK\left(n\right)$ has the \emph{existential amalgamation property} (EAP) if for every
$\left(A , f\right)$ there is $\left(B , g\right)$ which is definite on $A$, which means 
\begin{enumerate}[label = (\iii)]
\item $\left( A , f\right) \leq \left(B , g\right)$
\item For every embedding 
\begin{align}
\phi_1\left(B , g\right) \fromto \left(C_1 , h_1\right)
&& \phi_2\left(B , g\right) \fromto \left(C_2 , h_2\right)
\end{align}
there is some $\left(D , r\right)$ with embeddings 
\begin{align}
\psi_1 : \left(C_1 , h_1\right)\fromto \left(D , r\right)&&
\psi_2 : \left(C_2 , h_2\right)\fromto \left(D , r\right)
\end{align}
such that
\begin{equation}
\restr{\psi_1\comp \psi_2}{A} = \restr{\psi_2 \comp \psi_1}{A} \ .
\end{equation}
\end{enumerate}
In other words we have the diagram:
\begin{equation}
\begin{tikzcd}
&& \left(C_1 , h_1\right)\arrow[dashed]{dr}\\
\left(A , f\right)\arrow[dashed]{urr}\arrow[dashed]{drr}\arrow[hook]{r} &
\left(B , g\right)\arrow{ur}\arrow{dr}&&
\left(D , k\right)
\\
&& 
\left(C_2 , h_2\right)
\arrow[dashed]{ur}
\end{tikzcd}
\end{equation}
where the dashed arrows must commute. 
\end{defn}

\begin{prop}
For $\left(A , \oline{f}\right)\leq \left(B , \oline{g}\right)\in \cK\left(n\right)$
TFAE:
\begin{enumerate}[label = (\iii)]
\item $\left(B , g\right)$ is definite on $A$.
\item For any two nonempty open subsets $V_1, V_2 \subeq U_g$, we have
\begin{equation}
\left(U_{\restr{\id}{A}} \cdot V_1\right) \cap V_2 \neq \emp
\end{equation}
where $\cdot$ denotes conjugation.
\end{enumerate}
\end{prop}

\end{document}

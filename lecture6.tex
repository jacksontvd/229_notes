\documentclass{amsart}
\usepackage{master}
\begin{document}
\title{Lecture 6: EPPA\\Math 229}
\date{February 7, 2019}
\author{Lecture: Professor Pierre Simon\\Notes: Jackson Van Dyke}
\maketitle

Today we start chapter 2: combinatorics.\footnote{
Chapter 3 will be automorphism groups.}

\section{Introduction}

Today we will talk about the extension property for partial automorphisms (EPPA).
Consider some \Fraisse class $\cC$ (finite relational).
We know how to associate this to a homogeneous structure $\cM$. 
Then a natural question is if we can find elements in $\cC$ which are already homogeneous,
or at least have many automorphisms.

\begin{defn}
We say $\cM$ is smoothly approximable if for every $A\in \cC$ there is homogeneous $B\in \cC$ for
which $A\inj B$.
\end{defn}

This is the best possible situation, but it rarely happens.

\begin{thm}
The smoothly approximable finitely homogeneous structures are precisely the
stable finitely-homogeneous structures.\footnote{Essentially these are finite covers of
things like $\left( M^n, = \right)$. Recall the `zoo' lecture.}
\end{thm}

\begin{defn}
$\cC$ has EPPA if for every $A\in \cC$ there is $B\in \cC$ such that $A\inj B$ and every
partial automorphism of $A$ extends to an automorphism of $B$.
\end{defn}

When this is true, $B$ is called an EPPA witness for $A$.
This turns out to be satisfied very often.

\begin{rmk}
Note that EPPA and the joint embedding property (JEP) together imply the amalgamation property (AP).
\end{rmk}

\begin{proof}
Consider some $B$ and $C$ overlapping in $A$. Then JEP implies that there is some $D_0$
such that $B$ and $C$ embed 
and $\rho$ identifies $A$ living inside of these. 
Then this embeds in some EPPA witness $D_1$ of $D_0$, 
so $\rho$ extends to an automorphism $\sigma$.
Then $D_1$ amalgamates $\sigma\left( B \right)$ and $C$ over $A$.
\end{proof}

\section{Main theorems}

First we see some examples.

\begin{exm}
A class of linearly ordered structures cannot have EPPA.
\end{exm}

\begin{exm}
$\left( M , = \right)$ has EPPA.
Consider some partial bijection, then take the set $A$ itself as an EPPA witness,
and of course you can extend any injection from $A\fromto A$ to a bijection $A\fromto A$. 
\end{exm}

\begin{exm}
Consider the class of bipartite graphs $\left( G , U , V , R \right)$.
Consider some partial automorphism of $U$. First we extend this to a bijection of $U$, and
expand $V$ to $V'$ as follows.
For every $U_0 \subeq I$, extend $V$ to $V'$ by adding $\abs{V}$ many points related
exactly to $U_0$. So we have somehow symmetrically expanded $V$. This will be motivation
for the proof of the next theorem.
\end{exm}

\begin{thm}
The class of finite graphs has EPPA.
\end{thm}

\begin{proof}[Proof of Hubi\v{c}ka, Kone\v{c}n\'{y}, and Ne\v{s}et\v{r}il]
Let $A = \left( V\left( A \right), E\left( A \right) \right)$ be a finite graph.
We want to send $A$ into some very symmetric thing which somehow contains every possible
type over $A$.
Define the EPPA witness $B$ as follows. 
Define
\begin{equation}
V\left( B \right) = \left\{ \left( v , f_v \right) \st v\in V\left( A \right), f_v :
V\left( A \right) \minus \left\{ v \right\} \fromto \left\{ 0,1 \right\} \right\}
\end{equation}
and let $\left( \left( v , f_v \right)\right) \in E\left( B \right)$ iff
$v\neq w$, and $f_v\left( w \right) \neq f_w\left( v \right)$. $B$ has two types of
automorphisms that we're concerned over. First, any bijection $\phi$ of $A$ extends canonically
to an automorphism $\tilde \phi$ of $B$:
\begin{equation}
\tilde \phi : \left( v ,f_v \right) \mapsto \left( \phi\left( v \right) , f_v \comp
\phi^{-1} \right) \ .
\end{equation}
The second type is as follows. For any pair $\left( v,w \right) \in V\left( A
\right)^2$, where $v\neq w$, we get $\theta_{v,w} \in \Aut\left( B \right)$ which maps
$\theta_{v,w} : \left( v , f_v \right) \mapsto \left( v , f_v' \right)$ where $f'_v$ is
defined as
\begin{equation}
\begin{cases}
f_v'\left( w \right) = 1 - f_v\left( w \right) \\
f_v'\left( x \right) = f_v\left( x \right) & x\neq w
\end{cases}
\end{equation}
and
\begin{align}
\left( w , f_w \right) \mapsto \left( w , f_x' \right) &&
\left( x , f_x \right)\mapsto \left( x , f_x \right) , x\neq \left\{ v,w \right\} \ .
\end{align}

Now we need to embed $i : A\inj B$ as follows. Send $v$ to $\left( v , f_v \right)$ where we
choose $f_v$ inductively. Each each step, when we decide if we want an edge, just choose
the function to be the same or different correspondingly.

Now we extend partial automorphisms. Let $\rho: A\paut A$ be a partial automorphism.
First extend $\rho$ to a bijection $f = \hat \rho$ of $A$. 
Consider the automorphism $\tilde f$ of $B$.
$\tilde f$ and $\rho$ coincide on the first coordinate, wherever $\rho$ is defined inside
$B$. The reason $\tilde f$ might not extend $\rho$ is because they might not agree on the
second coordinate. Write:
\begin{align}
\rho: \left( v , f_v \right)  \mapsto \left( w , f_w \right) &&
\left( v , f_v \right) \mapsto \left( w' , f_{w'} \right) \ .
\end{align}
So look at $v\in \domain \rho$, and let $i\left( v \right) = \left( v , f_v \right)$, 
and $\left( \rho\left( v \right) , g \right) = i\left( \rho\left( v \right) \right)$.
Then $\left( \rho\left( v \right) , h \right) = \tilde f\left( \left( v , f_v \right)
\right)$. 
If for some $x$, $g\left( x \right) \neq h\left( x \right)$, 
compose $\tilde f$ with $\theta_{\rho\left( v \right) , x}$.
But this $\theta$ might create a mistake if $x\in \im\rho$.
If we do this inductively, we need to see that we are not going to change it again,
and compose by $\theta_{x,\rho\left( v \right)}$ later on. But the only way this can
happen, is if $x\in \im\rho$, say $x = \rho\left( y \right)$, but then since $\rho$ and
$\tilde f$ are both automorphisms, this will not be a problem.
\end{proof}

\begin{thm}
The class of finite $K_n$-free graphs has EPPA.
\end{thm}

\begin{proof}
Let $A$ be a finite $K_n$-free graph and let $B_0$ be an EPPA witness for $A$ as a graph.
This might have $K_n$s so we want to lift this to something which doesn't have them.
For every point of $B_0$, we will look at all of the $K_n$s that contain that
point, and assign a number to each of these.
For every pair, we will look at their lift, and decide if they have an edge between them.
They will only have an edge if they disagree on all of the triangles they are both in.
In particular, this will ensure there are no $K_n$s.

We now do this explicitly. Define $B$ to consist of
\begin{equation}
V\left( B \right) = \left\{ \left( v,t_v \right) \st v\in V_0 \right\}
\end{equation}
where $t_v$ is a function from the set of $K_n$s containing $v$ to $\left\{ 0, \cdots ,
n-1\right\}$. Then $\left( v , t_v \right)$ and $\left( w , t_w \right)$ have an edge iff
$\left( v,w \right)\in E\left( B_0 \right)$, and $t_v\left( c \right) \neq t_w\left( c
\right)$ for every $c$ containing $v$ and $w$. Write $\pi: B\fromto B_0$ for the
projection. Note that, as we would hope, $B$ is a
$K_n$-free graph. If $C\subeq B$ is a $K_n$, then $\pi\left( c \right)$ is a $K_n$ in
$B_0$, and the value on $\pi\left( c \right)$ of points in $C$ must all be distinct, but
this is impossible.
\begin{center}
\textbf{To be continued\ldots}
\end{center}
\end{proof}

\end{document}

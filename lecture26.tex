\documentclass{amsart}
\usepackage{master}
\begin{document}
\title{Lecture 26\\ Math 229}
\author{Lecture: Professor Pierre Simon\\ Notes: Jackson Van Dyke}
\date{April 30, 2019}
\maketitle

In the last week we will treat the stable case.

\section{Rank}

Let $M$ be an $\om$-categorical structure. 

\begin{defn}
For a definable set $D$ and ordinal $\al$, we define $\rank\left(D\right)\geq \al$
inductively:
\begin{enumerate}[label = (\iii)]
\item $\rank\left(D\right) \geq 0$ if $D\neq \emp$,
\item $\rank\left(D\right)\geq \al+1$ if in $M^{eq}$ there is a definable family 
$\left(X_{\oline{a}}\right)$, where $\oline{a}\satisfies \phi\left(x\right)$,
of subsets of $D$ which is $k$-inconsistent\footnote{I.e. the
intersection of any $k$ of them is empty.} for some $k$ for $k < \om$ and
$\rank\left(X_i\right)\geq \al$ for all $i$. 
\label{item:star}
\item $\rank\left(D\right)\geq \lam$, $\lam$ limit if $\rank\left(D\right)\geq \al$ for
all $\al < \lam$.
\end{enumerate}
We say $\rank\left(D\right) = \emp$ if $\rank\left(D\right) \geq \al$ for all $\al$.
\end{defn}

\begin{exm}
\begin{itemize}
\item If $\left(M , \leq\right)$ is DLO then $\rank\left(M\right) = 1$.
\item The random graph has rank $1$.
\item If $M$ is $\om$-stable then the rank is the Morley rank.
\item The generic tree has infinite rank.
\end{itemize}
\end{exm}

Note that \ref{item:star} is equivalent to the following:
there is a definable 
\begin{cd}
D'\arrow{d}{\pi}\\ D
\end{cd}
with finite fibers and a definable equivalence relation on $D'$ with infinitely many
classes of rank $\geq \al$.

\subsection{Properties of rank}

First note that:
\begin{equation}
\rank\left(a / A\right) = \rank\left(\type\left(a / A\right)\right) = 
\min\left\{\rank\left(D\right)\st D A\text{-definable, } a\in D\right\} \ .
\end{equation}

We have the following properties:

\begin{enumerate}[label = \numbers.]
\item $\rank\left(a/ A\right) = 0$ iff $a\in \algcl\left(A\right)$
\item $\rank\left(D_1 \un D_2\right) = \max\left\{\rank\left(D_1\right) ,
\rank\left(D_2\right)\right\}$
\item If $D$ is definable over $S$ there is $a\in D$ with $\rank\left(a / A\right) =
\rank\left(D\right)$.
\item If $\rank\left(a / bc\right)$, $\rank\left(b/c\right)$ are finite, then 
$\rank\left(ab/c\right) = \rank\left(a/bc\right) + \rank\left(b/c\right)$.
\end{enumerate}

If $\rank\left(M\right) < \om$ we write
$a\indfrom_c b$ if
\begin{align}
\rank\left(ab/c\right) = \rank\left(a/c\right)
+ \rank\left(b/c\right) 
&\iff 
\rank\left(a/bc\right) = \rank\left(a/c\right)\\
&\iff 
\rank\left(b/ac\right) = \rank\left(b/c\right)\\
&\iff 
b\indfrom_c a \ .
\end{align}
Equivalently 
$a\indfrom_E bc$ iff $a\indfrom_{Ec} b$ and $a\indfrom_E c$.

Note that if $\rank\left(D\right) = 1$, then $\algcl$ has the exchange property on $D$.
This means for $A\subeq D$, $a,b\in D$ we have that
$a\in \algcl\left(Ab\right)\minus \acl\left(A\right)$ iff
$b\in \algcl\left(Aa\right)\minus \acl\left(A\right)$ which is equivalent to
\begin{equation}
\rank\left(Aab\right) = 
\rank\left(A\right) + 1
= \rank\left(Ab\right) = \rank\left(Aa\right)\ . 
\end{equation}
What this means is that $\algcl$ defines a \emph{pregeometry} on $D$.

A pregeometry is a closure operation that satisfies certain properties:
\begin{itemize}
\item $A\subeq B \implies \acl\left(A\right)\subeq \acl\left(B\right)$,
\item $\acl\left(\acl\left(A\right)\right) = \acl\left(A\right)$:
\item exchange property
\item maybe another axiom\ldots
\end{itemize}
The associated dimension is the rank.

\begin{exm}
$\FF_p$ vector space is rank one and has a nontrivial $\algcl$.
\end{exm}

If $\rank\left(D\right) = 1$ and $D$ is primitive then $\acl$ defines a \emph{geometry} on
$D$. This means that we additionally have
\begin{itemize}
\item $\acl\left(\emp\right)  = \emp$ and
\item $\acl\left(\left\{a\right\}\right) = \left\{a\right\}$ for $a\in D$.
\end{itemize}

\subsection{Main theorem}

\begin{thm}
If $M$ is finite homogeneous, $\rank\left(M\right) = 1$, 
there is a unique $n$-type of an independent tuple for all $n$,
then it has trivial geometry.
\end{thm}

\begin{Proof}
Assume $M$ has $\rank = 1$ and is primitive for $A\subeq M$ finite. Let 
\begin{equation}
\left(A\right)_+ = 
\closure{A} - \bun_{B\subneq A} \closure{B} \ .
\end{equation}
Assume there is $\left\{a,b\right\}$ such that $\left(\left\{a,b\right\}\right)_+\neq
\emp$.
\begin{lem}
If $A\un \left\{a\right\}$ is independent, $b\in \left(A\right)_+$, 
$c\in \left(\left\{a,b\right\}\right)_+$, then $c\in \left(A\un\left\{a\right\}\right)_+$.
\end{lem}
\begin{proof}
Assume there is $B\subneq A\un \left\{a\right\}$, $c\in \closure{B}$. 
If $B\subeq A$, $c\in \closure{A}$, by exchange, $a\in \closure{ \left\{b,c\right\}}\subeq
\closure{A}$ which is a contradiction. 

Otherwise, $B = B'\un\left\{a\right\}$ for $B'\nsubeq A$. Let $d\in A\minus B$.
By exchange $b\in \closure{ \left\{a,c\right\}}\subeq \closure{B}$, 
\begin{equation}
b\in \closure{A\minus \left\{d\right\}\un \left\{d\right\}}\minus \closure{A\minus
\left\{d\right\}} \ .
\end{equation}
Again by exchange,
\begin{equation}
d\in \closure{A\minus \left\{d\right\}\un \left\{b\right\}}\subeq 
\closure{\ubr{A\minus \left\{d\right\}\un \left\{a\right\}}{\supeq B}}
\end{equation}
so certainly $B$ is in this closure.
This contradicts the independence of $A\un \left\{a\right\}$.
\end{proof}
\begin{lem}
For any independent $A$, $A_0 , A_1\subeq A$ ($A_0\neq A_1$) then
\begin{equation}
\left(A_0\right)_+ \cap \left(A_1\right)_+ = \emp \ .
\end{equation}
\end{lem}
\begin{proof}
True iff $A_0\subeq A_1$ or $A_1 \subeq A_0$. Otherwise we can find $c_0\in A_0\minus A_1$
or $c_1\in A_1 \minus A_0$ and contradict independence. 

Let $d\in \left(A_0\right)_+ \cap \left(A_1\right)_+$. Then by exchange
\begin{align}
c_0\in \closure{A_0\minus\left\{c_0\right\}\un \left\{d\right\}}&&
c_1\in \closure{A_1\minus\left\{c_1\right\}\un \left\{d\right\}}
\end{align}
so $A\subeq \closure{A\minus \left\{c_0 , c_1\right\}\un\left\{d\right\}}$
which is a
contradiction to the fact that $\rank\left(A\right) = \abs{A}$.
\end{proof}

Now let $A$ be an independent set of size $n$ by induction (by first lemma), for all
nonempty $B\subeq A$, $\left(B\right)_+\neq \emp$.
By the second lemma we get $2^n$ many one-types over $A$. 
This contradicts finite homogeneity.

In general assume $n \geq 2$ is minimal such that 
\begin{equation}
\left(\left\{a_1 , \ldots , a_n\right\}\right)_+\neq \emp \ .
\end{equation}
Then add $a_1 , \ldots , a_{n-2}$ as constraints to the language and apply
the previous case.
\end{Proof}

\begin{defn}
A definable set $D$ is \emph{strongly-minimal} if any parameter-definable subset of $D$ is
finite or co-finite. 
\end{defn}

If $D$ is strongly-minimal then $\rank\left(D\right) = 1$ and any 
two independent $n$-tuples have the same tuple.

The conclusion is that a strongly minimal primitive set definable in a finitely
homogeneous structure is an indiscernible set, i.e. isomorphic to $\left(A , =\right)$.

\section{Binary structure}

\begin{prop}
Let $M$ be a binary structure. Then $\rank\left(M\right)< \om$.
\end{prop}

\begin{proof}
Assume $\rank\left(M\right)\geq \om$. Fix $N< \om$ sufficiently large. 
We can find:
\begin{itemize}
\item An increasing family $\left(c\left(n\right) \st n < N\right)$ of finite tuples,
\item a $c\left(n\right)$-definable set $D_n$ transitive over $c\left(n\right)$,
\item a $c\left(n\right)$ definable family $\left(X_t^n \st t\in E_n\right)$,
$k\left(n\right)$-inconsistent, where there exists some $t$ such that $D_{n+1}\subeq
X_t^n$.
\end{itemize}
\begin{clm}
We can assume $X_t^n\cap X_{t'}^n$ is finite for $t\neq t'$.
\end{clm}
To see this replace $X_t^n$ by the max infinite intersections $X_{t_1}\cap \ldots \cap
X_{t_k}^n$.
\begin{clm}
For each $n$ there are $x,y\in D_n$ such that for no $t\in R_n$ do we have both $x_n\in
X_t$, $y_n\in X_t$.
\end{clm}
We assume this for now.

For each $n$ let $\phi_n\left(x,y\right)$ say that for some $t$ we have $x\in X_t$, $y\in X_t$.
By binary, on $D_n$ we have that $\phi_n\left(x,y\right)$ is equivalent to a
$\emp$-definable formula $\psi_n\left(x,y\right)$.
We have $\not \psi_n\left(x_n , y_n\right)$, and for $m < n$ we have
$\psi_m \left(x_n , y_n\right)$ so the $\psi_n$s are pairwise distinct which is
impossible.
\end{proof}

\end{document}

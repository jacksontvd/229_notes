\documentclass{amsart}
\usepackage{master}
\begin{document}
\title{Lecture 10\\Math 229}
\date{February 21, 2019}
\author{Lecture: Professor Pierre Simon\\Notes: Jackson Van Dyke}
\maketitle

We will continue with the partite construction, and just as we did for EPPA, prove the
Ramsey property for various structures.
We will see that the argument for Ramsey statements is easier to adapt, and indeed
we know more Ramsey statements than we do EPPA statements.

\section{$K_n$-free graphs}

\begin{thm}[Ne\v{s}et\v{r}il-R\"odl]
the class of finite ordered $K_n$-free graphs is Ramsey.
\end{thm}

\begin{proof}
Just as with EPPA, we will start with a Ramsey object for a larger class, i.e. all ordered
graphs, and fix it by removing some edges to get rid of $K_n$s.
Let $A$, $B$ be finite ordered $K_n$-free graphs. Let $r<\om$.
First take $C_0$ to be such that $C_0\fromto \left( B \right)_r^A$ just as ordered graphs. 
The issue is this might have $K_n$s.

We build a sequence of $d+1$ pictures $P_i$ where $d = \abs{\binom{C_0}{A}}$.
Enumerate the copies of $A$ in $C_0$ as
\begin{equation}
\binom{C_0}{A} = \left\{ A_1 , \cdots , A_d \right\} \ .
\end{equation}
First we build the $0$th picture, $P_0$, which is a disjoint (free) union of copies of $B$
in $C_0$ (as partite graphs).
In particular, we have a homomorphism $P_0\fromto C_0$ which respects partitions.

We now build the $P_i$ inductively. Assume we have built the picture $P_{i-1}$. 
Consider $A_i$, and let $B_i$ be the preimage $\pi^{-1}\left( A_i \right)$ where 
$\pi: P_{i-1}\fromto C_0$ is the canonical homomorphism.
The partite lemma applied to $A_i$, $B_i$ gives us $C_i$ such that 
$C_i \fromto \left( B_i \right)_r^{A_i}$ as partite graphs.
Note that $B_i$ has a homomorphism to $A_i$ and the partite lemma preserves this property, 
i.e. it does not put edges between two parts if there is no such edge in $B_i$.
Hence $C_i$ is $K_n$-free.
Now we build $P_i$ as follows. For every copy of $B_i$ in $C_i$
we extend this to a copy of $P_{i-1}$, and then amalgamate those freely over the $B_i$.
This does not create $K_n$s, and we still have a homomorphism $\pi: P_i \fromto C_0$.

Then the statement is that $P_d$ is our Ramsey object. As before we show by downward induction that if
$\binom{P_d}{A}$ is $r$-colored, there is a copy of $P_0$ in $P_d$ in which any two copies
of $A$ with the same projection to $C_0$ have the same color.
This gives us a coloring of $\binom{C_0}{A}$. By the Ramsey property of $C_0$, we find a
homogeneous copy of $B$. The corresponding copy of $B$ in $P_0$ is homogeneous.
\end{proof}

\begin{rmk}
This is somehow the same as the proof for ordered graphs from last lecture, except instead 
of starting with
the usual Ramsey theorem which gives $N\fromto \left( \abs{B} \right)^{\abs{A}}_r$, we
now start with $C_0\fromto \left( B \right)^A_r$ as ordered graphs.
But the rest of the steps are effectively the same.
\end{rmk}

\section{Irreducible structures}

In fact the techniques of the previous proof work for any structures which behave in a
similar way to ordered graphs. This notion is captured by the following more general
statement.

\begin{defn}
A relational structure is \emph{irreducible} if any two distinct elements belong to some
relation.
\end{defn}

Let $\cF$ be a family of finite irreducible structures in a relational language $\cL$. Let
$\Forb\left( \cF \right)$ be the class of $\cL$-structures which omit all structures which
have members of $\cF$ as induced substructures.
Then the same proof as above will give that the class of ordered
expansions of members of $\Forb\left( \cF \right)$ is Ramsey.

\section{Metric spaces}

The class of metric spaces is not covered by the previous theorem because bad $n$-cycles
for $n\geq 4$ are not irreducible.
The point is that somehow for $4$-cycles (and larger), opposite vertices have no relation, but we want
to sometimes forbid this to get the class of metric spaces, 
so metric spaces can't be defined only by forbidding things. 
Nonetheless we have the following:

\begin{thm}
The class of finite ordered metric spaces is Ramsey.
\end{thm}

\begin{proof}
Let $A$, $B$ be finite ordered metric spaces, and $r < \om$. Let
\begin{equation}
n = \frac{\max \text{ length in } B}{\min \text{ length in } B} + 1 \ .
\end{equation}
This is the maximal size of a bad cycle. First, build $C_0$ so that $C_0\fromto
\left( B \right)^A_r$ as colored ordered graphs and all distances in $C_0$ appear in $B$.

We now execute the construction of the previous proof $n-2$ times\footnote{One might
expect this to take $n$ times, but it only takes $n-2$ because already $C_0$ won't have
bad $2$-cycles.} to obtain $C_0$, $C_1$ ,
$\cdots$, $C_{n-2}$ with each new structure acting as the base, i.e. we have homomorphisms
$\pi_i: C_i \fromto C_{i-1}$. Now we claim by induction that $C_i$ doesn't have bad cycles
of size $i+2$.
Let us prove this for $i$. $C_i$ is built as an increasing union of pictures $P_0 , \cdots
, P_d$. $P_0$ is a disjoint union of copies of $B$ since it has no bad cycle at all.
We now build $P_i$ assuming we have built $P_{i-1}$. 
Note that the homomorphic image of a bad cycle at least contains a bad
cycle. So in $\pi^{-1}$ of a copy of $A$ in $C_{i-1}$ there is no bad cycle.
This property of having a homomorphism is preserved, so in particular applying the partite lemma
does not create bad cycles. 
But now the free amalgamation of things without bad cycles
might have a bad cycle. For example, if we amalgamate over the red points:
\begin{equation}
\begin{tikzcd}
&
\point
\arrow[dash]{ddl}
\\
\textcolor{red}{\point}
\arrow[dash]{ur}{1}\arrow[dash]{dr}{1}&\\
\textcolor{red}{\point}
\arrow[dash]{r}{5}&
\point
\end{tikzcd}
\end{equation}
we get a bad cycle. But as it turns out this can only happen if we already have a bad
triangle. I.e. the general
claim is that every projection of a bad cycle $K$ contains a bad cycle of smaller size.
This is because of the following. Assume we create a bad cycle by amalgamating two copies
of $P_{i-1}$. Any bad cycle has to have points in both copies by the induction hypothesis.
If the projection $\pi$ is not injective on $K$ we get a smaller bad cycle in the base
$C_{i-1}$, so we can assume the projection to the base is injective.
Now notice that $K$ must have at least two vertices which both project to $A_i$ with no
distance defined. But since $A_i$ is complete, we must add a distance when we project.
But this means we have broken the bad cycle into two pieces, one of which must therefore
be a smaller bad cycle.

This means $C_{n-2}$ has no bad cycle at all, and now we complete this arbitrarily to an
ordered metric space, and we are done.
\end{proof}

\section{Locally finite structures}

We now state a general definition which captures the features of metric spaces which we
took advantage of in the previous proof.

Recall a homomorphism is a map which sends relations to relations, but it might not be
injective, and might miss some relations.
As usual, an embedding is injective and preserves relations in both directions, 
i.e. an isomorphism with its image.

\begin{defn}
A \emph{homomorphism-embedding} $f: A\fromto B$ is a homomorphism whose restriction to any
irreducible substructure of $A$ is an embedding.
\end{defn}

\begin{defn}
A \emph{completion} $B$ of $A$ is an irreducible structure $B$ with a
homomorphism-embedding $f:A\fromto B$.
A \emph{strong completion} is a completion such that the homomorphism-embedding is
injective.
\end{defn}

\begin{defn}
Let $\cR$ be a class of finite irreducible structures. Then $\cK\subeq \cR$ is a \emph{locally
finite} subclass if for every $C_0\in \cR$ there is some integer\footnote{This is supposed
to be the maximal size of a bad substructure.} $n = n\left( C_0 \right)$
such that any structure $C$ has a strong $\cK$-completion provided that:
\begin{itemize}
\item $C_0$ is a completion of $C$
\item every substructure of size at most $n$ has a strong $\cK$-completion.
\end{itemize}
\end{defn}

\begin{thm}
Let $\cR$ be a Ramsey class of irreducible finite $\cL$-structures and let $\cK$ be a
hereditary locally finite subclass of $\cR$ with strong\footnote{This just means disjoint.} amalgamation.
Then $\cK$ is homogeneous.
\end{thm}

\end{document}

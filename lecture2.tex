\documentclass{amsart}
\usepackage{master}
\begin{document}
\title{Lecture 2\\Math 229}
\author{Lectures by: Professor Pierre Simon\\Notes by: Jackson Van Dyke}
\date{January 24, 2019}
\maketitle

\section{\texorpdfstring{$\om$}{Omega}-categorical structures}

\begin{defn}
A countable\footnote{All of our structures will be countable in this class.} 
structure $\cM$ is said to be \emph{$\om$-categorical}
if every (countable) structure $\cN$ which is elementarily equivalent to $\cM$ 
is actually isomorphic to $\cM$.
\end{defn}

\begin{thm}[Ryll-Nardzewski]
Let $M$ be a countable structure, then TFAE:
\begin{enumerate}[label=(\iii)]
\item $M$ is $\om$-categorical.
\item For every $n < \om$ there are only finitely many
$0$-definable subsets of $M^n$.
\item For every $n < \om$ there are only finitely many $n$-types over $\emp$
realized in $M$.
\item For every $n < \om$, the group $\Aut\left( M \right)$ has only finitely many orbits on each $M^n$.
In this case we say $\Aut\left( M \right)$ is \emph{oligomorphic}, or the action is oligomorphic.
\end{enumerate}
\end{thm}

\begin{proof}
The difficult part is relating the first to the rest.
This uses the theory of omitting types, so we won't prove this part.

$\left( ii \right)\iff \left( iii \right):$ If there are only finitely many $0$-definable subsets of $M^n$, 
then $n$-types are just ultrafilters in the boolean algebra of definable sets, 
so the result is immediate.
Conversely, if there are only finitely many $n$-types over $\emp$
realized in $M$, then a definable set is completely characterized by the
types it contains so there must be finitely many $0$-definable subsets
of $M^n$ as well.

$\left( iii \right)\iff \left( iv \right)$: 
For any $M$, if $\oline{a},\oline{b}\in M^n$ are in the same orbit under 
$\Aut\left( M \right)$ they have the same type.
The converse is false in general, but true if $M$ is homogeneous,
(it is sometimes even defined in this way)
and also true if $M$ is $\om$-categorical.
\end{proof}

\begin{exr}
Show that if $M$ is $\om$-categorical, 
then $a$ is in the algebraic closure of $A$ iff $a$ has a finite orbit under
\begin{equation}
\Aut\left( M / A \right) = 
\left\{ \sigma \in \Aut\left( M \right) \st \restr{\sigma}{A} = \id_A \right\} \ .
\end{equation}
Show that:
\begin{equation}
D\subeq M^n
\text{ is definable iff }
D \text{ is preserved set-wise
by all } \sigma\in \Aut\left( M \right) \ . 
\label{eqn:star}
\end{equation}
Finally show that property \eqref{eqn:star} is equivalent to being $\om$-categorical.
\end{exr}

\begin{rmk}
Note that it is always true that if a subset $D$ is definable then it is preserved. 
Often times in a first course on model theory a way to prove that a subset is definable
is to find an automorphism which doesn't preserve it.
The content of the exercise is then to show that the converse is true iff $M$ is $\om$-categorical.
\end{rmk}

The point here is that if we're working with an $\om$-categorical 
structure, we can forget about model theory and just think of this as a set
with a group acting on it, and then model theoretic properties can just be expressed 
as properties of the automorphism group.

\section{Automorphism group}

\begin{defn}
A \emph{permutation group} is a group $G$
equipped with a faithful action $G\acts X$ on some set $X$. 
I.e. there is an embedding $G\inj \Sym\left( X \right)$.
\end{defn}

\begin{exm}
$\Aut\left( M \right)\acts M$ is a permutation group.
\end{exm}

In general, $\Aut\left( M \right)$ carries little information about $M$
as in the following example:

\begin{exm}
$\left( \NN , \leq \right)$ and $\left( \NN , + , \times \right)$ both
have $\Aut\left( M \right) = \left\{ \id \right\}$ yet they are totally different structures.
\end{exm}

However, if $M$ is $\om$-categorical one can almost recover the structure on 
$M$ from $\Aut\left( M \right)$ as a permutation group. 
This works as follows.
In general, if $G\acts X$ is a permutation group, 
we can define a canonical structure on $X$, by adding an $n$-ary relation symbol for each $G$-orbit 
on $X^n$.
Let $X^G$ be this structure, then $G\leq \Aut\left( X^G \right)$ by construction.
In fact, this is almost equality.
Clearly it is not in general: if the group is trivial, 
this would mean there is a new symbol for every element, so this can not be equality.

To understand the sense in which is this is almost equality, we need a topology on $\Sym\left( X \right)$.
This is induced by the product topology on $X^X$.
More explicitly, a basic open set is given by finitely many point and their images:
\begin{equation}
\left\{ \sigma \in \Sym\left( X \right) \st\exists \oline{a} , \oline{b}\in X^n \suchthat
\sigma\left( \oline{a} \right)= \oline{b}  \right\} \ .
\end{equation}

\begin{prop}
$\Aut\left( X^G \right)$ is the closure of $G$ with respect to this topology.
\end{prop}

\begin{proof}
$\subeq$:
Let $\sigma\in \Aut\left( X^G \right)$, and $\oline{a} , \oline{b}\in X^n$ such that
$\sigma\left( \oline{a} \right) = \oline{b}$.
Now we just have to find $g\in G$ such that $g\left( \oline{a} \right) = \oline{b}$. 
Since $\sigma$ is an automorphism, 
$\oline{a}$, $\oline{b}$ satisfy the same relation and are therefore in the same orbit. 

$\supeq$: 
This is essentially the same argument. 
\end{proof}

So what we have seen so far, is that given a group acting on a set we have 
this canonical structure given by the group, 
and then the automorphism group of this structure is the closure of the group itself.
Of course, in general, if we start with a structure, look at its automorphism group
and build the canonical structure, we don't have something that is remotely related 
to what we started with.
We saw this in the example before. 
If start with a structure with no automorphisms, then the canonical structure
will just have a name for every point and nothing else, so you're losing the whole structure.
But if $G$ acts oligomorphically, then you actually recover the structure, but the language 
has obviously changed. 
The notion that the two structures are somehow the same
is captured by the following definition.

\begin{defn}
Two structures $M , M'$ on the same
underlying set are \emph{interdefinable}
if they have the same $0$-definable sets.
\end{defn}

\begin{rmk}
For this course, and generally in many cases, two interdefinable structures
are effectively the same.
\end{rmk}

\begin{prop}
If $G\acts X$ is oligomorphic, then there is,
up to interdefinability, a unique structure on $X$ such that
$G$ is dense in the automorphism group.
\end{prop}

\begin{proof}
We just have to prove uniqueness.
More specifically, let $X_0$ be a structure on $X$ such that
$G$ is dense in $\Aut\left( X_0 \right)$. 
Then for each $n$, there are finitely many 
$\Aut\left( X_0 \right)$-orbits on $X^n$, hence finitely many
$n$-types in $X_0$, hence finitely many $0$-definable subsets of $X^n$
which coincide with those orbits.
Therefore $X_0$ is interdefinable with $X^G$.
\end{proof}

If $G$ does not act oligomorphically, the issue is that
there are, in general, invariant sets which are not definable.

Now we understand that if we look at the automorphism group as a permutation group,
this exactly amounts to knowing the structure up to interdefinability.
The next step, is to consider the automorphism group as a topological group.
So imagine we're given the group with its topology, but not its action.
Then we might wonder how much of $M$ we can recover.
In general, of course, nothing. But as it turns out
we can recover a great deal about the structure in the $\om$-categorical case.
In particular, we can recover it up to bi-interpretability,
a concept which we introduce now.

\begin{defn}
An interpretation over\footnote{If we want to allow parameters, we can think of it as adding constants
to the language, but we will not deal with parameters right now.}
$\emp$ of a structure $N$ in a structure $M$ is given by 
\begin{enumerate}
\item a $0$-definable subset $D\subeq M^n$ for some $n$, 
\item a $0$-definable equivalence relation $E$ on $D$,
\item and a bijection $f: D / E \fromto N$, 
\end{enumerate}
such that the preimage under $f\comp \pi$, where $\pi: D\fromto D /E$,
of a $0$-definable set of $N^k$ (for any $k$) is $0$-definable in $M$.
\end{defn}

\begin{rmk}
Note that this isn't asking for the converse, it might have many more $0$-definable sets, 
but at least it has these.
\end{rmk}

\begin{exm}
$\left( \QQ , + , \cdot \right)$ is interpretable in $\left( \NN , + , \cdot \right)$.
\end{exm}

\begin{defn}
Two interpretations $f$ and $g$ of $N$ in $M$
are \emph{homotopic} if the set $f = g$
(a subset of $D_f / E_f \times D_g / E_g$) is $0$-definable in $M$.
\end{defn} 

\begin{exm}
Let $M_0 = \left( \om , = \right)$ for $\om$ some infinite set,
and $M = \left( \om^2 , E_1 , E_2 \right)$ where $E_1$ is equality in the first coordinate, 
and $E_2$ is equality on the second coordinate.
The two natural interpretations of $M_0$ in $M$, given by quotienting
$M / E_1$ and $M / E_2$, are not homotopic as there is no definable bijection between
$M / E_1$ and $M / E_1$.
If however, we had take $M_0^2$, so remembering not only the equivalence relation
of coordinates, but also we can somehow match between coordinates, 
then they would be homotopic.
\end{exm}

\begin{defn}
A bi-interpretation between $M$ and $N$
is given by two interpretations $f:M\fromto N$ and $g:N\to M$
such that $f\comp g$ is homotopic to $\id_M$, 
and $g\comp f$ is homotopic to $\id_N$.
\end{defn}

\begin{exm}
We saw that $M_0$ was interpretable in $M$ in two completely different ways, 
and of course $M$ is interpretable in $M_0$, but these two structures 
are not bi-interpretable. 
We won't prove this, but we will at least see that if we try to do it neively, it doesn't work.
If we compose $f:M / E_1 \fromto M_0$
and $g: M_0^2 \fromto M$ then we get 
$g\comp f : \left( M/ E_1 \right)^2 \fromto M$
which is not definable in $M$.
\end{exm}

\begin{rmk}
There exists a bi-interpretation between $M$ and $N$ 
iff $M^{\text{eq}}$ and $N^{\text{eq}}$ are interdefinable.
\end{rmk}

For almost anything we care about, two structures being bi-interpretable will mean 
they are effectively the same, though a bit `less the same' than two interdefinable
structures.

\begin{thm}
Let $M$ and $N$ be $\om$-categorical. 
Then $\Aut\left( M \right)$ and $\Aut\left( N \right)$
are homeomorphic iff $M$ and $N$ are bi-interpretable.
\end{thm}

\begin{Proof}
$\left( \converse \right)$:
Let $M$ and $N$ be bi-interpretable.
Note that an interpretation $f$ of $N$ in $M$ gives a map 
$\Aut\left( f \right): \Aut\left( M \right) \fromto \Aut\left( N \right)$
which is continuous.
To see that this is continuous, we need to show that the preimage of a basic open set is open.
If we have an automorphism with an image which sends a finite tuple
to a finite tuple, then there is a finite part of that automorphism
which already does this. But this is true because every finite tuple of elements of $N$ is represented
as a finite tuple of elements of $M$, so we just take any preimage,
and the image of the automorphism in that preimage is enough.
This is also functorial, i.e. if $f:M\fromto N$ and 
$g:N\fromto P$ are interpretations, then 
\begin{equation}
\Aut\left( g\comp f \right) = \Aut\left( g \right) \comp \Aut\left( f \right) \ .
\end{equation}
This is essentially immediate from the definitions.

Two interpretations $f$ and $g$ are homotopic iff $\Aut\left( f \right) = \Aut\left( g \right)$.
We just prove the forward implication.
The idea is that the set $f=g$ is a $0$-definable set, 
so it is preserved by the automorphisms.
For simplicity, let $f: M \lfromto{\sim}N$
and $g: N\lfromto{\sim} M$. If we take $\sigma\in \Aut\left( M\right)$
and $a\in N$, the pair $\left( f^{-1}\left( a \right) , g^{-1}\left( a \right) \right)$
is in the set $f= g$, so
$\left( \sigma f^{-1}\left( a \right) , \sigma g^{-1} \left( a \right) \right)$ is in $f= g$
which by the definition of the set $f=g$
implies $f\sigma f^{-1}\left( a \right) = g\sigma g^{-1}\left( a \right)$. 

If $f$ and $g$ give a bi-interpretation then this implies 
$\Aut\left( f \right) \comp \Aut\left( g \right) = \id_{\Aut\left( N \right)}$
and $\Aut\left( g \right) \comp \Aut\left( f \right) = \id_{\Aut\left( M \right)}$
and these are continuous, so homeomorphic.

$\left( \implies \right)$:
\begin{clm}
If $\phi:\Aut\left( M \right) \fromto\Aut\left( N \right)$ is continuous and
$\im \phi$ has finitely many orbits on $N$, then $\phi$ is of the form $\Aut\left( f \right)$
for $f$ an interpretation of $N$ in $M$.
\end{clm}
\begin{proof}
Choose representatives $b_1 , \cdots , b_k$ of the orbits of the image in $N$.
Then we can find $a_1, \cdots , a_m\in M$ such that 
$\Aut\left( M / \left\{ a_1 , \cdots , a_m \right\} \right)$ is mapped into
$\Aut\left( N  / \left\{ b_1 , \cdots , b_k \right\}\right)$. 
(Note the $a_i$ exist by continuity.)
WLOG assume $m \leq k$. Let $D\subeq M^{n+1}$ consist of the conjugates of the $k$ elements 
$\left( a_1 , a_1 , \cdots , a_m \right)$, 
$\left( a_2 , a_1 , \cdots , a_m \right)$, $\cdots$ ,
$\left( a_k , a_1 , \cdots , a_m \right)$.
Note that this is a $0$-definable set.
Now we can define $f:D\fromto N$ by
\begin{equation}
f\left( \sigma\left( a_i \right) , \sigma\left( a_1 \right) , \cdots , \sigma\left( a_m \right) \right) = 
\phi\left( \sigma \right) b_i
\end{equation}
for $i\in \left\{ 1 , \cdots , k \right\}$ and $\sigma \in \Aut\left( M \right)$.
First we need to check that this is well-defined.
Assume there are two representatives of the same element, 
then first the $i$ must be the same, and second this means there are two automorphisms $\sigma$
and $\tau$ with the same image in the $a_1 , \cdots , a_m$, 
but since $\Aut\left( M / \left\{ a_1 , \cdots , a_m \right\} \right) \fromto 
\Aut\left( N / \left\{ b_1 , \cdots , b_k \right\} \right)$
this means that $\sigma$ and $\tau$ would have to have the same image on the $b_1 , \cdots , b_k$. 
Now we have to check that two things don't get sent to the same thing,
but the equality is a $0$-definable set.
Also, any $0$-definable set in $N$ is mapped to a $0$-definable set in $M$.
Since $M$ and $N$ are both $\om$-categorical, 
this amounts to saying that invariant sets map to invariant sets.
This is an interpretation of $N$ in $M$,
and by construction, $\Aut\left( f \right)= \phi$. 
We hit the entirety of $M$, because the $b_i$s are representative of all of the orbits.
\end{proof}
\begin{center}
\textbf{To be continued\ldots}
\end{center}
\end{Proof}

\end{document}

\documentclass{amsart}
\usepackage{master}
\begin{document}
\title{Lecture 16\\Math 229}
\author{Lecture: Professor Pierre Simon\\Notes: Jackson Van Dyke}
\date{March 14, 2019}
\maketitle

\section{Descriptive set theory}

We will continue with our crash course in descriptive set theory. 

\subsection{Polish spaces}

\begin{cor}
Polish spaces are precisely the $G_{\dd}$ subspaces of $W$.
\end{cor}

Then we have an analogous result:

\begin{prop}
\begin{itemize}
\item Every perfect\footnote{Recall this means it has no isolated points.} Polish 
space has a subspace homeomorphic
to the cantor set $C$.
\item If $X$ is a compact polish space then there is a co-continuous surjective map
$C\fromto X$.
The idea is that the cantor set is universal among compact polish spaces. 
\end{itemize}
\end{prop}

\begin{proof}[Sketch proof]
Let $X$ be perfect Polish. Pick two points, 
pick two disjoint open sets containing them.
These are not isolated so we can break these into two again. Now we can continue this process with
decreasing radius of the balls going to $0$.
Then we get a homeomorphic embedding.

In the second part, since $X$ is compact we can cover it by finitely many balls possibly
not disjoint. And then we're naturally constructing a surjective map from the finite tree
to $X$, and then we check that this is just homeomorphic to $C$.
\end{proof}

A compact space $X$ is polish iff metrizable iff second countable.
Note this is not equivalent to separable. 

\begin{exm}[Counter-example]
$\left[ 0,1 \right]^{\left[ 0,1 \right]}$ is compact, separable, and not second countable.
\end{exm}

Let $X$ be Polish. Define $K\left( X \right)$ to consist of the compact subspaces of $X$
with the Vietoros topology. The base is given by
\begin{equation}
\left\{ K\subeq X \st \exists U_0 , \cdots , U_n\subeq X\suchthat
K\subeq U_0, K\cap U_1\neq \emp, \cdots , K\cap U_n \neq \emp \right\} \ .
\end{equation}
This is metrizable. In particular this is given by the Hausdorff metric. 
Define 
\begin{equation}
d_H\left( K , L \right) = 
\begin{cases}
0 & K = L = \emp\\
1 & K =  \emp \neq L \text{or} L = \emp \neq K\\
\max\left\{ \dd\left( K , L \right) , \dd\left( L , K \right) \right\} & K,L\neq \emp
\end{cases}
\end{equation}
where
\begin{equation}
\dd\left( K , L \right)\ceqq \sup_{x\in K} d\left( x,L \right) \ .
\end{equation}
Then it is a fact that this metric gives the above topology. 
In fact, if $D\subeq X$ is dense, 
\begin{equation}
K_f\left( D \right) = \left\{ K\subeq D,K\text{ finite} \right\}
\end{equation}
is dense in $K\left( X \right)$, so $K\left( X \right)$ is Polish.

\subsection{Baire category}

Let $X$ be a topological space.

\begin{defn}
A set $A\subeq X$ is \emph{nowhere dense} if $\clos{A}$ has empty interior. 
\begin{enumerate}
\item A set $A\subeq X$ is \emph{meager} (a first category)
if $A = \un A_n$ where $A_n$ is nowhere dense
(equivalently, $A\subeq \un F_n$, $F_n\subeq X$ closed with empty interior).
\item $A\subeq X$ is \emph{comeager} if $X\minus  A$ is meager\footnote{This is also
called residual.}
(equivalently $A$ contains a countable intersection of dense opens).
\end{enumerate}
\end{defn}

\begin{defn}
The space $X$ is \emph{Baire} if any of the following equivalent conditions are satisfied:
\begin{enumerate}[label = (\iii)]
\item Every nonempty open set is non-meager.
\item A meager set has empty interior. 
\item Every comeager set is dense.
\item Intersection of countably many dense open sets is dense.
\end{enumerate}
\end{defn}

\begin{thm}
Every completely metrizable space is Baire.
Every locally compact Hausdorff space is Baire.
\end{thm}

\begin{proof}
Consider a countable intersection of dense open sets
\begin{equation}
\binter_{n<\om} U_n\ .
\end{equation}
Take a ball in $U_1$ of radius $<1$. 
Then take a ball of radius $<1/2$ in $U_2$, and continue in this fashion.
Then what we get is a point in the intersection of the open sets.
\end{proof}

In a Baire space, meager sets form a $\sigma$-ideal. This means the collection of these is
closed under taking subsets, countable unions, and doesn't contain the whole
space.\footnote{Recall this is different from a $\sigma$-algebra because, first of all, a
$\sigma$-algebra must contain the whole space, and it is closed under
countable unions, but also complements, which implies it is closed under countable
intersections as well.}
The idea is that this gives a notion of smallness. Meager sets somehow have measure $0$, 
non-meager somehow have positive measure, and co-meager sets have measure $1$.

We will say that a property holds \emph{generically} if it holds on a comeager set.
We write $\forall^* x P\left( x \right)$ to mean that $P$ holds generically, 
and $\exists^* x P\left( x \right)$ means $\left\{ x\st P\left( x \right) \right\}$ is
non-meager.

\begin{defn}
A set $A\subeq X$ has the Baire property (BP) if $A \symd U$ is meager for some open $U$.
The notation is that $A=^* U$.
\end{defn}

\begin{prop}
The class of subsets of $X$ which have BP is a $\sigma$-algebra.
\end{prop}

\begin{proof}
Recall this means it is closed under complements, countable unions, and countable
intersections. Note that for open $U$ we have $U =^* \clos{U}$ and for closed $F$ we have
$F=^* \interior\left( F \right)$. 
If $A$ has BP, then $A=^* U$ for some $U$, and then
\begin{equation}
X\minus A =^* X\minus U =^* \interior\left( X\minus U \right) \ .
\end{equation}
If $A_n =^* U_n$ then
\begin{equation}
\bun_n A_n =^* \bun_n U_n
\end{equation}
so we are done.
\end{proof}

Notice that every Borel set has BP
(recall the Borel sets comprise
the smallest $\sigma$-algebra containing all open sets).

\begin{lem}
If $A$ has BP then we can write
\begin{equation}
G\subeq A \subeq F
\end{equation}
such that $G$ is a $G_\dd$ set, and $F$ is an $F_\sigma$ set such that $F\minus G$ is
meager.
\end{lem}

\begin{proof}
$A = ^* U$ for some open $U$, which means $A\symd U\subeq F$, where $F$ is closed and meager.
So $G = U\minus F$ is $G_\dd$, $G\subeq A$, and $A\minus G\subeq F$ is meager. 
Now we can apply this to the complement $X\minus A$ to get the $F_\sigma$ set.
\end{proof}

\begin{defn}
A function $f:X\fromto Y$ is  Baire-measurable if the preimage of any open set has BP.
\end{defn}

Now we have some sort of Fubini property.

\begin{thm}[Kuratowski-Ulam]
Let $X$ and $Y$ be second countable topological spaces. If $A\subeq X\times Y$
then
\begin{enumerate}[label=(\iii)]
\item $\forall^* x$ $A_x = \left\{ y : A\left( x,y \right) \right\}$ has BP.
\item $A$ is meager iff $\forall^* x, A_x$ is meager.
\item $A$ is comeager iff $\forall^* x A_x$ is comeager.
\end{enumerate}
\end{thm}

We said this is some sort of Fubini property because this can be rewritten as
\begin{equation}
\forall\left( x,y \right)A\left( x,y \right)\iff
\forall^* x \forall^* y A\left( x,y \right) \iff
\forall^* y\forall^* x A\left( x,y \right) \ .
\end{equation}

\begin{thm}
If $X$ is perfect Polish and $E\subeq X^2$ is a meager equivalence relation
then 
\begin{equation}
\abs{X/E} = 2^{\aleph_0}
\end{equation}
and in fact there is a cantor set $C\subeq X$ which is transversal.
\label{thm:previous}
\end{thm}

\subsection{Analytic sets}

\begin{defn}
If $X$ is Polish, a subset $A\subeq X$ is \emph{analytic} if there is $Y$ Polish and
$f:Y\fromto X$ continuous with $f\left( Y \right) = A$. 

We say $A\subeq X$ is \emph{co-analytic} if $X\minus A$ is analytic.
\end{defn}

\begin{thm}
If $X$ is Polish, then
\begin{itemize}
\item Every analytic (and hence co-analytic) set has BP.
\item A set $A\subeq X$ is Borel iff it is both analytic and co-analytic.
\end{itemize}
\end{thm}

\section{Polish groups}
% section 13

Recall the following.
A topological group is a group $G$ equipped with a topology so that the map $G^2 \fromto
G$ which sends $\left( x,y \right)\mapsto xy^{-1}$ is continuous. 
If $G$ is a topological group and $H\subeq G$ is an open subgroup, then $H$ is closed.

\begin{defn}
A \emph{Polish group} is a topological group whose topology is Polish.
\end{defn}

\begin{thm}[Birkhoff-Kakutani]
Let $G$ be a topological group. Then $G$ is metrizable iff
there is a countable basis of a neighborhood of $1$. 
In this case there is a left-invariant compatible metric.
\end{thm}

To be clear, left-invariant means that $d\left( x,y \right) = d\left( gx, gy \right)$.

\begin{rmk}
It is important that we define a Polish group as a topological space rather than a metric
space.
Since it is polish there is always a complete metric, and then by the theorem there is always this
second left-invariant metric. But in general there is no complete invariant metric. 
If it is locally compact this does exist (like for $\RR$).
\end{rmk}

\begin{prop}
Let $G$ be a Polish group and let $H\leq G$. Then $H$ is Polish iff $H$ is $G_\dd$. 
This is also equivalent to $H$ being closed.
In this case $G/H$ is a Polish space, and in particular a Polish group if $H\nsubeq G$.
\end{prop}

\begin{proof}
We prove only the second $\left( \iff \right)$. 
Let $H\leq G$ be $G_\dd$ and assume $G = \clos{H}$. 
Then $H$ is a dense $G_\dd$ set, and so is every coset of $H$. 
However we cannot have two disjoint comeager sets which means
there is only one coset so $H  = G$.
\end{proof}

\begin{thm}[Pettis]
Let $G$ be a topological group with non-meager
$A\subeq G$ having BP.
Then $A^{-1}A$ contains a neighborhood of $1$.
\end{thm}

\begin{proof}
We know that for some open $U$ we have $A =^* U$. Then we have $g\in G$
and $1\in V$ for $V$ open such that
$gVV^{-1} \subeq U$. Therefore $gV \subeq U\cap Uh$ for $h\in V$.
Then we claim $V\subeq A^{-1} A$.
Let $h\in V$. Then we have
\begin{equation}
\left( U\cap Uh \right)\symd \left( A\cap Ah \right) \subeq \left( A\symd A \right)\un
\left( \left( U\symd A \right)h \right)
\end{equation}
which means
\begin{equation}
A\cap Ah =^* U\cap Uh
\end{equation}
and now $U\cap Uh$ contains a neighborhood of the identity, so it is nonempty, and even
has nonempty interior so $A\cap Ah\neq \emp$ as well.

Let $x\in A\cap Ah$. So $x\in A$ and $x= yh$ for some $y\in A$. Then $y^{-1} x= h\in
A^{-1} A$. 
\end{proof}

\end{document}

\documentclass{amsart}
\usepackage{master}
\usepackage{overlay}
\begin{document}
\title{Lecture 14\\Math 229}
\date{Thursday March 7, 2019}
\author{Lecture: Professor Pierre Simon\\Notes: Jackson Van Dyke}
\maketitle

\appendix

\section{Hrushovski construction}

The goal is to construct an $\om$-categorical $2$-sparse graph of infinite degree. The
main starting point is to generalize the notion of \Fraisse structures/limits.
Recall for this we had a class
of finitely generated structures and considered all embeddings between them and
amalgamated. The generic strategy is to restrict out attention to only certain embeddings.

Consider a class $K$ of (finite/finitely generated) structures along with a set $S$ of
embeddings between elements of $K$ such that $S$
\begin{enumerate}
\item contains all isomorphisms, 
\item is closed under composition, and
\item whenever $A, C\in K$, and $f: A\inj C$ is in $S$, if $K\ni B\subeq C$ contains the image
of $f$, then $g:A\inj B$ such that $g\left( a \right) = f\left( a \right)$ is in $S$.
\end{enumerate}
We call elements of $S$ strong embeddings. 
We write $A\leq B$ to mean that $A$ is a strong substructure, i.e. $A\subeq B$ and the
inclusion map is in $S$. Note that if $f : A\inj B$ is an embedding, then $f\in S$ iff $f\left( A
\right)\leq B$.

\begin{defn}
$\left( K , \leq \right)$ satisfies the JEP if for all $A,B\in K$ there exists $C\in K$
such that 
\begin{equation}
A\inj C \invinj B 
\end{equation}
where the maps are in $S$. 
We say $\left( K , \leq \right)$ has AP if the usual definition holds where we insist the
maps are strong.
\end{defn}

Now assume $K$ has JEP and AP. Then we can consider the analogue of the \Fraisse limit,
which is sometimes called the Hrushovski-\Fraisse limit. 

\begin{defn}
A chain $A_0\leq A_1 \leq \cdots$ of elements of $K$ is \emph{rich} if 
\begin{enumerate}
\item For all $A\in K$ there is some $i$ such that $A\leq A_i$,
\item for all strong $f:A_i \fromto B$ there is $j$ and strong $g:B\fromto A_j$ such that
it commutes, i.e. $g\left( f\left( a \right) \right) = a$ for $a\in A_i$.
\end{enumerate}
\end{defn}

\begin{thm}
Suppose $\left( K , \leq \right)$ is countable and has JEP and AP. Then, up to isomorphism, there is
a unique countable structure $M$ which can be written as the union of a rich sequence.
Moreover $M$ is ``homogeneous'' in the following sense. If $A\leq M$ and $B\leq M$ are
isomorphic, then the isomorphism extends to an automorphism
of $M$.
\end{thm}

Note that $A\leq M$ means that there is some $i$ such that $A\leq A_i$, and this doesn't
depend on the choice of the rich sequence.
Let's see that it does not depend on the sequence. So let $M= \un A_i = \un B_i$ and
$A\leq A_i$. Say $A_i \subeq B_j \subeq A_k$. We know $A_i \leq A_k$, so $A_i \leq B_j$ as
well, and therefore $A\leq A_i \leq B_j$. But strong embeddings are closed under
composition, so $A\leq B_j$. 

\begin{exm}
Consider the class of graphs with degree $\leq 1$. 
So all we can have is matching, and isolated points. 
Then we can say that $A\leq B$ if there is no edge between $A$ and the complement.
Then the limit of this class consists of infinitely many matchings, and infinitely many
isolated points.
\label{exm:silly}
\end{exm}

\begin{rmk}
Assume $K$ has finitely many substructures of each size $n$ (and no infinite structure)
and there is $F: \NN \fromto \NN$ such that if $B\in K$, $A\subeq B$, and $\abs{A}\leq n$,
then there is $A\subeq C\leq B$ with $\abs{C}\leq F\left( n \right)$.
Then $M$ is $\om$-categorical.
\end{rmk}

If we take the smallest $C$, we should think of $C$ as the closure of $A$.
So somehow $M$ is homogeneous on strongly embedded sets. 
To know the orbit of a finite set in $M$, it is enough to know the automorphism type of
something which contains it and is a strong embedding in $M$.

\subsection{Hrushovski pre-dimension}

Fix some $k\geq 2$. This will be the sparsity parameter.
Let $C_0$ be the class of $k$-sparse graphs:
\begin{equation}
\abs{E\left( A \right)}\leq k\abs{V\left( A \right)}
\end{equation}
for any finite subset $A$. Recall $E\left( A \right)$ consists of the edges with both
endpoints in $A$.

Note first that this class does not have amalgamation. 

\begin{exm}[Counterexample]
Take $k = 2$ and consider the black graph. Then consider two graphs built by adding one
vertex and three edges, written in blue and red respectively:
\begin{equation}
\begin{cd}
&&& \textcolor{red}{\something}\arrow[dash,red]{dl}\arrow[dash, bend left, red]{ddl}
\arrow[dash,bend right,red]{dll}
\\
& \something \arrow[dash]{r}\arrow[dash]{d}\arrow[dash]{dr}
& \something\arrow[dash]{d}\arrow[dash]{dl}
& \\
\something\arrow[dash]{ur}\arrow[dash]{r}
& \something\arrow[dash]{r}
& \something
&\\
&&& \textcolor{blue}{\something}\arrow[dash,blue]{ul}\arrow[dash,blue,bend left]{ull}
\arrow[dash,blue,bend right]{uul}
\end{cd}
\end{equation}
We can't amalgamate the blue and the red graphs.
The point is that if we add $k$ edges when we add a single vertex
this is somehow the saturated case and adding any more breaks amalgamation.
We now make this formal.
\end{exm}

Define the predimension $\dd$ on $C_0$ to be
\begin{equation}
\dd\left( A \right) = k\abs{V\left( A \right)} - \abs{E\left( A \right)} \ .
\end{equation}
Note that our class is precisely the class such that $\dd\left( A \right) \geq 0$. 

Now we specify strong embeddings.

\begin{defn}
For $A, B \in C_0$ and $A\subeq B$ we say that $A\leq B$ if for all $A\subeq C\subeq B$, 
\begin{equation}
\dd\left( C \right) \geq \dd\left( A \right) \ .
\end{equation}
\end{defn}

\begin{exm}
For $k = 2$, there is no vertex in $B\minus A$
which has three neighbors in $A$.
\end{exm}

\begin{lem}
$\left( C_0 , \leq \right)$ is a (generalized) \Fraisse class with JEP and AP.
\end{lem}

Note that amalgamation is obtained by free amalgamation
\begin{equation}
\begin{cd}
& B \arrow[hook]{dr} & \\
A\arrow[hook]{ur}\arrow[hook]{dr}&&
D = B\dun_A C \\
& C\arrow[hook]{ur}
\end{cd} \ .
\end{equation}
The argument that this works is straightforward.

Note that the \Fraisse limit of this is not quite what we want because it is not
$\om$-categorical. There is no finite closure.

\begin{exm}
For $k = 1$ we can have paths, but the limit will be tree-like which is not
$\om$-categorical. A similar story holds for $k = 2$.
Note that a strong substructure of this is a path, and closures are unbounded.
\end{exm}

\subsection{$d$-closed}

We define the dimension to be the minimal predimension of something which contains $A$.

\begin{defn}
Let $A\subeq B$ in $C_0$. Then we write $A\leq_d B$, and say $A$ is $d$-closed in $B$ if
$\dd\left( A \right)\lneq \dd\left( C \right)$ for any $A\lneq C \subeq B$.
\end{defn}

\begin{lem}
$\left( C_0 , \leq_d \right)$ is a (generalized) \Fraisse class with JEP and AP.
\end{lem}

\begin{exm}
For $k = 1$ the \Fraisse limit is an arbitrary disjoint union of finite trees. So we have
all sorts of finite connected components because if we add any vertex the dimension
doesn't increase.
\end{exm}

Since our connected components are finite we can put a bound on them.

\begin{defn}
Let $F:\RR^{\geq 0} \fromto\RR^{\geq 0}$ and define
\begin{equation}
C_F = \left\{ A\in C_0 \st \forall B\subeq A, \dd\left( B \right) \geq  F\left( \abs{B} \right) 
\right\}
\end{equation}
\end{defn}

Now we will be interested in the case when $F$ looks like a logarithm.
So for a subset $B$ this is saying our predimension can't be too small. 
In the case of trees, any finite tree has predimension $1$, so as soon as this function is
greater than $1$ there is no finite tree of this size, and we get a bound on the size.

\begin{thm}
Let $F$ be smooth, increasing, $F\left( x \right) \fromto \infty$, and $F\left( 0
\right)=0$. Then if
\begin{enumerate}
\item $F'$ is non-increasing, and
\item $F'\left( x \right)\leq 1/x$ for $x>0$
\end{enumerate}
then $\left( C_F , \leq _d \right)$ is a free amalgamation class and the limit is
$\om$-categorical.
\end{thm}

\begin{proof}[Proof sketch]
So the points allowed are above the curve.
So say we have a strong embedding of $A$ into $B_1$ and $B_2$. 
We know the size and codimension of $B_1$ and $B_2$ are at least as large as $A$, so
we can plot the predimension as a function of the size as in \cref{fig:prediagonal}.
Then the free amalgamation $C$ forms the final
vertex of the parallelogram in the plot as in \cref{fig:prediagonal}.
\begin{figure}
\begin{overlay}
\pict{prediagonal.pdf}{0.8}
\toptext{$\something$}{-2,1}
\toptext{$\something$}{-1,1.5}
\toptext{$\something$}{-1.5,2}
\toptext{$\something$}{-0.5,2.5}
\toptext{$A$}{-2.3,1}
\toptext{$B_1$}{-0.6,1.5}
\toptext{$B_2$}{-1.9,2}
\toptext{$C$}{-0.1,2.5}
\toptext{$\abs{A}$}{4.7,-1.3}
\toptext{$\dd\left( A \right)$}{-4.8,2.2}
\end{overlay}
\caption{$A$ has a strong embedding into $B_1$ and $B_2$. The amalgamation forms the final
corner of a parallelogram.}
\label{fig:prediagonal}
\end{figure}
So now we just need a condition on $F$ such that the interior of such parallelograms is
disjoint from the portion under the curve.
As it turns out this gives us exactly the conditions in the theorem.

$\om$-categoricity follows from the fact that $F\fromto \infty$ because the predimension
of the closure can only go down, but $F$ eventually goes to infinity, so the closure has bounded size.
\end{proof}

\begin{exm}
Let $k = 2$ and take:
\begin{align}
F\left( 1 \right)  = 2 
&&
F\left( 2 \right) = 3
\\
F\left( 3 \right) = 4
&&
F\left( 4 \right) = 5
\\
F\left( 5 \right) = 5
&&
F\left( k \right)  = \log\left( k \right) + \left( 5- \log \left( 5 \right) \right)
\end{align}
for $k\geq 5$.
If we have two points with edges, the predimension is $4$, 
and if we have an edge the predimension is $3$ which is fine, but anything more and it is
not allowed. For three vertices, we allow:
\begin{equation}
\begin{cd}
& \something\arrow[dash]{dl}\\
\something\arrow[dash]{dr}&\\
& \something
\end{cd}
\end{equation}
but we don't allow a triangle. 
For $4$-vertices this won't allow a $4$-cycle.
Notice however that $5$-cycles are allowed.
Every vertex has infinite degree because of the following. 
If we have any graph and add a vertex with one
edge, this increases the predimension so this is always an allowed move.
So the theorem is already proved. 

Note the following other properties. 
This graph has a unique $1$-type, as the type of a point, $\left\{ a \right\}$, is always
closed. It also has a unique type of an edge as 
\begin{cd}
\something\arrow[dash]{r}&\something
\end{cd}
is closed.
It is clear that the limit is connected as well. Assume we have two points $a$ and $b$
which are not connected. Then the predimension is
\begin{equation}
\dd\left( \left\{ a,b \right\} \right) = 4\ .
\end{equation}
There are two cases. If $\left\{ a,b \right\}$ is strongly embedded, then 
we have
\begin{equation}
\begin{tikzcd}
a\arrow[dash]{d}&
b\arrow[dash]{d}\\
c\arrow[dash]{r}&
d
\end{tikzcd}
\end{equation}
and if not, then there is $c$ such that we have
\begin{equation}
\begin{tikzcd}
a\arrow[dash]{r}&
c\arrow[dash]{r}&b
\end{tikzcd} \ .
\end{equation}
\end{exm}

\end{document}

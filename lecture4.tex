\documentclass{amsart}
\usepackage{master}
\begin{document}
\title{Lecture 4\\Math 229}
\date{January 31, 2019}
\author{Lecture: Professor Pierre Simon\\Notes: Jackson Van Dyke}
\maketitle

The title of this lecture is 
\begin{center}
\textbf{a day at the zoo: examples.}
\end{center}

This is typically a subject where there are
very few general theorems, and 
one tests a technique by running through examples, 
so it is important to know them.

\section{Stable structures}

Consider the structure
$X_0 = \left( X , = \right)$ where $X$ is some countable set.
$\Aut\left( X_0 \right)$ is the full symmetric group
$\Sym\left( X \right)$.
One thing we can do is try to build finite covers of $X_0$. 
A finite cover\footnote{This same definition holds for any structure.}
is an $\om$-categorical structure $M$ with a projection
\begin{equation}
\begin{cd}
M\arrow{d}{\pi}\\ X_0 \simeq M / E
\end{cd}
\end{equation}
where $E$ is a definable equivalence relation on $M$ with finite bounded classes.
The fact that the quotient $M / E$ is exactly $X_0$ means 
there is no extra induced structure, i.e.
there are no extra $0$-definable sets.
Equivalently, the canonical map
$\Aut\left( \pi \right) : \Aut\left( M \right) \fromto \Aut\left( X_0 \right)$
is surjective.

\begin{exm}
A $2$-cover is a finite cover where the equivalence classes have size $2$.
An immediate example is given by the disjoint union $M = X_0 \times\left\{ 0
\right\}\un X_0 \times \left\{ 1 \right\}$ with the canonical projection.
The automorphism group $\Aut\left( M \right) = \Aut\left( X_0 \right) \times \ZZ / 2\ZZ$.
\end{exm}

\begin{exm}
Another $2$-cover of $X_0$ is given by putting $2$ points above every points of $X_0$.
So we have the equivalence relation mapping to $X_0$, but not the equivalence relation
giving us the two copies.
This is a reduct of the first one. So now the automorphism group is much bigger.
In particular it is the wreath product $\Aut\left( X_0 \right) \wr \ZZ/ 2\ZZ$.
\begin{exr}[*]
Prove these are the only two transitive $2$-covers.
\end{exr}
\end{exm}

\begin{exm}
Now we construct $4$-covers.
If we put $4$ points over every point of $X_0$, 
with the following arrows:
\begin{equation}
\begin{cd}
&\cdot \arrow{dr}&\\
\cdot \arrow{ur}&&\cdot \arrow{dl}\\
&\cdot \arrow{ul}
\end{cd}
\label{eqn:4points}
\end{equation}
so the fibers are independent and the automorphism group is
again the wreath product $\Aut\left( X_0 \right) \wr \ZZ / 4\ZZ$,
i.e. because of the arrows the only automorphisms of the fibers are rotations.

Now we could of course trivialize everything as with the $2$-covers 
and get a direct product, or we could try to do something in between. 
I.e. we can also add an equivalence relation which has $2$-classes 
that picks out $2$ opposite points.
In particular, we can take the top and bottom points in each fiber to
be equivalent to the top and bottom points in every other fiber.
Of course if we just look at each fiber, we just have this $\ZZ / 4\ZZ$ acting on it,
but now the fibers aren't completely independent.
If we rotate one fiber by $1$, it doesn't force us to turn the other fibers exactly by
$1$, but it does force us to either rotate it by $1$ or $3$.
So the automorphism group is something in between $\Aut\left( X_0 \right)\times \ZZ/4\ZZ$
and $\Aut\left( X_0 \right) \wr \ZZ / 4\ZZ$.

So in the $2$-cover case, we couldn't sort of mix the situations, we only had these two.
But now we have discovered a $4$-cover which is sort of in between the two extremes.
This turns out to somehow be as complicated as it gets.
\end{exm}

Now a much more fun and complicated situation is the following sort of example:

\begin{defn}
The $2$-Grassmannian of $X_0$ is the set of $2$-element subsets of $X_0$
with the induced structure.
\end{defn}

The idea is that each point is a $2$-element subset, and then 
there is an edge whenever two of these subsets intersect.

\begin{exm}
There is an interesting $4$-cover of this.
Above every point we can put $4$ points, as in \eqref{eqn:4points},
except now we have $1$ equivalence relation with classes
of size $2$, where each class is mapped to an elements in $X_0$.
The picture is like this:
\begin{equation}
\begin{cd}
&\textcolor{red}{\point} \arrow{dr}\arrow[dash,red]{dd}&\\
\point \arrow{ur}\arrow[dash]{rr}&
\arrow[bend left,red]{dddr}\arrow[bend right]{dddl}&\point \arrow{dl}\\
&\textcolor{red}{\point} \arrow{ul}\arrow[dotted,no head]{d}
\\
& \left\{ 1,2 \right\} & \\
1 && 2 
\end{cd}
\end{equation}
where the lines without arrows denote the $2$-classes, and the color of the arrow mapping
to $X_0$ indicates the corresponding class.
\begin{prop}
This structure is not interpretable in $X_0$.
\end{prop}
This is hard to prove, but we can at least see it isn't obviously interpretable. 
Say we introduce $4$ parameters $a$, $b$, $c$, and $d$ and then the points
in a fiber are labelled by the pair of element of $X_0$
and one of the new parameters. 
So maybe we identify $\left( \left( 1,2 \right) , a \right)$
with $\left( \left( 2,1 \right) , c\right)$
But then we don't know what to send to $1$, and what to send to $2$
because of the orientation of the arrows. 
But then we have to check that no other identification works either.
Note however that it is interpretable in DLO.
\end{exm}

\begin{exm}
Let $M$ be the set of $4$-element subsets of $X_0$ equipped with a partition into two
sets of size $2$.
This is interpretable in $X_0$.
Also, $M$ is $\om$-categorical, but cannot be made homogeneous in a finite relational
language. I.e.
it is not interdefinable with a structure which is homogeneous in a finite relational
language.

\begin{exr}
Show that this is true.
\end{exr}

\begin{rmk}
We know that a structure being $\om$-categorical means there are only finitely many $n$-types
for any $n$. So what is the extra property we get from being homogeneous in a finite
relational language which is not syntactic? As it turns out,
an $\om$-categorical structure $M$ is interdefinable with a structure homogeneous in a
finite relational language of maximal arity $k$ iff $n$-types are determined by 
their $k$-subtypes. In other words
\begin{equation}
\bors_{i_1 < \ldots < i_k < n} \type\left( a_{i_1} \cdots a_{i_k} \right)
\determines
\type\left( a_1 \cdots a_n \right) \ .
\end{equation}
\end{rmk} 

This is surprising, however if we built the same $M$ starting with DLO instead of $X_0$, 
then we get a finitely homogeneous structure. I.e. once we have an order, everything
trivializes.
This gives an example of a reduct of a finitely homogeneous structure which is not
finitely homogeneous.
\end{exm}

\begin{thm}[Lachlan]
Every fintiely homogeneous stable structure (in particular finite covers of $X_0$)
is interpretable in DLO.
\end{thm}

\begin{rmk}
Note that DLO is not stable.
\end{rmk}

\section{Orders}

Linear orders are easy to deal with. 
DLO is the unique linear order which is homogeneous in the language
consisting of just $\left\{ \leq \right\}$.
Partial orders become more complicated. 
We might ask the same question of which partial orders are homogeneous in $\left\{ \leq\right\}$ 
(the language with just a partial order symbol).
There is a classification of this due to Schmerl:
\begin{enumerate}
\item DLO
\item Fraiss\'e limit of all partial orders. (This exists and is unique as a consequence of partial
orders having amalgamation.)
\item $X_0$
\item Dense chain of antichains. This is where we put an anti-chain of size $\leq
\aleph_0$ above every point of DLO. 
\item Disjoint union of chain. So an anti-chain of chains of a fixed size $\leq \aleph_0$.
\end{enumerate}

\begin{exr}[*]
Show that any $\om$-categorical structure has an expansion by adding 
a linear order in such a way that it remains $\om$-categorical.
\end{exr}

\section{Graphs}

The classification of homogeneous\footnote{In the language of graphs.} graphs is due to
Lachlan-Woodrow:
\begin{enumerate}
\item Random graph
\item Empty graph/complete graph
\item $K_n$-free random graphs and complements of these.
\item Disjoint union of cliques of the same size
(includes some finite graphs).
\item $C_5$ (Note $C_4$ is the complement of the equivalence relation with two classes of size $2$, 
so included in the previous one and $C_6$ is not homogeneous)
\item $K_3\tp K_3 = 
\left\{ \left( a,b \right) \st 1\leq a\leq 3, 1\leq b \leq 3 \right\}$ where we put an
edge between $\left( a,b \right)$ and $\left( c,d \right)$ iff $a\neq c$ and $b\neq d$.
\end{enumerate}

\section{Metric spaces}

In this course, a metric space is presented in a binary language
where we have one binary relation for every possible distance.
Specifically the language
\begin{equation}
\left\{ d_r\left( x,y \right) \st r\in \RR_{r\geq 0} \right\}
\end{equation}
such that:
\begin{enumerate}[label=(\iii)]
\item $d_r\left( x,y \right) \ands
d_s\left( y,z \right) \ands d_t\left( x,z \right) \implies r\leq s+t$,
\item $d_r\left( x,y \right) \iff d_r\left( y,x \right)$,
\item $d_0\left( x,y \right) \iff x=y$,
\item $d_r\left( x,y \right) \ands d_s\left( x,y \right) \implies  r = s$,
\item $\forall x,y\bors_{r\in \RR_{\geq 0}} d_r\left( x,y \right)$.
\end{enumerate}
For $S\subeq \RR_{\geq 0}$, an $S$-metric space is the same 
but with only $\left\{ d_r\left( x,y \right) \st r\in S \right\}$.

\begin{exm}
For $S = \left\{ 0,1 \right\}$, this is just equality.
For $S = \left\{ 0,1,2 \right\}$, this is just a graph where
edges are given by distance $1$, nonedges are given by distance $2$, 
and $0$ is equality.
\end{exm}

\begin{qn}
Given $S$, does the class of $S$-metric spaces have amalgamation?
\end{qn}

For $S = \QQ$, the answer is yes, and the Fraiss\'e limit 
(which is not $\om$-categorical) is called the rational Urysohn space.
If $S$ is closed under $+$, then the same. 
\begin{exm}
For $S = \left\{ 0,1,3,4,5 \right\}$ there is no amalgamation:
\begin{equation}
\begin{tikzcd}
\cdot \arrow[dash]{dr}{1}\arrow[dash]{ddr}{3}\arrow[bend left,dotted,no head]{rr}&&
\cdot\arrow[dash]{ddl}{5}\arrow[dash,"1"']{dl}\\
& \cdot \arrow[dash]{d}{4}&\\
& \cdot &
\end{tikzcd}
\end{equation}
since the dotted path cannot be filled in with anything satisfying the triangle
inequality.
\end{exm}

\end{document}

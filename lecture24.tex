\documentclass{amsart}
\usepackage{master}
\begin{document}
\title{Lecture 24\\ Math 229}
\author{Lecture: Professor Pierre Simon\\ Notes: Jackson Van Dyke}
\date{April 23, 2019}
\maketitle

Recall we're still in this situation where $M$ has few substructures.
Then we're studying tameness of definable linear orders.
The last lemma we had was:

\begin{lem}
Let $\oline{V}$ be transitive and linearly ordered. Then
for $f: X\fromto \oline{V}$, $\pi: D\fromto \oline{V}$
we have $X = D$ and $f = \pi$.
\end{lem}

It follows that the induced structure on $B$ is $o$-minimal.\footnote{This means a
definable set is a finite union of intervals.}
We now have a stronger statement which will effectively tell us that $V$ has no
structure.

\begin{prop}
Let $\pi : D\fromto V$ be as above. In particular $V$ is transitive and linear ordered. 
Then $\pi$ has finite fibers and any formula $\phi\left(x_1 , \ldots , x_n\right)$
(defining a subset of $D^n$)
definable over some $\oline{b}$ is equivalent to a Boolean combination of formulas of the
form
\begin{enumerate}[label = (\abc)]
\item $\pi\left(x_i\right)\square\ \pi\left(x_j\right)$, where $\square \in \left\{= ,
\leq\right\}$,
\item $\pi\left(x_i\right) < \pi\left(a\right)$, where $a\in \oline{b}\cap D$, and
\item $x_i = a$ for some $a\in \pi^{-1} \left(\pi\left(\oline{b}\cap D\right)\right)$.
\end{enumerate}
\label{prop:linear}
\end{prop}

\begin{proof}
Let $\oline{c} = \left(c_0 , \ldots , c_{l-1}\right) \in M^l$, and let $v\in V$ be definable
over $\oline{c}$. Then there is some index $i < l$ such that $v = \pi\left(c_i\right)$.
Take $l$ minimal such that this is false. Then let $\oline{c_*} = \left(c_0 , \ldots ,
c_{l-2}\right)$, $\oline{c} \oline{c_*}\land c_{l-1}$. First notice $c_{l-1}$ is not
algebraic over $\oline{c_*}$ 
because otherwise $v$ would be algebraic over $\oline{c_*}$, and
\begin{equation}
v\in \algcl\left(\oline{c_*}\right) = \dcl\left(\oline{c_*}\right)\cap V \ .
\end{equation}
So over $\oline{c_*}$ we have a definable function $f_{\oline{c_*}}$
such that
\begin{equation}
f_{\oline{c_*}}\left(c_{l-1}\right) = v \ .
\end{equation}
Let 
\begin{equation}
W = \left\{w\in V \st \type\left(w / \oline{c_*}\right) = \type\left(v / \oline{c_*}\right)\right\}
\end{equation}
and apply the previous lemma to this (replacing $D$ with $\pi^{-1}\left(W\right)$). So
$f_{\oline{c_*}} = \pi$ and thus $v = \pi\left(c_{l-1}\right)$.
We also have that $\acl\left(\oline{c}\right)\cap D = \pi^{-1}\left(\pi\left(\oline{c}\cap
D\right)\right)$.

We prove the proposition by induction on $n$.
For $n = 1$
$\phi\left(x_1\right)$ is a Boolean combination of convex
subsets of $V$. The end cuts of those convex sets are in 
\begin{equation}
\dcl\left(\oline{b}\right)\cap \oline{V} = \pi\left(\oline{v}\cap D\right)
\end{equation}
so $\phi\left(x_1\right)$ is as desired.

Now we complete the inductive step. Fix values for $x_1 , \ldots , x_{n-1}$ then apply the
case $n = 1$. There are finitely many possibilities for the set $\phi\left(c_1 , \ldots ,
x_{n-1} , x_n\right)$. Now for each of these possibilities
we can define the set of $c_1 , \ldots , c_{n-1}$ for which each possibility is realized.
Hence by induction the formula has the desired form.
\end{proof}

As a consequence of this, the induced structure on $V$ is just DLO. 

\begin{prop}
A similar statement holds for circular orders, i.e. $\pi: D\fromto V$, $V$ transitive, has a
definable circular order.
\label{prop:circular}
\end{prop}

The point is that instead of inequalities we have things like $C\left(\pi\left(x_i\right)
, \pi\left(x_j\right) , \pi\left(c\right)\right)$ for $c\in \pi\left(D\cap
\oline{b}\right)$. 

\begin{proof}
Take $\phi\left(x_1 , \ldots , x_n\right)$ over $\oline{b}$ as in the previous
proposition. If $D\cap \oline{b}\neq \emp$, then over $\oline{b}$, $V$ splits into linear
orders and we can apply \cref{prop:linear}.

In general, pick any $a\in D$, then working over $a$ we can apply \cref{prop:linear}. 
Since we can change $a$, $a$ cannot appear in the decomposition of $\phi$.
\end{proof}

The conclusion is the following. 
\begin{cor}
Let $D$ be definable, $\pi:D\to V$ an interpretable map, where $V$ is transitive,
infinite, and has a definable separation relation. Take $\oline{b}\in M\minus D$. Then $V$
is transitive over $\oline{b}$ and its structure over $\oline{b}$ is precisely one of the
four unstable reducts of DLO.
\end{cor}

\section{Gluing orders}

Say that a definable set $D_{\oline{a}}\subeq M$ (over $\oline{a}$) is \emph{almost
linear (over $\oline{a}$)} if there is $\pi: D_{\oline{a}}\fromto V_{\oline{a}}$
interpretable over $\oline{a}$ and $V_{\oline{a}}$ has an $\oline{a}$-definable linear
order. The point is that $\pi$ might have finite fibers so it might not actually be linear.
We can see $\leq_{\oline{a}}$ as a quasi-order over $D_{\oline{a}}$.

Now we want to see what happens when we vary $\oline{a}$. 

\begin{lem}
Let $D_{\oline{a}}$ be almost linear, transitive over $\oline{a}$, $\pi:
D_{\oline{a}}\fromto V_{\oline{a}}$. Define the equivalence relation $E_{\oline{a}}$ as:
\begin{align}
xE_{\oline{a}} y&& \iff &&\pi\left(x\right) = \pi\left(y\right) \ .
\end{align}
Let $c\in D_{\oline{a}}$. Then for $c'\in M$ the following are equivalent:
\begin{enumerate}
\item $c'\in D_{\oline{a}}$, $E_{\oline{a}}$-equivalent to $c$
\item $c\in \algcl\left(c'\right)$ \ .
\end{enumerate}
\end{lem}

\begin{proof}
$\left(2\right)\implies \left(1\right)$: 
This follows from \cref{prop:linear}.
The point is that if $c$ is algebraic over $c'$ so is its
projection, but the only way a point in $M$ can know a point in $V$ is if it lies above
it.

$\left(1\right)\implies \left(2\right)$: Enumerate the $E_{\oline{a}}$ class of $c$ as
$\oline{c}_0\land \oline{c_1}$ where $\oline{c_0}\in \algcl\left(c\right)$,
$\oline{c_1}\not\in \algcl\left(c\right)$. 
There is $\sigma\in \Aut\left(M\right)$ fixing $c$
and such that 
\begin{equation}
\sigma\left(\oline{c_1}\right)\cap \algcl\left(\oline{a}\land c\right) = \emp \ .
\end{equation}
Let $\oline{a'} = \sigma\left(\oline{a}\right)$. 
Then we claim that $c\in \algcl\left(\oline{a}\land \oline{a'}\right)$, as the set of
$E_{\oline{a}}$-classes which $D_{\oline{a}'}$ intersects non-trivially is finite.
Now $\oline{a}$ must contain a point in the $E_{\oline{a}'}$-class of $c$. Now
this point is algebraic over $\oline{a}\land c$ since it is in $\oline{a}$.
But this means it is in $\oline{c_0}$. But this is not possible since $D_{\oline{a}}$ is
transitive. So we are done.
\end{proof}

Note that $E_{\oline{a}}$ is just the relation of inter-algebraicity over $\emp$.

\begin{lem}
Let $D_{\oline{a}}$ and $D'_{\oline{b}}$ be transitive and almost-linear over $\oline{a}$
and $\oline{b}$ respectively. 
Assume there is $c\in D_{\oline{a}}\cap D'_{\oline{b}}$. Then
$D_{\oline{a}}$ and $D'_{\oline{b}}$ agree up to reversal on an open neighborhood of $c$.
\end{lem}

\begin{proof}
First we know that the $E_{\oline{a}}$-class of $c$ and the $E'_{\oline{b}}$-class of $c$
coincide. 
We claim that $c\not\in \acl\left(\oline{a}\land \oline{b}\right)$. 
Working over $\oline{a}$, if $c\in \algcl\left(\oline{a}\land\oline{b}\right)$, then
$\oline{b}$ has a point in the $E_{\oline{a}}$-class of $c$. This is the
$E_{\oline{b}}$-class of $c$, so this would contradict transitivity of $D'_{\oline{b}}$.

It follows that $c$ is not an endpoint of the intersection (in $D_{\oline{a}}$). 
Therefore there is an open neighborhood of $c$ in $D_{\oline{a}}$ inside the intersection.
Then the order induced on this intersection by $D'_{\oline{b}}$ must either be the original order, or
the reversed order. 
More specifically we can work in a small neighborhood of $c$ which doesn't intersect
$\acl\left(\oline{a}\land\oline{b}\right)$ so $\oline{b}$ cannot do anything besides
reverse the order in this neighborhood.

Now repeat the same argument for $\oline{b}$ and it is easy to see we are done.
\end{proof}

\end{document}

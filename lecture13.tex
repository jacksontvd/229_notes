\documentclass{amsart}
\usepackage{master}
\begin{document}
\title{Lecture 13\\Math 229}
\author{Lecture: Professor Pierre Simon\\Notes: Jackson Van Dyke}
\date{March 5, 2019}
\maketitle

\section{$G$-flows}

We will continue discussing $G$-flows.

\begin{defn}
Let $X$ be a $G$-flow. A \emph{factor} of $X$ is a $G$-flow $Y$ for which there is a
surjective morphism $\pi : X\fromto Y$, i.e. a continuous map respecting the $G$ action.
\end{defn}

The idea is that $X$ is `at least as complicated' as $Y$. 
\begin{rmk}
Note that if $f:X\fromto Y$ is a morphism of $G$ flows and $Y$ is minimal, then $f$ is
surjective.
\end{rmk}

\begin{defn}
A \emph{universal minimal flow} is a minimal $G$-flow $X$ which has every minimal $G$-flow
as a factor.
\end{defn}

\begin{thm}
There is a unique universal minimal $G$-flow.
\end{thm}
\begin{proof}
Interestingly enough, uniqueness is harder to show than existence here.
We deal with existence first. Note that if $X$ is a minimal $G$-flow, then it has a dense orbit of
size at most $\abs{G}$, so $\abs{X}\leq 2^{2^{\abs{G}}}$.

Let $\left( X_i , i<\al \right)$ be an enumeration of all minimal $G$-flows (up to
isomorphism). 
Take
\begin{equation}
X_* = \prod_{i < \al} X_i
\end{equation}
with the coordinate-wise action of $G$. Note the product of compact spaces is compact so
this is a $G$-flow. Let $\tilde X \subeq X_*$ be a minimal subflow.
Then $\tilde X$ admits each $X_i$ as a factor, so by the above remark, it is a universal
minimal flow.

The universal property here somehow doesn't have uniqueness built in (as it usually is),
so uniqueness isn't so easy here. The formal proof involves introducing new machinery, so
we just sketch the proof. It is enough to find some coalescent minimal flow $Z$.
A flow is coalescent if every endomorphism is an automorphism.
For a minimal flow this is just saying it is injective since every map to it is already
surjective.
This is enough because if $M$ and $M'$ are universal minimal flows, and $Z$ is coalescent,
then we have a sequence of maps $Z\fromto M \fromto M' \fromto Z$ and the composition has
to be an automorphism, and since they're all already surjective, they are all
automorphisms.
\end{proof}

Note that $G$ is extremely amenable iff its universal minimal flow is a point.
So the universal minimal flow is supposed to be a measure of how far a flow is from being
extremely amenable.

\section{Back to \Fraisse classes}

Let $L_0$ be a language and let $L = L_0\un \left\{ \leq \right\}$ where $\leq \not\in L$.
Let $K_0$ be a \Fraisse class in $L_0$ and $K$ be a \Fraisse class in $L$.
We want some notion of $K$ being an expansion of $K_0$.

\begin{defn}
Say that $K$ is \emph{reasonable} if:
\begin{enumerate}
\item the $L_0$ reduct of $K$ is $K_0$, and
\item if $A\leq B$ in $L_0$ and $A'$ is an expansion of $A$ in $K$ then there is an
expansion $B'$ of $B$ in $K$ such that $A'\leq B'$.
\end{enumerate}
\end{defn}

\begin{defn}
Let $M_0$ be the \Fraisse limit of $K_0$. Then a \emph{$K$-admissible} ordering on $M_0$
is an expansion of $M_0$ to $L$ whose age is included in $K$.
\end{defn}

Note that this is not the \Fraisse limit of $K$, which would not be compact. 
A $K$-admissible ordering is naturally an $\Aut\left( M_0 \right)$-flow, so in particular
it is compact. 

\begin{defn}
Say that $K$ has the \emph{ordering property} (relative to $K_0$) if for every $A\in
K_0$, there is $B\in K_0$ such that for any $A',B'\in K$ expanding $A$ and $B$
respectively, we have $A'\subeq B'$.
\end{defn}

\begin{thm}[KPT]
With notation as above, let $X_K$ be the space of $K$-admissible orderings. Then TFAE:
\begin{enumerate}[label=(\iii)]
\item $K$ has the Ramsey and ordering properties.
\item $X_K$ is the universal minimal $\Aut\left( M_0 \right)$-flow.
\end{enumerate}
\end{thm}

\begin{exm}
The universal minimal flow of $S_\infty = \Aut\left( \om , = \right)$ is the space of
linear orders on $\om$.
This was first proved by Glasner-Weiss.
\end{exm}

Note that with the assumptions of the above theorem, the universal minimal flow is
metrizable. In general the compact space $2^{X}$ with $X$ countable is metrizable.
The metric is as follows.
Fix a bijection $X\simeq \om$ and then the distance between $\left( a_i , i < \om \right)$
and $\left( b_i, i < \om \right)$ is $2^{-m}$ where $m$ is maximal such that $a_i = b_i$
for $i < m$.

Extremely amenable means the universal minimal flow is a point, and having metrizable
universal minimal flow says that it is still somehow small.
For example, discrete groups don't have metrizable universal minimal flows unless they are finite.

\begin{defn}
Let $K_0$ be a \Fraisse class. 
Then $A\in K_0$ has Ramsey degree $d$ if 
for every $B\in K_0$
there is $C\in K_0$
such that for any coloring
$f:\binom{C}{A} \fromto r$ there is a copy $B'$ of $B$ in $C$ such that $f$ takes at most
$d$ values on $\binom{B'}{A}$. 

We say $K_0$ has finite Ramsey degree if for every $A\in K_0$, there is $d<\om$ such that $A$ has
finite Ramsey degree $d$.
\end{defn}

Note that $K_0$ has finite Ramsey degree for embeddings iff it has finite Ramsey degree
for substructures.

Also observe that if there is $K$ which is Ramsey and expands $K_0$ reasonably by finitely many relation
symbols, then $K_0$ has finite Ramsey degree and it is bounded by the number
of expansions of $A$ to a structure in $K$.
The converse is not clear, but it turns out to be true.

\begin{thm}[Zucker]
Let $K_0$ be a \Fraisse class (homogeneous and locally finite). Then TFAE:
\begin{enumerate}[label = (\iii)]
\item $\Aut\left( M_0 \right)$ has metrizable universal minimal flow (where $M_0$ is the
\Fraisse limit of $K_0$).
\item $K_0$ has finite Ramsey degree.
\item There is a Fraisse class $K$ which is Ramsey, is a reasonable expansion of $K_0$ obtained by
adding finitely many relational symbols, (and has the expansion property).\footnote{This
is true with or without this last piece.}
\end{enumerate}
\end{thm}

Note that the universal minimal flow of $\ZZ$ 
is a minimal flow of the Stone-\v{C}ech compactification $\b \ZZ$ which
is not metrizable.
In general locally compact groups which are not compact have non-metrizable universal
minimal flow.

\begin{qn}
If $K$ is a \Fraisse class with $\om$-categorical limit does it have metrizable universal
minimal flow?
\end{qn}

The answer is no in general. But it is open in the finitely homogeneous case, i.e. $L$ is
finite relational.

\subsection{Hrushovski construction}

We offer an example of
a \Fraisse class with $\om$-categorical limit which does not have metrizable universal
minimal flow.

\begin{thm}[Hrushovski]
There exists an $\om$-categorical graph $G$ which is $2$-sparse, i.e. 
for any finite $A\subeq G$, 
\begin{equation}
\abs{E\left( A \right)}\leq 2 \abs{V\left( A \right)}
\end{equation}
and every vertex has infinite degree.
\end{thm}

If we remove any one of these conditions this is easy. If we remove the last, we 
can just take the empty graph. 
If we remove the $\om$-categorical condition we can just use a tree.
But this is not $\om$-categorical.
We now sketch why this thing doesn't have an $\om$-categorical Ramsey expansion.

\begin{thm}[Evans]
Such a structure does not have an $\om$-categorical Ramsey expansion, and hence does not
have metrizable universal minimal flow.
\end{thm}

\begin{proof}
We will find a universal theory such that no completion of it can be $\om$-categorical,
since we can define distances. 
The universal theory will be as follows. 
We will orient the edges with the assumption that there are at most $2$ outgoing edges
from every vertex. We call this a $2$-orientation
For example, if we have a tree, we can choose a root, and orient every edge towards this root.

The following essentially follows from the Hall marriage theorem:
\begin{fact}
A graph has a $2$-orientation iff it is $2$-sparse.
\end{fact}

An $\om$-categorical Ramsey expansion of $G$ has to define a $2$-orientation (as this is a
universal theory). But this implies there are infinitely many $2$-types. The idea is that
if we follow the orientations, every point we can reach from a point $a$ is in the
algebraic closure of $a$, and this must end at some point. Write the size of the closure
as $k$. Assume there is some bound, and take $a$ which maximizes this $k$. $a$ has
infinite degree, so it must have some point $b$ not reachable from $a$ which maps into
it, which means the closure of $b$ contains the closure of $a$, but it also
contains $a$ so the size of the closure of $b$ is at least $k+1$ which is a contradiction
to $a$ having maximal closure.
\end{proof}

\end{document}

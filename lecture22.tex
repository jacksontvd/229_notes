\documentclass{amsart}
\usepackage{master}
\begin{document}
\title{Lecture 22\\ Math 229}
\author{Lecture: Professor Pierre Simon\\ Notes: Jackson Van Dyke}
\date{April 16, 2019}
\maketitle

\section{Model theoretic properties of structures}

Today we will mostly state things and not prove much, and then we will do the NIP case in
more detail. 

\subsection{Counting substructures}

To motivate the introduction we start with the combinatorial question from last time of
counting the number of substructures.

Assume the language is relational.
Let $M$ be a $\om$-categorical and $G = \Aut\left(M\right)$. 
Define $f_n\left(M\right)$ to be the number of substructures of $M$ of size $n$ up to
isomorphism. We could also define some other function (which we will not be using) that we
just mention. First $F_n\left(M\right)$ is the number of $n$-types over $\emp$, and
$\tilde F_n\left(M\right)$ is the number of $n$-types of pairwise distinct elements.
The difference between the two is somehow that
in $f$ we somehow don't have an ordering on the things we are counting so we have
\begin{equation}
f_n\left(M\right) \leq \tilde F_n\left(M\right) < F_n\left(M\right) \ .
\end{equation}
If substructures are rigid (e.g. we have a linear order) then 
\begin{equation}
f_n\left(M\right) = \frac{\tilde F_n\left(M\right)}{n!}\ .
\end{equation}

The problem of $\tilde F_n\left(M\right)$ is that this thing somehow grows too quickly
whereas $f_n$ has a slower growth rate so it's somehow more interesting to the
combinatorialists. 

\begin{exm}
For DLO, $f_n\left(M\right) = 1$.
\end{exm}

\begin{exm}
For $M$ the random graph we roughly have that 
\begin{equation}
f_n\left(M\right)\approx F_n\left(M\right) \approx 2^{n^2} 
\end{equation}
since this is somehow asking for graphs with $n$ vertices.
\end{exm}

\begin{exm}
Let $L = \left\{\leq\right\}$ be a partial order then a tree is when this is actually a
linear order. Then
$f_n\left(M\right)$ is the number of trees on $n$ vertices, so it's $c^n$ for some
constant $c$ which is between $2 < c < 3$. This is some kind of Catalan number. 
Note that this is smaller than a factorial. 

If we take two linear orders $L = \left\{\leq_1 , \leq_2\right\}$ then 
$f_n\left(M\right) = n!$.
This is why these things are sometimes called permutation structures.
\end{exm}

So we have seen constant growth rates, exponential, and factorial growth rates. 
Now we have some basic facts. 

\begin{fact}[Cameron]
\begin{enumerate}[label = (\iii)]
\item $f_n$ is non-decreasing.
\item $f_n$ can grow as fact as one wants (with an infinite language).
\item If the language has max arity $k$ and $M$ is homogeneous in $L$ then
$f_n\left(M\right) = \cO\left(e^{n^k}\right)$.
\end{enumerate}
\end{fact}

\begin{qn}
Understand structures with slow growth rate of $f_n\left(M\right)$.
\end{qn}

\begin{fact}[Cameron]
If $f_n\left(M\right) = 1$ for all $n$ then $M$ is one of the five reducts
of DLO, i.e. one of the following:
\begin{enumerate}[label = (\iii)]
\item $\left(\from\right)$: DLO
\item $\left(\bij\right)$: Betweenness relation $B\left(x,y,z\right)$
\item $\left(\acts\right)$: Circular order $C\left(x,y,z\right)$
\item $\left(\ \dsum\ \right)$: Separation relation $S\left(x,y,u,r\right)$
\item Pure equality
\end{enumerate}
\end{fact}

\begin{rmk}
This class is a priori included in the reducts of DLO (since $f_n$ only decreases when one
takes reducts) so this fact was shown by showing the other inclusion manually on each
reduct.
\end{rmk}

\begin{fact}[Macpherson]
If $M$ is primitive, either $f_n\left(M\right) = 1$ for all $n$ or 
\begin{equation}
f_n\left(M\right) \geq\frac{c^n}{p\left(n\right)}
\end{equation}
for $c = 2^{1/5}\approx 1.148$ and $p$ is some polynomial depending on $M$.
\end{fact}

\begin{con}
In fact, it is true with $c = 2$.
\end{con}

\begin{exm}
Note that $2$ is possible.
Take a circular order and add a relation where from every point we add an arrow to the
other half of the circle. This should be thought of as a topological $2$-cover of the
circle. 
\end{exm}

\subsection{Model theoretical notions}

There is a model theoretic property (NIP) 
which is implied by a small growth rate of $f_n$.

\begin{defn}
A formula $\phi\left(\oline{x} , \oline{y}\right)$
has IP (in $M$) if for all $n < \om$ we can find tuples 
\begin{align}
\left(\oline{a}_i \st i < n\right) &&
\left(\oline{b}_j \st j\in \cP\left(n\right)\right)
\end{align}
in $M$ such that 
\begin{equation}
M\models \phi\left(\oline{a}_i , \oline{b}_j\right) \iff i\in j \ .
\end{equation}
\end{defn}

\begin{defn}
We say $M$ is NIP if no formula has IP.
\end{defn}

So IP somehow gives us something complicated and it has NIP if this doesn't happen. 

\begin{exm}
The graph relation $R\left(x , y\right)$ has IP
in $M$ the random graph. 
\end{exm}

\begin{exm}
Higher order things like DLO and trees are NIP. 
NIP formulas are closed under Boolean combinations. 
\end{exm}

\begin{lem}[Sauer-Shelah]
If $\phi\left(\oline{x}, \oline{y}\right)$ is NIP then for finite $A\subeq M^{\oline{y}}$
there is some $k$ such that
\begin{equation}
\abs{S_\phi\left(A\right)} = \cO\left(\abs{A}^k\right)
\end{equation}
where $S_\phi\left(A\right)$ consists of the $\phi$-types over $A$, i.e. the maximal
consistent set of formulas of the form
$\phi\left(x , \oline{b}\right)$ where $\oline{b}\in A$.
\end{lem}

Note if $\phi$ has IP
\begin{equation}
\abs{S_\phi\left(A\right)} = 2^{\abs{A}}
\end{equation}
for some $A$ of arbitrarily large size.

This has the following consequence. Assume $M$ is homogeneous in a finite relational
language and every relation in $L$ is NIP (for every partition of the variables) then
\begin{equation}
f_n\left(M\right) = \cO\left(e^{cn\ln n}\right)
\end{equation}
for some $c > 0$ so its growth rate is at most factorial.

For simplicity assume we have $1$-formula. Consider some $n$-elements $a_1 , \ldots , a_n$.
We might as well count $F_n\left(M\right)$ since the factorial won't change whether or not
we satisfy this bound. 
Then the number of $n$-types over $\emp$ is at most
\begin{equation}
S_n\left(\emp\right)\leq S_1\left(\emp\right)
\cdot S_1\left(\left\{a_0\right\}\right) \cdot S_1\left(\left\{a_0 , a_1\right\}\right)
\cdot \ldots
\end{equation}
and by NIP this is a product which looks like:
\begin{equation}
S_n\left(\emp\right) \leq 1^k \cdot 2^k \cdot 3^k \cdot \ldots \cdot n^k \leq \left(n!\right)^c
\end{equation}
for some $c$.

\begin{fact}[Macpherson]
If some $\phi\left(\oline{x} , \oline{y}\right)$ has IP then 
\begin{equation}
f_n\left(M\right) \geq 2^{p\left(n\right)}
\end{equation}
for some polynomial $p$ of degree at least $2$.
\end{fact}

For finitely homogeneous structures we get a gap that either 
\begin{equation}
f_n\left(M\right) = \cO\left(e^{cn \ln n}\right)
\end{equation}
where $M$ is NIP, and
\begin{equation}
f_n\left(M\right)\geq e^{p\left(n\right)}
\end{equation}
(for $\deg p\geq 2$) in the IP case.
So the picture is:
\begin{equation}
\ubr{
1 
\qquad\vert\text{gap}\vert \qquad
\ubr{e^{cn}}{\text{trees}}
\quad\ldots\quad
\ubr{e^{cn\ln n}}{2\text{ lin ord}}
}{\text{NIP}}
\qquad\vert\text{gap}\vert \qquad
e^{cn^2} \ldots
\end{equation}

New goal: understand NIP for finite homogeneous (or $\om$-categorical) structures.

\subsection{Stability}

\begin{defn}
$\phi\left(\oline{x} , \oline{y}\right)$ is unstable (in $M$) if for every $n$ there are
$\left(\oline{a}_i \st i < n\right)$
$\left(\oline{b}_j \st j < n\right)$
such that
\begin{equation}
M\models \phi\left(\oline{a}_i , \oline{b}_j\right) \iff i\leq j \ .
\end{equation}
\end{defn}

\begin{defn}
$M$ is stable if all formulas are stable. 
\end{defn}

Note that stable implies NIP.

\begin{exm}
$L = \left\{E\right\}$ where $E$ is an equivalence relation then $M$ is stable. 
DLO is NIP but not stable.
\end{exm}

\begin{exm}
$\FF_p$ vector spaces are stable $\om$-categorical but not finitely homogeneous.
\end{exm}

\begin{enumerate}
\item If $M$ is finitely homogeneous and stable then it is $\om$-stable.
\item $\om$-categorical and $\om$-stable structures are very understood. They are all
build out of $\left(M , =\right)$, $\FF_q$ vector spaces by families of covers.
More precisely:
\end{enumerate}

\begin{fact}

\begin{itemize}
\item A primitive $\om$-categorical $\om$-stable is a Grassmannian over one of:
\begin{itemize}
\item $\left(M , =\right)$ 
\item Affine or projective space over a finite field.
\end{itemize}
\item A finite homogeneous stable primitive structure is interdefinable with unordered
$k$-tuples of distinct elements of $\left(M , =\right)$.
\end{itemize}
\end{fact}

Once we know that these are the primitive ones, one can just check what happens with
$f_n$. For $\left(M , =\right)$, $f_n = 1$. The $2$-Grassmannian of $\left(M , =\right)$,
then $f_n$ is roughly the set of graphs with $n$ edges. 
This is hard to compute, but certainly it grows faster than $c^n$ but the actual
asymptotic behavior is not known.

In the case of the $\FF_p$ vector space $f_n$ looks like the following. 
This has at least the number of partitions, so $f_n$ also grows faster than $c^n$.

The conclusion is that if $\cM$ is $\om$-stable, primitive, and $f_n = \cO\left(c^n\right)$,
then $\cM \simeqq \left(M , =\right)$.
So then the unstable case is left which we will deal with next time. 

Stable not $\om$-stable things are not understood at all.
Pseudo-planes can get some lower bounds of $f_n$ by powers of $\sqrt{n}$
but even powers of $n$ hasn't been done. 

\end{document}
